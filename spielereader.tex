\documentclass[a4paper,twoside,11pt,titlepage,openany]{scrbook}
\usepackage[utf8]{inputenc}
\usepackage[plainpages=false,pdfpagelabels]{hyperref}
\usepackage{a4wide,ae,avant,graphics,makeidx,marvosym,url}

\usepackage[american,ngerman]{babel}
\usepackage[babel]{csquotes}

\newcommand{\fett}[1]{\textsf{\textbf{#1}}}

% weniger Abstand unter Paragraph
\makeatletter
\renewcommand{\paragraph}{
  \@startsection{paragraph}{4}
  {\z@}{1ex}{-0.5em}
  {\fett}
}
\makeatother

\renewcommand{\descriptionlabel}[1]{\hspace*{1em} \hspace\labelsep \fett{#1}}

\author{\raggedright Oliver Klee | \url{http://www.spielereader.org/} | \texttt{@oliklee}}
\title{Spiele und Methoden für Workshops, Seminare, Erstsemestereinführungen oder einfach so zum Spaß}
\date{Version vom 13.\,10.\,2013}

\pagestyle{headings}
\raggedbottom
\sloppy
\bibliographystyle{alpha}
\hyphenation{
Be-ar-bei-ten
Dau-men-wrest-ling
down-loa-den
Down-load-ge-schwin-dig-keit
Ein-stel-lun-gen
Fach-sch
Fach-schaf
Fach-schaft
Fens-ter
Flip-chart
hoch-auf-lö-sen-de
In-for-ma-ti-o-nen
Kre-dit-kar-ten
markt-schrei-e-ri-scher
Me-nü-struk-tur
Mo-de-ra-ti-ons-stif-te
Nag-ware
News-rea-der
Sau-er-stoff-tanks
schar-fe
Staats-an-ge-hö-rig-keit
The-a-ter
ü-ber-flieg-bar
Vek-tor-zei-chen-pro-gramm
ver-bes-ser-te
ver-lang-sa-men
ver-schie-den-ar-ti-ge
Wein-gum-mi
wo-rauf-hin
Zei-ge-fin-ger
}


\makeindex

\setlength{\parindent}{1em}
\setlength{\parskip}{0em}
\setlength{\parsep}{0em}

\begin{document}
% einfach mal die komplette Literatur im Verzeichnis aufführen
\nocite*{}

\frontmatter
\maketitle
\tableofcontents

\mainmatter

\chapter{Einleitung}
\section{Willkommen zum Spiele- und Methodenreader!}

Wenn ihr auf einem Seminar spielen wollt, dann tut das von Anfang an. Wenn ihr erst später mit Spielen anfangt, sind die Teilnehmerinnen oft nur noch schwer dazu zu bewegen, zu laufen oder alberne Dinge zu tun. Diskutiert außerdem nicht, ob gespielt wird oder nicht, sondern verbreitet einfach Enthusiasmus und Motivation und reißt die TeilnehmerInnen mit. Wer nicht mitspielen möchte, kann trotzdem nicht gezwungen werden.

Spielt mit den TeilnehmerInnen möglichst nur die Spiele, die ihr schon einmal ausprobiert habt. Dann klappt's mit dem Erklären besser, und ihr kennt den Ablauf schon. Ausnahmen bestätigen wie üblich die Regel (wenn ihr schon sehr viel Erfahrung mit Spielen und Seminaren habt oder wenn ihr einfach neugierig seid).

Wenn ihr ein Spiel für ein Seminar plant, sucht euch auch noch ein, zwei Reservespiele aus, falls ihr spontan feststellt, das das Spiel doch nicht ganz in die entsprechende Situation passt. (Das passiert erstaunlich oft.)

Spielt viel und oft! Auf der OE-Vorbereitungsfahrt 1999 fanden die Leute die meisten Spiele sehr spaßig. Und falls noch ihr zweifeln solltet, ob eure TeilnehmerInnen wirklich mitspielen werden: Auf einem Java-Seminar habe ich sogar einen Haufen spießige BankerInnen zum Spielen bewegen können.

\emph{Verdammt, das ist ja fürchterlich. Ich schmeiße jetzt alle geschlechtsneutralen Doppelformen raus und nehme an den üblen Stellen die weibliche Form. Männchen und Weibchen mögen sich bitte gleichermaßen angesprochen fühlen.}

\section{Wo kommen die Spiele und Methoden her?}
Die meisten Spiele und Methoden habe ich auf folgenden Veranstaltungen kennen gelernt:
\begin{itemize}
  \item Seminare des \emph{Vereins zur Förderung politischen Handelns (v.\,f.\,h.)}, \url{http://www.vfh-online.de/}
  \item Seminare der \emph{Werkstatt für Gewaltfreie Aktion, Baden}, \url{http://www.wfga.de/}
  \item Konferenz der deutschsprachigen Informatikfachschaften (KIF), \url{http://kif.fsinf.de/}
  \item Seminare, die ich mit anderen Menschen zusammen geleitet habe
\end{itemize}

\section{Unter welchen Bedigungen könnt ihr den Spielereader benutzen?}
Dieser Reader ist unter einer \emph{Creative-Commons}-Lizenz lizensiert, und zwar unter der \emph{Namensnennung-Weitergabe unter gleichen Bedingungen License Deutschland}. Das bedeutet, dass ihr den Reader unter diesen Bedingungen für euch kostenlos verbreiten, bearbeiten und nutzen könnt (auch kommerziell):
\begin{description}
  \item[Namensnennung.] Ihr müsst den Namen des Autors (Oliver Klee) nennen. Wenn ihr außerdem auch noch die Quelle (also \url{http://www.spielereader.org/}) nennt, wäre das nett.
  \item[Weitergabe unter gleichen Bedingungen.] Wenn ihr diesen Inhalt bearbeitet oder in anderer Weise umgestaltet, verändert oder als Grundlage für einen anderen Inhalt verwendet, dann dürft ihr den neu entstandenen Inhalt nur unter Verwendung identischer Lizenzbedingungen weitergeben.
  \item Wenn ihr den Reader weiter verbreitet, müsst ihr dabei auch die Lizenzbedinungen nennen oder beifügen.
\end{description} 

Die ausführliche Version dieser Lizenz findet ihr unter \url{http://creativecommons.org/licenses/by-sa/2.0/de/}.

\section{Wie könnt ihr Lob und Kritik loswerden?}
Wenn ihr Lob, Kritik, Korrekturen\footnote{bitte auch zu Tippfehlern, Sprachgurken und inhaltlichen Fehlern}, Anregungen oder sonstige Kommentare habt, würde ich mich über eine E-Mail an \texttt{oliver@spielereader.org} freuen.

Ich wünsche euch viel Spaß beim Spielen!


\chapter{Kennenlernspiele}
\index{Kennenlernspiele}

\section{Standpunkte}
\index{Standpunkte}
\index{Aufstellung|see{Standpunkte}}
\index{Menschenaufläufe|see{Standpunkte}}
\label{standpunkte}

\paragraph{Alias:} Aufstellung, Menschenaufläufe
\paragraph{Art:} sehr transparente Gruppenbildung nach inhaltlichen Kriterien
\paragraph{Ziel:} Aufzeigen von Gemeinsamkeiten, inhaltliches Gruppieren
\paragraph{Dauer:} 1--5 Minuten
\paragraph{Wir brauchen dazu:} ein bisschen Platz
\paragraph{So geht es:} Die Tutorin gibt vor, welcher Platz was bedeutet, zum Beispiel:
  \begin{itemize}
    \item am Fenster: will Thema 1 bearbeiten
    \item in der rechten hinteren Ecke: will Thema 2 bearbeiten
    \item an der Tür: will Thema 3 bearbeiten
  \end{itemize}
  Dann stellen sich die Teilis entsprechend ihrer Interessen auf. Es wird schnell sichtbar, wie viele sich für welches Thema interessieren, und ob eine Gruppe möglicherweise zu klein oder zu groß wird.
\paragraph{Varianten:}
  \begin{description}
    \item[Meinungsbild:] Stimme ich der These zu oder nicht?
    \item[Meinungsbild mit Zwischenstufen:] Die Teilis stellen sich auf einer gedachten Linie auf. Die beiden Endpunkte stellen jeweils die Extrempositionen dar, alles dazwischen sind entsprechend ein abgestuftes "`Sowohl-als-auch"'.
    \item [Orgakram:] Kaffee, Tee oder Kakao zum Frühstück? Vegetarisch oder totes Tier?
    \item[Sortieren:] siehe nächstes Spiel
  \end{description}
\paragraph{Wann einsetzen:} Um Gruppen nach Interesse zu bilden.

\section{Sortieren}
\index{Sortieren}
\label{sortieren}
\paragraph{Art:} Sortierspiel
\paragraph{Ziel:} die Teilis lernen sich in Bezug auf eine Eigenschaft kennen und mischen sich gleichzeitig
\paragraph{Dauer:} 1--5 Minuten
\paragraph{Wir brauchen dazu:} ein bisschen Platz
\paragraph{So geht es:} Die Tutorin gibt vor, wonach sich die Teilis sortieren, zum Beispiel:
  \begin{itemize}
    \item nach Matrikelnummern
    \item nach Anzahl der Semester
    \item nach Länge des Anreiseweges
    \item nach Anzahl der bisher besuchten Seminare
  \end{itemize}
\paragraph{Wann einsetzen:}
\begin{itemize}
  \item um die Gruppe neu zu mischen (etwa für eine zweite Runde von \emph{Mirko Mondsüchtig} (nächstes Spiel, Seite \pageref{mirko}))
  \item um das Kennenlernen in Bezug auf eine bestimmte Eigenschaft zu fördern und etwas Bewegung in die Gruppe zu bringen
\end{itemize}

\section{Mirko Mondsüchtig}
\label{mirko}
\index{Mirko Mondsüchtig}
\index{Alliterationen|see{Mirko Mondsüchtig}}
\index{Adjektiv-Name|see{Mirko Mondsüchtig}}
\paragraph{Alias:} Alliterationen, Adjektiv-Name
\paragraph{Art:} ruhiges, sehr lustiges Namenslernspiel
\paragraph{Ziel:} Namen hören und durch Assoziation mit Worten einprägen
\paragraph{Dauer:} bei $n$ Spielerinnen etwa $\frac{n^2}{10}$ Minuten
\paragraph{Wir brauchen dazu:} Platz für einen Stehkreis; 5-15 Spielerinnen (darüber dauert es zu lange)
\paragraph{So geht es:} Einen Kreis bilden. Reihum nennt jede einen Spruch wie "`Ich bin der jubelnde Julian"' oder "`Ich bin die singende Sabine"': Das Adjektiv soll also mit dem gleichen Buchstaben (oder Laut) beginnen wir der eigene Vorname. Dazu macht die Spielerin eine passende Geste oder Bewegung. Wer dran ist, wiederholt die vorherigen Sprüche und Bewegungen, bevor sie den eigenen bringt~-- ganz ähnlich dem Spiel \emph{Kofferpacken}.
\paragraph{Besondere Hinweise:} Durch die Verknüpfung mit anderen Eindrücken (Worte, Bewegungen) und Wahrnehmungen verbessert dieses Spiel die Gedächtnisleistung bei den Namen.
\paragraph{Varianten:} Wenn ihr einen besonders guten Lerneffekt habt, dann spielt zwei Runden, und verändert zwischen beiden Runden die Reihenfolge~-- zum Beispiel, indem ihr sie sich nach Matrikelnummern sortieren lasst (siehe die \emph{Sortieren} auf Seite \pageref{sortieren}).
\paragraph{Wann einsetzen:} Wenn die Gruppe die Namen noch nicht oder kaum kennt.

\section{Zipp-Zapp}
\index{Zipp-Zapp}
\paragraph{Art:} Schnelles, actionreiches Namenswiederholspiel
\paragraph{Ziel:} bereits gelernte Namen schnell wiederholen
\paragraph{Dauer:} 5--10 Minuten
\paragraph{Wir brauchen dazu:} ---
\paragraph{So geht es:}
Wieder sitzen alle im Kreis, nur diesmal eine in der Mitte. Diese dreht sich herum, zeigt auf Personen und sagt:
\begin{itemize}
\item Zipp: Betreffende Person nennt den Namen der linken Nachbarin, oder auch nicht
\item Zapp: Name der rechten Nachbarin
\item Zipp-Zapp: Alle tauschen die Plätze
\end{itemize}
Die angesprochene Person muss schnell reagieren und den richtigen Namen nennen! Gelingt ihr das nicht, muss sie in die Mitte. Dadurch ändert es sich schnell, wen man zum Nachbarn hat.
\paragraph{Wann einsetzen:} Wenn die Gruppen die Namen schon einmal gehört hat.

\section{Partnerinneninterview}
\index{Partnerinneninterview}
\paragraph{Art:} klassisches Kennenlernspiel über die Eigenschaften der Leute
\paragraph{Ziel:} Partnerin der Gruppe vorstellen
\paragraph{Dauer:} 20 Minuten plus (Teilnehmerinnen$\cdot$2) Minuten
\paragraph{Wir brauchen dazu:} Schreibzeug für alle, einen Sitzkreis
\paragraph{So geht es:}
Die Spielerinnen finden sich zu Pärchen zusammen (oder werden von der Seminarleiterin eingeteilt), die sich im Raum oder Gebäude verteilen. Beide Teile jedes Pärchens interviewen sich gegenseitig je 10 Minuten (also insgesamt 20 Minuten pro Pärchen).

Die Interviewerin kann alles fragen, was sie interessiert: Namen, Wohnort, Arbeit, Alter, Hobbys, Erwartungen, Haustiere, Anekdoten, Erlebnisse, \ldots\ Die Leute können sich auch ruhig ein paar privatere Fragen stellen (zum Beispiel nach dem Beziehungsstatus). Wenn es für das Seminar sinnvoll ist, kann die Seminarleiterin auch vorher ein paar Leitfragen anschreiben, an denen sich die Spielerinnen orientieren können. Beispiele:
  \begin{itemize}
    \item Was ging dir auf dem Weg hierher durch den Kopf?
    \item Was würdest du tun, wenn Geld keine Rolle spielte?
    \item Als was für ein Tier wärst du geboren worden?
    \item Was möchtest du in 5--10 Jahren sein?
    \item Was für Erwartungen hast du an das Seminar?
  \end{itemize}

Beim Interview kann es hilfreich sein, sich die Fakten aufzuschreiben.

Nach der Interviewphase kommen die Spielerinnen wieder zum Stuhlkreis zusammen. Die Spielerinnen stellen nun zwei Minuten lang ihre jeweilige Partnerin der Gruppe vor (und entsprechend umgekehrt natürlich auch).

Nach jeder Vorstellung fragt die Seminarleiterin, ob die Vorgestellte sich gut dargestellt findet. Wenn dem so ist, geht es mit einem Applaus\footnote{Sehr wichtig für das Selbstbewusstsein der Spielerinnen. Außerdem lässt sich für ein Rhetorikseminar der Applaus an dieser Stelle sehr gut einführen.} und der nächsten Vorstellung weiter.
\paragraph{Besondere Hinweise:} Weist die Spielerinnen darauf hin, dass sie selbst auf die Zeit achten sollen, damit beide Interviews etwa gleich lang werden.

Die Interviews können auch auf einem Spaziergang stattfinden. Dann wird es allerdings mit dem Aufschreiben schwierig.
\paragraph{Varianten:}
\begin{itemize}
  \item Lügen-Porträt (nächstes Spiel, Seite~\pageref{luegenportrait})
  \item Pöstchenvergabe (Seite~\pageref{poestchenvergabe})
  \item \emph{vor} dem Interview noch nicht erwähnen, dass die Spielerinnen danach ihre Partnerin vorstellen sollen~-- so wird das Interview viel persönlicher
\end{itemize}

\paragraph{Wann einsetzen:} Wenn die Leute die Namen schon halbwegs kennen und sich jetzt etwas detaillierte kennen lernen sollen. Auch gut als erste Präsentationsübung auf einem Rhetorikseminar geeignet.

\section{Lügen-Porträt}
\index{Lügen-Porträt}
\label{luegenportrait}
\paragraph{Art:} ruhiges Kennenlern-Ratespiel über Eigenschaften der Leute
\paragraph{Ziel:} Partnerin mit wahren und erfundenen Informationen vorstellen
\paragraph{Dauer:} 30--45 Minuten
\paragraph{Wir brauchen dazu:} Schreibzeug für alle, leere Plakate und Moderationsstifte
\paragraph{So geht es:} Funktioniert wie das Partnerinneninterview. Bei der Vorstellungsphase gibt es allerdings Unterschiede:
Die Interviewerin berichtet der Gruppe die vier interessantesten Einzelheiten über die Interviewte. Die Tutorin kann diese Einzelheiten auf einem Plakat visualisieren, da sich die Erstis erfahrungsgemäß nicht alles merken können.

Eine Einzelheit soll dabei "`gelogen"' (von der Interviewerin erfunden) sein. Die ganze Gruppe soll dann raten, welches die erfundene Information war.
\paragraph{Wann einsetzen:} Wenn die Gruppe die Namen kennt und sich die Leute gegenseitig schon ein bisschen einschätzen können. Vielleicht nach dem Kennenlern-Bingo.

\section{Pöstchenvergabe}
\label{poestchenvergabe}
\index{Pöstchenvergabe}
\paragraph{Art:} ruhiges Kennenlernspiel über Eigenschaften der Leute mit "`Kandidatur"' als Aufhänger
\paragraph{Ziel:} Partnerin als "`Kandidatin"' für ein erfundenes Pöstchen vorstellen
\paragraph{Dauer:} 20 Minuten plus (Teilnehmerinnen$\cdot$2) Minuten
\paragraph{Wir brauchen dazu:} Schreibzeug für alle, einen Sitzkreis
\paragraph{So geht es:} Funktioniert wie das Partnerinneninterview. Die Leitfragen sind allerdings:
  \begin{itemize}
    \item In welcher Rolle hast du dich beim letzten Plenum oder deiner letzten Gruppendiskussion gesehen? (Für diesen "`Posten"' wird die Interviewte kandidieren.) Beispiele: Zwischenruferin, Flaschenumstoßerin, Protokollantin, Sprüchermacherin, Inzurückhaltungüberin, Zusammenfasserin \ldots
    \item Warum bist du für diesen "`Posten"' gut geeignet?
    \item Was sind deine weiteren Qualifikationen?
  \end{itemize}

Nach der Vorstellung jeder "`Kandidatin"' darf das Publikum noch Fragen an sie stellen. Gewählt wird allerdings nicht.
\paragraph{Besondere Hinweise:} Dieses Spiel haben Marlies und ich erfunden und auf der 30ten KIF in Dortmund erstmalig ausprobiert.
\paragraph{Wann einsetzen:} Nach der Namensrunde bei einem Seminar, das Kandidaturen, Diskussionen oder Ähnliches behandelt.

\section{Chaosrunde}
\index{Chaosrunde}
\paragraph{Art:} bewegtes Kennenlernspiel zu allen Aspekten
\paragraph{Ziel:} sich nacheinander kurz mit vielen anderen unterhalten
\paragraph{Dauer:} 15--20 Minuten
\paragraph{Wir brauchen dazu:} ---
\paragraph{So geht es:}
Alle gehen kreuz und quer durch den Raum. Wenn die Moderatorin in die Hände klatscht, finden sich die Teilis zu zweit zusammen und fragen sich gegenseitig aus. Wenn die Moderatorin wieder klatscht, gehen alle weiter, bis sie sich beim nächsten Klatschen mit jemand anderem unterhalten.

Nach mehreren Durchgängen setzen sich alle wieder in den Kreis. Dann werden reihum die Teilis vorgestellt, indem alle erzählen, was sie (eventuell) in den Gesprächen von der Teilnehmerin erfahren haben.
\paragraph{Wann einsetzen:} Wenn die Gruppe die Namen halbwegs kennt.

\section{Kennenlern-Obstsalat}
\index{Kennenlern-Obstsalat}
\index{Obstsalat, Kennenlern-}
\paragraph{Art:} sehr actionreiches Kennenlernspiel über die Eigenschaften der Leute
\paragraph{Ziel:} Gemeinsamkeiten finden, Auflockerung, Wachwerden.
\paragraph{Dauer:} 10--30 Minuten (je nach Lust und Laune kann es auch schon mal eine Stunde werden)
\paragraph{Wir brauchen dazu:} ---
\paragraph{So geht es:} Geschlossenen Sitzkreis mit einem Stuhl weniger, als Leute da sind. Rucksäcke, Blöcke, Stifte und andere Gegenstände sollten weit weg in Sicherheit sein. Alle Teilis sollten außerdem ihre Schuhe zugebunden haben.

 Eine steht in der Mitte und sagt "`Ich mag alle, die \ldots"' und danach etwas über sich. Beispiele: "`Ich mag alle, die keine Brille tragen."' (sie trägt also selbst keine Brille), "`\ldots\ die im ersten Quartal des Jahres Geburtstag haben."', "`\ldots\ die Informatik studieren."' (großes Gedränge) oder "`\ldots\ die morgens schlecht aus dem Bett kommen."'

Dann stehen alle auf, auf die dieses Merkmal zutrifft, und suchen sich einen neuen Platz (nicht den Platz der Nachbarin, sonst wird es zu einfach). Wer vorher in der Mitte war, sollte dabei versuchen, einen der freien Plätze zu bekommen. Wer keinen Platz kriegt, steht als Nächste in der Mitte.
\paragraph{Besondere Hinweise:} Wenn sich die Gruppe schon vertrauter ist, können auch Eigenschaften genannt werden wie "`\ldots\ die sich schon einmal in ihren Lehrer verliebt haben."',  "`\ldots\ die in der Schule schon einmal sitzen geblieben sind."' oder "`\ldots\ die erst nach 18 das erste Mal Sex hatten."' (Diese Frage kam mal beim Spielen spät abends auf einem Seminar.)

Achtet aber darauf, dass sich die Gruppe mit dem entsprechenden Level von "`Outing"' auch wohl fühlt. Es müssen ja auch nicht immer alle wahrheitsgemäß aufstehen bzw.~sitzen bleiben.

Schaut vorher auch, ob die Stühle so ein Spiel überhaupt aushalten~-- bei der Ersti-Fahrt 1999 haben wir mit diesem Spiel fünf Stühle geschrottet.

Außerdem sollten vorher alle Teilis ihre Schuhe zugebunden haben (wir hatten auch schon einen Schnürsenkel-Stolper-Unfall).

\paragraph{Wann einsetzen:} Eigentlich immer~-- egal, wie gut sich die Gruppe kennt.  Auch gut zur Auflockerung oder zum Wachwerden geeignet.

\section{Kennenlern-Bingo}
\index{Kennenlern-Bingo}
\index{Bingo|see{Kennenlern-Bingo}}
\paragraph{Art:} angeregtes Kennenlern-Ratespiel über Eigenschaften der Leute
\paragraph{Ziel:} Leute zu finden, die bestimmte Eigenschaften haben; Gemeinsamkeiten finden
\paragraph{Dauer:} 15--20 Minuten
\paragraph{Wir brauchen dazu:} pro Person eine Kopie des "`Kennenlern-Bingo"'-Zettels (Seite \pageref{bingo})
\paragraph{So geht es:} Alle bekommen eine Kopie des "`Kennenlern-Bingo"'-Zettels und gehen damit im Raum herum. Dabei versuchen sie jemanden zu finden, auf die die Beschreibung in einem der Kästchen zutrifft. Diejenige unterschreibt dann im entsprechenden Kästchen. Wer vier Kästchen in einer Reihe ausgefüllt bekommen hat~-- vertikal, horizontal oder diagonal~--, hat ein \emph{Bingo} und ruft lauft "`Bingo"'. Das Spiel geht dann aber trotzdem noch weiter, bis es von der Tutorin nach 10--15 Minuten beendet wird.

Danach in der Runde können alle noch kurz sagen, was sie besonders Interessantes beim Spielen herausgefunden haben (z.\,B.~"`Niemand trägt etwas Handgemachtes."' oder "`\ldots\ spricht russisch."').
\paragraph{Wann einsetzen:} Wenn die Gruppe schon die Namen kennt.

\section{Streichholzvorstellung}
\index{Streichholzvorstellung}
\paragraph{Art:} Selbstvorstellung mit begrenzter Redezeit
\paragraph{Ziel:} jede darf sich der Gruppe vorstellen, solange ein Streichholz brennt
\paragraph{Dauer:} 30 Sekunden pro Spielerin
\paragraph{Wir brauchen dazu:} eine Schachtel normal große Streichhölzer, halbwegs feuerfeste Tische, einen Sitzkreis
\paragraph{So geht es:} Eine Schachtel Streichhölzer geht im Sitzkreis herum. Wer die Schachtel hat, entzündet ein Streichholz und stellt sich der Gruppe vor. Wenn das Streichholz ausgeht (oder die Spielerin es ausschüttelt, um sich nicht die Finger zu verbrennen), gibt die Spielerin die Schachtel (und damit das Wort) weiter.
\paragraph{Besondere Hinweise:} Nicht auf Polstersesseln oder bei Tischdecken benutzen, da Spielerinnen heiße Streichhölzer schon mal fallen lassen. Vorsicht ist auch bei praktisch extrem unbegabten Spielerinnen geboten. Einige Spielerinnen fallen bei der Vorstellung erfahrungsgemäß heraus, da ihnen das Streichholz sehr kurz nach dem Anzünden wieder ausgeht.
\paragraph{Wann einsetzen:} Zur kurzen, überblicksartigen Vorstellung, wenn die Leute sich im Laufe des Seminars noch besser kennen lernen können. Auch bei relativ großen Gruppen und Vielrednerinnen sehr effektiv.

\section{Gegenstand aussuchen}
\index{Gegenstand aussuhen}
\paragraph{Art:} Kennenlernspiel mit Metaphern und persönlichen Assoziationen
\paragraph{Ziel:} einen Gegenstand aussuchen, mit dem man etwas assoziiert
\paragraph{Dauer:} bei $n$ Personen und $m$ Runden etwa $\frac{m\cdot n}{2}$ Minuten \
\paragraph{Wir brauchen dazu:} 10--20 unterschiedliche Gegenstände, zum Beispiel ein Buch, ein Stofftier, einen Jonglierball, einen Filzstift, ein Apfel, einen Schal, eine Kaffeetasse \ldots je verschiedener die Gegenstände, desto besser.
\paragraph{So geht es:} Die Teilis sitzen im Stuhlkreis, in dessen Mitte die Gegenstände liegen. Jetzt sucht sich jede der Reihe nach einen Gegenstand aus, mit dem sie sich indentifiziert oder mit dem sie etwas Persönliches assoziiert. Sie zeigt den anderen den Gegenstand, erzählt die dazu passende Geschichte und legt den Gegenstand dann wieder zurück.
\paragraph{Wann einsetzen:} Für spielerisches Kennenlernen über die bloßen Fakten hinaus. Funktioniert am besten bei Veranstaltungen, die einen gewissen Selbsterfahrungsanteil besitzen, weniger bei einer Office-Schulung.
\paragraph{Varianten:} Lässt sich auch mit dem Partnerinterview kombinieren oder als Feed-back-Technik zur Persönlichkeit benutzen, wenn die anderen aus der Gruppe Gegenstände für diejenige aussuchen, die auf dem "`heißen Stuhl"' sitzt.


\chapter{Spiele zur Gruppeneinteilung}
\index{Gruppeneinteilung}

\section{Fäden ziehen}
\index{Fäden ziehen}
\paragraph{Art:} zufällige Pärchenbildung
\paragraph{Ziel:} alle ziehen an einem Ende eines Fadenbüschels
\paragraph{Dauer:} 2--3 Minuten
\paragraph{Wir brauchen dazu:} pro 2 Teilnehmerinnen ei\-nen etwa 1~m langen Bindfaden (zum Beispiel Paketschnur)
\paragraph{So geht es:} Die Moderatorin hält alle Fäden mit einer Hand in der Mitte hoch, so dass ganz viele Fadenenden herunterhängen. Dann greifen sich alle Teilis je ein Fadenende. Die Teilis, (nach dem Entwirren) die zwei Enden jeweils eines Fadens erwischt haben, gehören danach zusammen.
\paragraph{Besondere Hinweise:} Dieses Spiel ist natürlich nur bei einer geraden Anzahl Teilis sinnvoll.
\paragraph{Wann einsetzen:} Um zufällige Pärchen zu bilden.

\section{Süßigkeiten ziehen}
\index{Süßigkeiten ziehen}
\paragraph{Art:} zufällige Gruppenbildung
\paragraph{Ziel:} alle nehmen sich Süßigkeiten und teilen sich so in Gruppen ein
\paragraph{Dauer:} 2--3 Minuten
\paragraph{Wir brauchen dazu:} verschiedenartige Süßwaren (z.\,B.~Weingummi), pro Teili 1 Stück, die Süßigkeiten für die gleiche Gruppe sehen jeweils gleich aus
\paragraph{So geht es:} Die Moderatorin hat vorher die Süßigkeiten entsprechend sortiert. Dann darf sich jede Teilnehmerin ein Teil nehmen. Je nach Art des Teils bestimmt sich die Gruppenzugehörigkeit. Beispiel: Grünes Weingummi ist Gruppe 1, rotes Weingummi Gruppe 2 usw.
\paragraph{Besondere Hinweise:} Die Teilis sollten sich die Süßigkeiten merken, bevor sie sie essen!

Lakritz ist nicht so gut für dieses Spiel geeignet, weil es viele Leute nicht mögen oder dagegen allergisch sind. Gelatine ist problematisch, wenn Veganerinnen mitspielen.
\paragraph{Wann einsetzen:} Um zufällige Gruppen zu bilden.

\section{Standpunkte}
Auf Seite \pageref{standpunkte} zu finden.

\chapter{Auflockerungsspiele}
\index{Auflockerungsspiele}

\section{Au ja!}
\index{Au ja!}
\paragraph{Art:} lustiges, unterschiedlich aktives Rumalberspiel
\paragraph{Ziel:} jede darf etwas sagen, was dann alle machen
\paragraph{Dauer:} 5--10 Minuten
\paragraph{Wir brauchen dazu:} ---
\paragraph{So geht es:} Alle stehen im Kreis. Wer anfängt, sagt "`Wir machen jetzt alle \ldots"'. Alle rufen "`Au ja! Au ja!"' und machen das Gesagte. So geht es reihum.

Beispiele: "`Wir machen jetzt alle eine Grimasse."', "`Wir klopfen uns alle auf den Bauch (den eigenen)."' oder "`Wir küssen uns alle auf die linke Schulter."'

Das Spiel ist beendet, wenn eine Tutorin sagt: "`Wir arbeiten jetzt alle weiter."'
\paragraph{Varianten:} Margarete (s.~u.)
\paragraph{Wann einsetzen:} Wenn die Gruppe lange sehr ernsthaft gearbeitet hat. Oder wenn sie ganz viel Albernheit herauslassen muss.

\section{Margerite}
\index{Margerite}
\paragraph{Art:} lustiges, unterschiedlich aktives Rumalberspiel
\paragraph{Ziel:} jede darf etwas vormachen, was dann im Kreis herumgeht
\paragraph{Dauer:} 10--15 Minuten
\paragraph{Wir brauchen dazu:} ---
\paragraph{So geht es:} Im Prinzip ahmt die Gruppe eine Margeriten-Blüte nach, deren Blütenblätter sich eins nach dem anderen öffnen oder schließen.

Alle stehen im Kreis. Wer anfängt, macht eine Bewegung und ein Geräusch. Die rechte Nachbarin macht es nach. Danach macht es deren rechte Nachbarin nach und so weiter, bis alle die Bewegung und das Geräusch machen.

Wenn Bewegung und Geräusch wieder bei der angekommen sind, die es vorgemacht hat, macht deren rechte Nachbarin etwas Neues vor.
\paragraph{Wann einsetzen:} Wenn die Gruppe lange sehr ernsthaft gearbeitet hat. Oder wenn sie ganz viel Albernheit herauslassen muss.

\section{Die Milch kocht über!}
\index{Milch kocht über!}
\index{Kochende Milch|\see{Milch kocht über!}}
\index{Brüllspiel|\see{Milch kocht über!}}
\paragraph{Alias:} Kochende Milch
\paragraph{Art:} heftiges Brüllspiel
\paragraph{Ziel:} zwei Gruppen brüllen sich immer lauter an
\paragraph{Dauer:} 5 Minuten
\paragraph{Wir brauchen dazu:} ---
\paragraph{So geht es:} Die Leute bilden zwei Gruppen, die sich gegenüberstehen. Die eine Gruppe sagt: "`Die Milch kocht über!"', worauf die andere Gruppe erwidert: "`Dann hol du sie doch vom Feuer!"', worauf die erste Gruppe wieder sagt: "`Die Milch kocht über!"' und so weiter. Da die jeweils andere Gruppe das eigene Anliegen offensichtlich nicht versteht, werden beide Gruppen nach und nach immer lauter, bis sich beide Gruppen schließlich anbrüllen. Das Spiel ist zu Ende, wenn es die Tutorin abbricht (oder wenn alle Teilnehmerinnen heiser sind).
\paragraph{Varianten:}
	\begin{description}
		\item[Im Restaurant beim Kampf um einen Platz:] "`Sie stehen jetzt sofort auf!"'~-- "`Nein, ich werde jetzt nicht aufstehen!"'
		\item[Kinder am Gartenzaun:] "`Nein!"'~-- "`Doch!"'
	\end{description}
\paragraph{Besondere Hinweise:} Es ist nicht weiter schlimm, wenn euch die Leute aus den Nebenräumen später etwas komisch angucken.
\paragraph{Wann einsetzen:} Wenn die Gruppe gefrustet ist und Dampf ablassen muss. Oder als Erfahrung, wie laut die eigene Stimme sein kann (z.\,B.~bei einem Rhetorikseminar).

\section{Jammern}
\index{Jammern}
\index{Stöhnen|see{Jammern}}
\paragraph{Alias:} Stöhnen
\paragraph{Art:} alle stöhnen oder jammern gleichzeitig
\paragraph{Ziel:} kontrolliert die Genervtheit herauslassen
\paragraph{Dauer:} 2-5 Minuten
\paragraph{Wir brauchen dazu:} ---
\paragraph{So geht es:} Alle stöhnen oder jammern gleichzeitig.
\paragraph{Besondere Hinweise:} Tut enorm gut und ist gar nicht so albern, wie es sich zuerst liest.
\paragraph{Wann einsetzen:} Wenn die Gruppe genervt oder gefrustet ist.

\section{Intelligenztest}
\index{Intelligenztest}
\paragraph{Art:} Gemeines Spiel der hinterhältigen Art.
\paragraph{Ziel:} Ein neues Bewusstsein schaffen für Aufmerksamkeit, vor allem für Klausuren.
\paragraph{Dauer:} 3 Minuten.
\paragraph{Wir brauchen dazu:} pro Person eine Kopie des Intelligenztests (Seite \pageref{iq})
\paragraph{So geht es:} Alle bekommen je eine Kopie des Testes und sollen den Bogen ohne zu sprechen innerhalb von drei Minuten ausfüllen. Wer fertig ist, dreht das Blatt um. Wer den Test schon einmal gemacht hat und im Vorteil wäre, wird gebeten, das Blatt sofort umzudrehen.
\paragraph{Besondere Hinweise:} Dieser "`Test"' dient nur dem Aha-Erlebnis und kann bei einigen Leuten zu enormen Heiterkeitsausbrüchen führen.
\paragraph{Wann einsetzen:} Am Beginn einer Phase, in der es um besondere Aufmerksamkeit geht. Auch gut als Pausenfüller, wenn eine Tutorin noch etwas vorbereiten muss.

\section{Pärchenfangen}
\index{Pärchenfangen}
\index{Anarchofangen|see{Pärchenfangen}}
\index{Ransetzen|see{Pärchenfangen}}
\index{Taxifangen|see{Pärchenfangen}}
\paragraph{Alias:} Anarchofangen, Ransetzen, Taxifangen
\paragraph{Art:} sehr schnelles Fangspiel mit Gehirneinsatz
\paragraph{Ziel:} Gejagte und Jäger, die schnell zu Gejagten werden
\paragraph{Dauer:} 10--15 Minuten
\paragraph{Wir brauchen dazu:} genügend Platz zum Laufen, am besten draußen
\paragraph{So geht es:} Alle teilen sich in Zweiergruppen\footnote{und eine Dreiergruppe, wenn die Anzahl der Spielerinnen ungerade ist} auf, die sich einhaken (nur zu zweit jeweils, nicht alle Pärchen zusammen) und auf der Spielfläche verteilen. Ein Pärchen teilt sich in Jägerin und Gejagte. Wenn die Jägerin die Gejagte erwischt, vertauschen sich die Rollen (Gejagte zu Jägerin und umgekehrt).

Die Gejagte kann sich retten, indem sie sich bei einem Pärchen einhakt. Dabei muss sie sich mit gleicher Blickrichtung wie das Pärchen einhaken (also nicht von vorne einhaken). Sobald die Spielerin, die jetzt in der Mitte ist, merkt, dass die Gejagte fest angedockt hat, kann sie die Spielerin am anderen Ende des Pärchens loslassen. Diese wird dann zur neuen Jägerin, und die alte Jägerin wird zur neuen Gejagten. So schnell kann's kommen!
\paragraph{Besondere Hinweise:} Es ist lustig: Immer gibt es ein, zwei Nasen, die die Regeln partout nicht raffen und als Gejagte auf die Jägerin zulaufen anstatt wegzulaufen.
\paragraph{Varianten:} Die Pärchen stehen nicht, sondern sitzen. Anstatt sich bei einem Pärchen einzuhaken setzt sich die Teilnehmerin auf der Flucht dort heran ("`Ransetzen"').
\paragraph{Wann einsetzen:} Wenn das Hirn qualmt oder auf Stand-by steht und Bewegung an frischer Luft angebracht ist.

\section{Knäuel weitergeben}
\index{Knäuel weitergeben}
\paragraph{Art:} sehr schnelles Actionspiel im Sitzen
\paragraph{Ziel:} zwei Knäuel schnell weitergeben, um nicht beide auf einmal zu bekommen.
\paragraph{Dauer:} 10--20 Minuten
\paragraph{Wir brauchen dazu:} 2 (getrennt) zu Knäuels geknotete Schals, Tücher, Mützen o.~Ä.
\paragraph{So geht es:} Die Gruppe sitzt eng im Kreis (auf Stühlen oder im Schneidersitz), so dass sich die Knie berühren. Die Knäuel starten an zwei verschiedenen Stellen im Kreis. Die Spielerinnen geben die Knäuel an eine Nachbarin weiter oder legen sie ihr in den Schoss. Die Knäuel dürfen dabei durchaus öfters die Richtung wechseln. Wer beide Knäuel auf einmal hat, scheidet aus, und der Kreis wird kleiner.

Das Spiel ist zu Ende, wenn nur noch zwei Spielerinnen übrig sind.
\paragraph{Wann einsetzen:} Bei allgemeiner Lethargie und wenn sich die Gruppe schon ein bisschen kennt.

\section{Zauberwald}
\index{Zauberwald}
\index{Zwerg, Riese, Zauberer|see{Zauberwald}}
\paragraph{Alias:} Zwerg, Riese, Zauberer
\paragraph{Art:} Fangspiel nach ähnlichem Prinzip wie das Knobelspiel \emph{Schnick-Schnack-Schnuck} \emph{(Stein, Schere, Papier)}
\paragraph{Ziel:} schnell entscheiden, schnell reagieren~-- und schnell weglaufen
\paragraph{Dauer:} 10--20 Minuten
\paragraph{Wir brauchen dazu:} genügend Platz zum Laufen, am besten draußen
\paragraph{So geht es:} Die Gruppe teilt sich in zwei ungefähr gleichstarke Fraktionen, die sich im Abstand von gut zwei Metern gegenüber aufstellen. Jeder Gruppe überlegt sich in einer Blitzberatung, welche der drei folgenden Figuren sie einheitlich darstellen möchte:
\begin{description}
  \item[Zwerg:] Mit beiden Armen wird eine stilisierte Zwergenmütze dargestellt. Der Zwerg besiegt den Zauberer, wird aber selbst vom Riesen besiegt.
  \item[Riese:] Beide Arme über dem Kopf hoch strecken, um die enorme Größe des Riesen darzustellen. Der Riese besiegt den Zwergen, kann selbst aber vom Zauberer verzaubert werden.
  \item[Zauberer:] Den rechten Arm als Zauberstab nach vorne strecken ("`\emph{\ldots bsssst!}"'). Der Zauberer verzaubert den Riesen, kann selbst aber vom Zwerg überwältigt werden.
\end{description}
Nach der Beratung zählen alle bis drei. Bei drei stellt jede Gruppe die Figur dar, die sie sich ausgesucht hat, und läuft weg oder versucht entsprechend die anderen zu fangen. Wer gefangen wird, wechselt zu anderen Gruppe über. Wer die "`sichere Zone"' erreicht (3--5~Meter hinter jeder Gruppe), ist in Sicherheit. Wenn beide Gruppen das Gleiche darstellen, beraten sich beide Gruppen neu.

Und weiter geht es mit der nächsten Runde, bis eine Gruppe völlig verschwunden (gefangen) ist oder niemand mehr kann oder will.
\paragraph{Besondere Hinweise:} Mit großen Gruppen macht das Spiel besonders viel Spaß (\emph{the more, the merrier})~-- ich selbst habe es einmal mit insgesamt 50~Leuten gespielt. Vielleicht könnt ihr euch mit ein oder zwei anderen OE-Gruppen dafür zusammentun?
\paragraph{Warnung:} Da auch bei bei diesem Spiel immer Leute dabei sind, die ab und an die Regeln durcheinander bringen, seid bitte vorsichtig. Es hat schon böse Platzwunden gegeben, weil zwei Spielerinnen frontal mit den Köpfen zusammengestoßen sind.
\paragraph{Wann einsetzen:} Bei allgemeiner Lethargie und zum schnellen Wachwerden.

\section{Gordischer Knoten}
\index{Gordischer Knoten}
\index{Knoten, Gordischer}
\paragraph{Art:} Problemlösung. Bewegungsspiel, nicht ganz sanft.
\paragraph{Ziel:} Auflösen des Gordischen Knotens, aber nicht mit Gewalt.
\paragraph{Dauer:} 10 Minuten.
\paragraph{Wir brauchen dazu:} $\geq 8$ Teilis, genug Platz für einen Stehkreis plus Sicherheitszone
\paragraph{So geht es:} Alle stehen in einem engen Kreis und strecken die Arme nach vorne in die Mitte des Kreises. Auf Kommando schließen alle die Augen, gehen auf die Mitte zu und fassen mit jeder Hand (genau) eine andere Hand~-- möglichst nicht die des Nachbarn.

Wenn alle Hände angedockt sind, öffnen alle wieder die Augen und versuchen, den entstandenen Knoten ohne Loslassen zu entwirren.
\paragraph{Besondere Hinweise:} Die Gruppe sollte schon ein Weilchen zusammen gearbeitet haben, damit die gröbsten Berührungsängste abgebaut sind.
\paragraph{Wann einsetzen:} Zum Auflockern. Besonders nützlich, wenn sich die Gruppe während der Arbeit geistig "`verknotet"' hat oder bei einem Problem nicht mehr weiterkommt.

\section{Maschine-Spiel}
\index{Maschine-Spiel}
\paragraph{Art:} kreatives, sehr lustiges Darstellungsspiel
\paragraph{Ziel:} Die Gruppe bildet zusammen eine große Maschine.
\paragraph{Dauer:} 5--10 Minuten
\paragraph{Wir brauchen dazu:} ---
\paragraph{So geht es:} Die Tutorin fängt an und sagt "`Ich bin eine Maschine. Ich mache eine Bewegung und ein Geräusch."' Dazu macht sie wie eine eine eine wiederkehrende Bewegung und ein Geräusch. Nach und nach wird die Maschine durch weitere Mitspielerinnen erweitert, die sich durch Kontakt mit der schon bestehenden Maschine verbinden und selber eine Bewegung ausführen und ein Geräusch machen.

Am Schluss bilden alle gemeinsam eine große, abgefahrene sinnfreie Maschine.
\paragraph{Wann einsetzen:} Bei Lust auf etwas ganz Anderes, und wenn die Gruppe die gröbsten Berührungsängste abgebaut hat.

\section{Wanderndes Klatschen}
\index{Wanderndes Klatschen}
\index{Klatschen, wanderndes}
\paragraph{Art:} ruhiges Aufmerksamkeitsspiel ohne Verliererinnen
\paragraph{Ziel:} gemeinsamen Rhythmus finden, Gruppengefühl stärken
\paragraph{Dauer:} 5--10 Minuten
\paragraph{Wir brauchen dazu:} ---
\paragraph{So geht es:} Die Spielerinnen stellen sich im Kreis auf. Jemand wendet sich einer Nachbarin zu und klatscht in die Hände. Die Nachbarin hat sich inzwischen ebenfalls sich der ersten zugewandt und klatscht zeitgleich in die Hände. Danach gibt sie das Klatschen in die gleiche Richtung weiter~-- oder gibt es wieder zurück. Direkt nach dem Zurückgeben darf nicht noch einmal zurückgegeben werden. Das Spiel läuft gut und macht richtig Spaß, wenn ein gemeinsamer Rhythmus entstanden ist. Nach und nach kann dann das Tempo erhöht werden.
\paragraph{Besondere Hinweise:} Dieses Spiel ist \emph{miteinander}, nicht \emph{gegeneinander}~-- es geht dabei nicht darum, die anderen auszutricksen! Ziel ist der gemeinsame Rhythmus und das Gruppenerlebnis dabei.
\paragraph{Wann einsetzen:} Zur Entlastung der linken Gehirnhälfte und um die Gruppe zusammenzubringen.

\section{Smaug}
\index{Smaug}
\index{Hobbit, Der kleine|see{Smaug}}
\index{Tolkien|see{Smaug}}
\paragraph{Art:} Actionspiel mit Fantasy-Hintergrund
\paragraph{Ziel:}
\begin{quote}
	"`Meine Rüstung ist ein zehnfacher Schild, meine Zähne sind Schwerter, meine Klauen Speere, das Aufschlagen meines Schwanzes ist ein Donnerkeil, meine Schwingen sind Wirbelstürme und mein Atem bringt den Tod!"', prahlt der Drache Smaug vor Bilbo Beutlin.
\end{quote}

In J.~R.~R.~Tolkiens Fantasy-Erzählung \emph{Der kleine Hobbit} zieht Bilbo aus, um Smaugs unermesslichen Gold- und Juwelenschatz zu rauben. In unserer Version der Geschichte ist zwar weder der Einsatz so hoch noch der Drache so schrecklich, aber auch hier geht es darum, den Kontakt mit dem tödlichen Drachen zu meiden und den Schatz zu rauben.
\paragraph{Dauer:} 5--15 Minuten
\paragraph{Wir brauchen dazu:} Taschentuch, Geschirrtuch, zusammengeknoteten Schal o.\,Ä.
\paragraph{So geht es:}
Eine Spielerin verwandelt sich in Smaug, der über seine Juwelen wacht. Ein am Boden ausgebreitetes Taschentuch ist zwar kein so kostbarer Schatz, aber doch sehr viel praktischer. Die anderen Spieler bilden um Smaug einen Kreis und versuchen, den Schatz zu stehlen ohne erwischt zu werden.

Smaug der Mächtige kann sich von seinem Schatz so weit entfernen, wie er es wagt. Wenn er eine Spielerin berührt, erstarrt diese auf der Stelle und bleibt so bis zum Ende des Spieles. Aber keine Sorge: Drachen, die länger als dreißig Sekunden regieren, sind äußerst selten.

Ein beliebter Trick der SchatzräuberInnen ist es, sich von hinten anzuschleichen und die Juwelen zwischen Smaugs Beinen hindurch an sich zu reißen. Auch wenn man nur so tut, als ob man schon erstarrt wäre, kann man manchmal den Drachen überlisten. Und schließlich bleibt euch noch die Möglichkeit eines Massenangriffs, bei dem es zwar die meisten erwischt, aber doch jemand den Schatz zu fassen kriegt. Die erfolgreiche Schatzräuberin ist der nächste Drache. Sollte es Smaug gelingen, alle zu versteinern, bevor es jemand schafft, den Schatz zu ergattern, dann darf er alle für die nächsten fünfhundert Jahre als Salzsäulen stehen lassen.
\paragraph{Wann einsetzen:} Zur Auflockerung

\section{Roboter parken}
\index{Roboter parken}
\index{Roboterspiel}
\paragraph{Art:} Koordinationsspiel mit viel Bewegung
\paragraph{Ziel:} zwei Roboter in Parkposition steuern
\paragraph{Dauer:} 10--15 Minuten
\paragraph{Wir brauchen dazu:} etwas Platz ohne böse Hindernisse, geht aber auch drinnen
\paragraph{So geht es:} Die Teilnehmerinnen finden sich in Dreierteams zusammen. Zwei der Teilnehmerinnen sind die Roboter und stellen sich Rücken an Rücken. Die dritte ist die "`Roboterführerin"'.

Die Roboter bewegen sich im An-Zustand mit kleinen, stetigen Schritten \emph{(stampf stampf)} immer geradeaus. Läuft ein Roboter gegen ein Hindernis oder einen anderen Roboter, dann läuft er auf der Stelle und piept dabei \emph{(miep miep)}.

Mit drei Kommandos lassen sich die Roboter steuern:
\begin{description}
  \item[An-Knopf:] Durch (leichtes!) Tippen auf den Kopf lassen sich die Roboter anschalten, woraufhin sie loslaufen. Es können immer nur beide Roboter gleichzeitig angeschaltet werden.
  \item[Linksdrehung:] Durch Tippen auf die linke Schulter eines Roboters dreht sich dieser Roboter um 90 Grad nach links.
  \item[Rechtsdrehung:] Analog zur Linksdrehung.
\end{description}
Ziel ist, beide Roboter so zu steuern, dass sie schließlich in "`Parkposition"' (Gesicht an Gesicht gegenüber) stehen und sich dann automatisch abschalten. Stehen alle Roboter in Parkposition, ist das Spiel beendet.
\paragraph{Wann einsetzen:} Zur allgemeinen Auflockerung und Erheiterung, zum Aufwärmen nach einer Pause.
\paragraph{Varianten:}
\begin{itemize}
	\item Die Roboter können auch kontinuierlich beschleunigen~-- langsam starten und dann immer schneller werden.
	\item Die Roboterführerin darf die Roboter nur abwechselnd bedienen.
	\item Die Roboter dürfen nicht parallel oder orthogonal zu einer Wand laufen.\footnote{Das macht die Strategie unmöglich, einfach beide Roboter gegen dieselbe Wand zu steuern und dann beide so zu drehen, dass sie an der Wand entlang aufeinander zugehen.}
\end{itemize}

\section{Was machst du denn?}
\index{Was machst du denn?}
\paragraph{Art:} Koordinationsspiel mit nicht zu viel Bewegung
\paragraph{Ziel:} etwas anderes sagen, als man tut
\paragraph{Dauer:} 5--10 Minuten
\paragraph{Wir brauchen dazu:} ---
\paragraph{So geht es:} Alle stellen sich im Kreis auf. Ein Beispielablauf: A macht eine Bewegung, z.\,B.~sie kratzt sich am Kopf. Die Nachbarin B fragt: "`Was machst du denn?"' A sagt etwas anderes: "`Ich hüpfe auf einem Bein."'
Daraufhin hüpft B auf einem Bein und von der nächsten Nachbarin gefragt, was sie denn macht. Sie antwortet z.\,B.: "`Ich spiele Klavier."' Und so weiter.

Das Spiel ist nach ein, zwei Runden beenden, wenn eine Tutorin antwortet: "`Ich beende dieses Spiel."'
\paragraph{Besondere Hinweise:} Es kann sehr unterschiedlich lange dauern, bis alle die Regeln auf die Reihe bekommen haben. Erschwert wird das Spiel, wenn die Tutorinnen die Regeln nicht explizit erklären, sondern einfach anfangen, und die Spielerinnen die Regeln dann selbst herausfinden müssen.
\paragraph{Wann einsetzen:} Zur allgemeinen Auflockerung und Erheiterung, nach einer Pause zum Brain-Booting.

\section{Bewegungskanon}
\index{Bewegungskanon}
\index{Kanon, Bewegungs-}
\paragraph{Art:} Rhythmisches Bewegungsspiel im Sitzen
\paragraph{Ziel:} Action~-- Schwitzen im Sitzen
\paragraph{Dauer:} 10--15 Minuten
\paragraph{Wir brauchen dazu:} Sitzkreis (ohne Stühle), mindestens zwölf Leute
\paragraph{So geht es:} Die OE-Tutorin bittet im ersten Durchgang alle, ihre Bewegungen nachzumachen:
\begin{enumerate}
\item 3x in die Hände klatschen
\item 3x auf die Oberschenkel schlagen
\item 3x die Hände in die Luft strecken
\item 3x mit beiden Füßen auf den Boden stampfen
\end{enumerate}
Wenn der Grundablauf sitzt, wird es zunehmend schwieriger: Die Tutorin teilt die Gruppe in zwei Hälften. Dann beginnt der Kanon:

Die erste Hälfte beginnt mit dem dem Klatschen. Wenn sie sich zum ersten Mal auf die Schenkel schlägt, beginnt die zweite Gruppe mit dem In-die-Hände-Klatschen.

Wenn auch das sitzt, kommt die schwierigste Stufe: Die Tutorin teilt die Teilis in vier Gruppen und dirigiert.
\paragraph{Wann einsetzen:} Zur allgemeinen Auflockerung und Erheiterung

\section{Labyrinth}
\index{Labyrinth}
\index{Matrix|see{Labyrinth}}
\index{Katze und Maus|see{Labyrinth}}
\paragraph{Alias:} Matrix, Katze und Maus
\paragraph{Art:} Fangen spielen mit Köpfchen
\paragraph{Ziel:} Die Maus muss der Katze entkommen und darf dazu das gemeinsame Labyrinth verändern.
\paragraph{Dauer:} 10--15 Minuten
\paragraph{Wir brauchen dazu:} Platz und einen trockenen Untergrund zum Laufen, mindestens 11~Teilis.
\paragraph{So geht es:} Eine Teili ist die Katze, eine andere Teili ist die Maus. Die anderen stellen sich in einer (möglichst quadratischen) Matrix in Armspannen-Abstand auf, also zum Beispiel in 4 Reihen \`{a} 4 Teilis. Dabei schauen alle in eine Richtung und halten die Arme in Schulterhöhe ausgestreckt. Die Maus hat nun drei Optionen:
\begin{itemize}
	\item Vor der Katze weglaufen durch die Reihen und um die Matrix herum. Unter den ausgestreckten Armen dürfen weder Katze noch Maus hindurchgehen.
	\item "`Labyrinth"' oder "`Matrix"' rufen. Daraufhin drehen sich alle stehen Teilis um 90 Grad. Dadurch werden aus den Längsgängen im Labyrinth plötzlich Quergänge.
	\item An eine Reihe andocken. Die Maus wird dann Teil des Labyrinths. Die alte Katze wird dann Maus, und die Teili am anderen Ende der Reihe, an die sich die Maus angedockt hat, wird neue Katze (Klicker-Effekt). Die Reihe rückt dann auf, damit sie wieder im Raster drin ist.
\end{itemize}
Wenn die Katze die Maus fängt, vertauschen sich die Rollen: Aus Katze wird Maus und umgekehrt.
\paragraph{Wann einsetzen:} Zum Wachwerden und weil's Spaß macht.

\section{Schlange und Hase}
\index{Schlange und Hase}
\index{Adler und Hase|see{Schlange und Hase}}
\paragraph{Alias:} Adler und Hase
\paragraph{Art:} Schlangenfangenspiel
\paragraph{Ziel:} Polonaisenkopf fängt Polonaisenschwanz
\paragraph{Dauer:} 5--10 Minuten
\paragraph{Wir brauchen dazu:} Platz zum Laufen
\paragraph{So geht es:} Alle Teilis stellen sich hintereinander auf (wie bei der Schlange im Supermarkt) und halten sich wie bei einer Polonaise an den Schultern. Der Kopf dieser Schlange ist der Adler, der den Hasen (das Ende der Schlange) fangen muss. Dabei darf die Schlange nicht auseinander reißen.

Wenn ihr Spiel mehrfach hintereinander spielen wollt, kann jemand anders der Hase sein. Der bisherige Schlangenkopf kann dann ans Ende der Schlange gehen.
\paragraph{Varianten:} Der Hase kann auch von der Schlange losgelöst sein. Damit es fair ist, darf der Hase aber nur hoppeln und nicht normal laufen. Das Spielfeld sollte dabei eingegrenzt sein (rundherum ist die "`gefährliche Autobahn"'), damit sich der Hase nicht einfach über eine Mauer retten kann.
\paragraph{Wann einsetzen:} Zum Wachwerden und weil's Spaß macht.

\section{Taaa-Tung!}
\index{Taaa-Tung!}
\index{Eddings weitergeben|see{Taaa-Tung!}}
\paragraph{Art:} Spaßspiel mit Konkurrenz
\paragraph{Ziel:} schnell Stifte im Rhythmus weitergeben
\paragraph{Dauer:} 10--15 Minuten
\paragraph{Wir brauchen dazu:} einen Kniekreis, 1 Stift o.~Ä.~pro Teilnehmerin
\paragraph{So geht es:} Alle knien dicht zusammen im Kreis. Jede hat einen Stift vor sich liegen. Die Tutorin gibt den Rhythmus vor: Taaa-tung (Taaa: Edding mit der rechten Hand greifen, tung: Edding der rechten Nachbarin hinlegen), Taaa-tung, Ta-tung-ta-tung (Ta: Edding greifen, tung: Edding nach rechts legen und dabei in der Hand behalten, ta: den selben Edding wieder nach links legen, tung: Edding der rechten Nachbarin hinlegen). Also: Taaa-tung, Taaa-tung, Ta-tung-ta-tung, Taaa-tung und so weiter. Dabei immer schneller werden. Wer gurkt\footnote{Edding nicht richtig weitergegeben, zwei Eddings am Platz, etc.}, fliegt raus. Danach geht es von vorne los, bis nur noch eine übrig ist.
\paragraph{Besondere Hinweise:} Solange es niemand merkt, kann man bei diesem Spiel auch gut schummeln.
\paragraph{Wann einsetzen:} Zur allgemeinen Erheiterung. Besonders gut nach der letzten Arbeitsphase am Tag.

%\section{Lachspiel}
%\index{Lachspiel}
%\paragraph{Art:} das Lachspiel halt
%\paragraph{Ziel:} alle liegen im Kreis und fangen irgendwann an zu lachen
%\paragraph{Dauer:} 5--10 Minuten
%\paragraph{Wir brauchen dazu:} einen Liegekreis auf einem halbwegs sauberen Fußboden, Kopf an Kopf, Köpfe nach innen
%\paragraph{So geht es:} Wer anfängt, sagt ein laut und deutlich vernehmliches "`Ha!"'. Danach sagt ihre Nachbarin "`Ha-Ha!"'. So geht es immer weiter im Kreis. Bei jeder Lacherin wird es ein "`Ha!"' mehr. Irgendwann wird sehr wahrscheinlich die ganze Runde in Lachen ausbrechen (das ist zumindest das Ziel des Spiels).
%\paragraph{Besondere Hinweise:} Nur einsetzen bei Gruppen, die ernsthaft genug sind, um hemmungslos albern sein zu können.
%\paragraph{Varianten:} Nur ein "`Ha!"' pro Teili. Dafür soll das "`Ha"' immer schneller (quasi: mit positiver Beschleunigung) im Kreis herumwandern. Diese Variante ist aber nicht ganz so lachhaft wie die Originalversion.
%\paragraph{Wann einsetzen:} Bei allgemeiner Anspannung oder nach einer langen anstrengenden Arbeitsphase.

\section{Tropengewitter}
\index{Tropengewitter}
\index{Gewitter|\see{Tropengewitter}}
\paragraph{Alias:} Gewitter
\paragraph{Art:} Geräuschspiel
\paragraph{Ziel:} ein selbst gemachtes Tropengewitter
\paragraph{Dauer:} 5--10 Minuten
\paragraph{Wir brauchen dazu:} einen Stehkreis und eine ernsthaft-spielerische Atmosphäre
\paragraph{So geht es:} Die Tutorin stellt sich in die Mitte des Kreises. Wenn sie eine Spielerin anschaut und etwas vormacht, macht diese es nach~-- solange, bis sie etwas anderes machen soll. Die Tutorin lässt alle zusammen nacheinander folgendes machen:
	\begin{description}
		\item[Die Ruhe vor dem Sturm:] konzentrierte Ruhe, Schweigen!
		\item[Das erste Rauschen der Blätter im Wind:] Fingerspitzen in einer Geldzählbewegung aneinander reiben
		\item[Das Rauschen des nahenden Regens:] Hände aneinander reiben
		\item[Die ersten schweren Tropfen:] mit den Fingern einer Hand langsam schnippen (können erfahrungsgemäß nicht alle)
		\item[Die Tropfen fallen dichter:] mit den Fingern beider Hände schnell schnippen
		\item[Der Regen prasselt vom Himmel:] schnell in die Hände klatschen (wie beim Applaus)
		\item[Es donnert:] auf den Boden springen (nur ein, zwei Leute)
	\end{description}
Nachdem der Höhepunkt des Unwetters erreicht ist, baut die Tutorin das Gewitter in umgekehrter Reihenfolge wieder langsam ab, bis am Ende die Ruhe nach dem Sturm folgt: Die Sonne ist wieder hervorgekommen!
\paragraph{Besondere Hinweise:} Das Spiel klappt meiner Erfahrung nach nicht mit pubertierenden Jugendlichen, da diese zwischendrin zu kichern anfangen oder zu unsicher zum ernsthaften Spielen sind.
\paragraph{Wann einsetzen:} Um den Kopf wieder klar zu bekommen (morgens oder nach einer Pause), oder um ein wenig zur Ruhe zu kommen.

\section{Im Kreis hinsetzen}
\index{Im Kreis hinsetzen}
\index{Hinsetzen, im Kreis}
\paragraph{Art:} nettes "`Einfach-so"'-Spiel
\paragraph{Ziel:} Alle sitzen im Kreis auf den Oberschenkeln der Hinterfrau.
\paragraph{Dauer:} 5 Minuten
\paragraph{Wir brauchen dazu:} einen Stehkreis mit mindestens 8 Mitspielerinnen
\paragraph{So geht es:} Alle drehen sich um 90~Grad nach links, so dass alle hintereinander im Kreis stehen. Dann rücken alle möglichst nahe zusammen. Dabei wird der Kreis kleiner. Sobald es nicht mehr näher geht (Haut an Haut), setzen sich alle gleichzeitig auf die Beine der Hinterfrau. Ist echt bequem!
\paragraph{Besondere Hinweise:} Problematisch bei Menschen, die körperliche Nähe überhaupt nicht mögen (Informatiker? Kleiner Scherz am Rande \ldots).
\paragraph{Wann einsetzen:} Zwischendrin für eine schnelle Spielpause (oder ein schnelles Pausenspiel).

\section{Paranoia}
\index{Paranoia}
\paragraph{Art:} Chaosdynamisches Laufspiel mit allen gleichzeitig
\paragraph{Ziel:} jede versucht, eine "`Beschützerin"' zwischen sich und eine "`Verfolgerin"' zu bringen
\paragraph{Dauer:} 5--10 Minuten
\paragraph{Wir brauchen dazu:} Platz zum Laufen
\paragraph{So geht es:} Jede Spielerin sucht sich eine Beschützerin und eine Verfolgerin, teilt dies aber niemandem mit. (Daher weiß niemand, ob und für wen sie Beschützerin oder Verfolgerin ist.) Dann versucht sie, sich so hinzustellen, dass sie die Beschützerin zwischen sich und der Verfolgerin bringt.

Nach der Halbzeit sind die Rollen vertauscht: Jede Spielerin hat jetzt vor der bisherigen Beschützerin Angst und fühlt sich durch die bisherige Verfolgerin beschützt.
\paragraph{Wann einsetzen:} Zum schnellen Auflockern draußen.

\section{Ich fühle mich jetzt so \ldots}
\index{Ich fühle mich jetzt so \ldots}
\paragraph{Art:} Kurzes Darstellungsspiel
\paragraph{Ziel:} durch Bewegung und Geräusch zeigen, wie man sich gerade fühlt
\paragraph{Dauer:} 5--10 Minuten
\paragraph{Wir brauchen dazu:} Stehkreis
\paragraph{So geht es:} Es geht reihum. Wer dran ist, sagt "`Ich fühle mich heute morgen (jetzt) so:"' und stellt dies durch eine Bewegung und ein Geräusch dar.
\paragraph{Wann einsetzen:} Zur Auflockerung und als Stimmungsbild. Sehr schön auch am Morgen vor der ersten Arbeitsphase.

\section{Zulutanz}
\index{Zulutanz}
\index{Tanz, Zulu-}
\paragraph{Art:} alberner (Pseudo-) Stammestanz der Zulu
\paragraph{Ziel:} singen und dabei im Kreis herumhüpfen
\paragraph{Dauer:} 10--15 Minuten
\paragraph{Wir brauchen dazu:} einen Stehkreis und keine neugierigen Nachbarn
\paragraph{So geht es:} Alle stehen im Kreis. Die Tutorin singt vor, alle singen mit:
\begin{quote}
	If you look at me,\\
	Zulu you will see.\\
	If you stand by me,\\
	Zulu you can be.\\
	Hey! Zulu! Attention!\\
	Look at me!
\end{quote}
Dann macht die Tutorin die Aktionen vor. Erst eins, dann nach jedem Durchgang eins mehr, während die anderen Bewegungen beibehalten werden:
\begin{description}
	\item[Right Hand:] die rechte Hand (im Takt) auf den rechten Oberschenkel schlagen
	\item[Left Hand:] auch die linke Hand auf den linke Oberschenkel schlagen
	\item[Right Foot:] mit den rechten Fuß leicht aufstampfen
	\item[Left Foot:] auch mit den linken Fuß leicht aufstampfen (also mit beiden Füßen hüpfen)
	\item[In a Circle:]	als ganze Gruppe im Kreis herumhüpfen (wie um ein Feuer)
	\item[Down:] in die Knie gehen
	\item[Nod:] mit dem Kopf nicken
	\item[Spin:] alle drehen sich zusätzlich um sich selbst
  \item[Backwards:] jetzt hüpft der Kreis rückwärts
  \item[Double time:] doppelt so schnell singen und tanzen
\end{description}
Das Spiel ist zu Ende, wenn niemand mehr kann oder alle durcheinander purzeln.
\paragraph{Besondere Hinweise:} Nicht nach dem Essen spielen. Die Melodie vorher üben.
\paragraph{Wann einsetzen:} Zur heftigen Auflockerung und zum Herumalbern.


\section{Schuhsalat}
\index{Schuhsalat}
\paragraph{Art:} lustiges Durcheinanderwuselspiel
\paragraph{Ziel:} einen Kreis mit zufällig ausgewählten Schuhen an den Füßen bilden
\paragraph{Dauer:} 10--15 Minuten
\paragraph{Wir brauchen dazu:} Platz (die ganze Gruppe muss in einen Kreis passen)
\paragraph{So geht es:} Alle ziehen ihre Schuhe aus und machen damit in der Mitte einen Haufen. Anschließend nimmt sich jede einen rechten und einen linken Schuh (aus unterschiedlichen Paaren). Jede zieht die Schuhe an (so gut es geht \ldots\ nur nicht kaputt machen). Nun müssen sich die Schuhpaare (an den Füßen) wiederfinden und nebeneinander stellen. Es bilden sich dadurch ein oder mehrere Kreise.
\paragraph{Besondere Hinweise:} Funktioniert nur, wenn die Gruppe halbwegs unverkrampft mit Körper- und Fußschweißkontakt umgehen kann.
\paragraph{Wann einsetzen:} Zum Auflockern zwischendurch.

Vielen Dank an Christoph aus Darmstadt für dieses Spiel.


\section{Daumenwrestling}
\index{Daumenwrestling}
\index{Fingerwrestling|see {Daumenwrestling}}
\index{Fingerringer|see {Daumenwrestling}}
\index{Multiplayer-Daumenwrestling|see {Daumenwrestling}}
\paragraph{Alias:} monochroms massives Multiplayer-Daumenwrestling
\paragraph{Art:} handfestes Daumenwrestling
\paragraph{Ziel:} mit dem eigenen Daumen einen anderen Daumen niederringen
\paragraph{Dauer:} 2--30 Minuten (je nach Anzahl der Runden)
\paragraph{Wir brauchen dazu:} 2--12 Spielerinnen (geht auch mit mehr)
\paragraph{So geht es:}
Die rechten Hände von zwei (oder mehr) Spielerinnen an den Fingerspitzen zusammen, dann die Finger zu einer Faust, Daumen hoch. Wer zuerst den Daumen der anderen runterdrückt und fixiert, hat gewonnen.

Bei sehr vielen Spielerinnen kann man die linke Hand dazu benutzen, verschiedene Netzwerkstrukturen zu bilden (sehr lustig für Informatikerinnen!).

\paragraph{Besondere Hinweise:} Aufwärmen per Fingergymnastik nicht vergessen. Achtet auf kurze Fingernägel und spielt fair!
\paragraph{Wann einsetzen:} zum Auflockern zwischendurch
\paragraph{Varianten:} siehe \url{http://www.monochrom.at/daumen/}


\section{Wäscheklammern}
\index{Wäscheklammern}
\index{Geben \& Nehmen|see{Wäscheklammern}}
\index{Klammern|see{Wäscheklammern}}
\paragraph{Alias:} Geben \& Nehmen
\paragraph{Art:} \emph{sehr} intensives Lauf- und Wuselspiel
\paragraph{Ziel:} den anderen Spielerinnen Wäscheklammern von der Kleidung abnehmen oder anstecken
\paragraph{Dauer:} 5--10 Minuten
\paragraph{Wir brauchen dazu:} drei Wäscheklammern pro Teili, viel Platz zum Laufen (am besten draußen)
\paragraph{So geht es:} Jede Spielerin bekommt drei Wäscheklammern und steckt sich diese gut sichtbar an die Kleidung. Die Tutorin erklärt vor jeder Runde die Regeln und eröffnet und beendet die Runden.

\emph{Erste Runde:} Jede versucht, den anderen Spielerinnen ihre Wäscheklammern abzunehmen und sich selbst an die Kleidung zu stecken.

\emph{Zweite Runde:} Jede Spielerin versucht, ihre eigenen Wäscheklammern anderen Spielerinnen an die Kleidung zu stecken. Wer eine Wäscheklammer fallen lässt, steckt diese wieder bei sich selbst an die Kleidung.

\paragraph{Besondere Hinweise:} Räumt mögliche Stolperfallen, Flaschen und Ähnliches vorher aus dem Weg. Wenn möglich, zieht Kleidungsstücke vorher aus, die leicht reißen (beispielsweise dünne Stoffjacken). Achtet darauf, dass niemand scharfkantige Uhren, Ringe oder andere gefährliche Schmuckstücke trägt.
\paragraph{Wann einsetzen:} Zur intensiven Auflockerung zwischendurch. Besser nicht direkt nach dem Mittagessen.


\section{Pause}
\index{Pause}
\index{Auszeit|see{Pause}}
\paragraph{Alias:} Auszeit
\paragraph{Art:} Pause zum Auslüften, Rauchen, Pinkeln oder einfach für andere Gedanken
\paragraph{Ziel:} besseres Weiterarbeiten danach
\paragraph{Dauer:} 5--15 Minuten
\paragraph{Wir brauchen dazu:} ---
\paragraph{So geht es:} Die Tutorin fragt die Gruppe, wie groß der Bedarf nach einer Pause ist. Wenn der Bedarf vorhanden ist, setzt die Tutorin eine Uhrzeit fest, zu der es nach der Pause weitergeht.
\paragraph{Besondere Hinweise:} Sehr wichtiges Spiel! Auch kleine Pausen können die Arbeitsfähigkeit der Gruppe enorm steigern und die gelebte Genervtheit enorm verringern.
\paragraph{Wann einsetzen:} Wenn die Gruppe genervt, erschöpft, hibbelig oder was auch immer ist. Nur am Anfang zum Wachwerden ist die Pause nicht so geeignet.


\chapter{Abends bei einem Bierchen}

\section{Reise nach Kanikano}
\index{Reise nach Kanikano}
\index{Kanikano, Reise nach}
\index{Willi|see{Reise nach Kanikano}}
\index{Newscafé|see{Reise nach Kanikano}}
\index{Piratenschiff|see{Reise nach Kanikano}}
\index{Ratespaß|see{Reise nach Kanikano}}
\paragraph{Alias:} Ratespaß, Willi, Newscaf\'{e}, Piratenschiff
\paragraph{Art:} Ratespiel mit ein, zwei Eingeweihten (Eingeweiden?)
\paragraph{Ziel:} ein Muster bei der Auswahl bestimmter Gegenstände/Wörter erkennen
\paragraph{Dauer:} 15 Minuten bis mehrere Stunden (wenn's schlecht läuft)
\paragraph{Wir brauchen dazu:} einen Haufen Teilis, von denen einige das Spiel noch nicht kennen dürfen
\paragraph{So geht es:} Die Tutorin spielt eine Grenzwächterin oder Türwächterin, die die Spielerinnen nur mit bestimmten Gegenstände und Personen hineinlässt. Die Spielerinnen machen daher Vorschläge, was sie mitnehmen möchten, die die Zollbeamte dann zulässt oder ablehnt.

Wenn eine Spielerin das Muster erkannt hat, sollte sie es nicht laut sagen, sondern zuerst durch weitere Versuche zu bestätigen versuchen. Danach sollte sie den Mund halten, damit sie den anderen den Spaß nicht verdirbt. Das Spiel geht so lange, bis alle das Schema herausbekommen haben.
\paragraph{Varianten:}
	\begin{description}

		\item[Reise nach Kanikano:]
			\rotatebox{180}{
				\begin{minipage}[c]{28em}
					Die Tutorin spielt die Grenzwächterin des Landes Kanikano. Sie lässt nur Gegenstände und Personen ins Land, die in ihrem Namen \emph{kein~I} und \emph{kein~O} haben.
				\end{minipage}
			}

		\item[Willi:]
			\rotatebox{180}{
				\begin{minipage}[c]{36em}
					Willi will nur Gegenstände und Personen, in deren Namen mindestens ein Buchstabe doppelt vorkommt.
				\end{minipage}
			}

		\item[Newscafé:]
			\rotatebox{180}{
				\begin{minipage}[c]{33em}
					Ins Newscafé kommen nur Gegenstände und Personen, die den Namen einer Zeitung oder Zeitschrift tragen. Beispiel: ein Bild, deine Freundin, einen Gong, Zeit \ldots
				\end{minipage}
			}

		\item[Piratenschiff:]
			\rotatebox{180}{
				\begin{minipage}[c]{32em}
					Die Piraten auf dem Schiff (alles harte Männer und Frauen) nehmen nur mit, was mit einem Buchstaben anfängt, der in "`Piratenschiff"' enthalten ist.
				\end{minipage}
			}

		\item[Schere-Spiel/Flaschenspiel:] Siehe dort (Seite~\pageref{flaschenspiel}).
	\end{description}
\paragraph{Wann einsetzen:} Abends in geselliger Runde, wenn die Musik nicht zu laut ist.

\section{Rapunzel}
\index{Rapunzel}
\paragraph{Art:} noch ein Eingeweihten-Ratespiel
\paragraph{Ziel:} herausfinden, wem man als "`Rapunzel"' die Hand geben muss
\paragraph{Dauer:} 10--30 Minuten
\paragraph{Wir brauchen dazu:} einen Haufen Teilis, von denen einige das Spiel noch nicht kennen dürfen
\paragraph{So geht es:} Eine Spielerin ist die "`Rapunzel"' (also diejenige, die raten soll). Jemand, der das Spiel schon kennt, sagt:
  \begin{quote}
    Rapunzel, Rapunzel, hör aufs Wort!\\
    Geh nicht eher, bis ich sage: Geh!
  \end{quote}
Nach einiger Zeit schickt sie Rapunzel mit dem Wort "`Geh!"' aus dem Raum. Nach ein paar Sekunden wird Rapunzel durch die geschlossene Tür wieder hereingerufen. Sie muss dann einer Spielerin die Hand geben.

Das Ganze wiederholt sich mit dieser Rapunzel so lange, bis diese das Muster erkannt hat oder keine Lust mehr hat. Dann ist jemand anderes die Rapunzel.

\subparagraph{Lösung:}
\rotatebox{180}{
  \begin{minipage}[c]{35em}
    Rapunzel muss derjenigen die Hand geben, die nach dem obligatorischen Spruch als Erste redet. Deswegen kann Rapunzel auch erst hinausgeschickt werden, wenn jemand etwas gesagt hat.
  \end{minipage}
}
\vspace{.5em}

\paragraph{Wann einsetzen:} abends in geselliger Runde

\section{Die verrückte Professorin}
\index{Die verrückte Professorin}
\index{verrückte Professorin|see{Die verrückte Professorin}}
\index{Professorin|see{Die verrückte Professorin}}
\paragraph{Art:} total witziges Schauspiel-Ratespiel
\paragraph{Ziel:} eine Spielerin muss nur anhand von Gesten eine "`Erfindung"' erraten 
\paragraph{Dauer:} pro Runde 10--20 Minuten
\paragraph{Wir brauchen dazu:} eine Jacke, zwei Stühle sowie Sitzgelegenheiten für das Publikum
\paragraph{So geht es:} Eine Spielerin wird zur \emph{Professorin} auserkoren, die zwar sehr genial, aber auch etwas verrückt und vor allem extrem vergesslich ist. Sie hat vor kurzem eine geniale Erfindung gemacht, aber leider vergessen, was genau es war. Sie hat nur noch die Bewegung ihrer Hände, um sich wieder daran zu erinnern. 

Die Professorin verlässt zuerst den Raum. In ihrer Abwesenheit einigt sich die Gruppe auf eine "`Erfindung"', die die Professorin "`erfunden"' hat und die sich (hoffentlich) gut in Gesten darstellen lässt: Das Internet, die Kaffeemaschine, das Handy, der Laminator, der motorisierte Milchaufschäumer \ldots

Danach sucht die Gruppe noch eine zwei Teilnehmerinnen aus, die die \emph{Reporterin} sowie \emph{die Hände der Professorin} spielen.

Die "`Professorin"' wird hereingerufen, setzt sich vor der Gruppe auf den Stuhl und lässt die Arme hängen. Die "`Hände der Professorin"' kniet sich hinter die Professorin und steckt ihre Arme unter den Achseln der Professorin durch. Dann zieht sie die Jacke umgekehrt über die Arme, so dass die echten Arme der Professorin nicht zu sehen sind und das zweite Paar Arme stattdessen gestikulieren kann.

Die "`Reporterin"' setzt sich schräg gegenüber der Professorin auf einen Stuhl und fängt mit dem Interview an. Auf die Fragen antworten "`die Hände der Professorin"' mit Gesten, die die Professorin zu verstehen zu versucht und dann ihren Teil dazu sagt. Nach dem Smalltalk am Anfang werden die Fragen der Reporterin immer konkreter, bis die Professorin schließlich erraten hat, um welche Erfindung es geht. 

Beispiel:

\emph{Reporterin:} Guten Tag, Frau Professorin. Wie geht es Ihnen heute?

\emph{Hände:} (Daumen hoch.)

\emph{Professorin:} Danke, danke, sehr gut.

\emph{Reporterin:} Ich habe schon viel von Ihrer neuen Erfindung gehört. Was ist aus Ihrer Sicht das Herausragende an Ihrer Erfindung.

\emph{Hände:} (ziehen ein sehr großes Quadrat auf)

\emph{Professorin:} Dass \ldots\ dass sie sehr groß und eckig ist. Und dass man damit \ldots

\emph{Hände:} (zeigen in die Mitte)

\emph{Professorin:} \ldots\ ähm, und dass man etwas hineintun kann.

\emph{Reporterin:} Ah, sehr interessant. Und was kann man dort hineintun?

\emph{Hände:} (Trinkbewegung)

\emph{Professorin:} Ähm, Tassen. Ja, Tassen!

\paragraph{Wann einsetzen:} Abends in geselliger Runde. Wenn die Professorin und die Hände gut sind, werden sich alle vor Lachen kringeln.


\section{Assoziationsspiel}
\index{Assoziationsspiel}
\index{Heißer Stuhl|see{Assoziationsspiel}}
\paragraph{Alias:} Heißer Stuhl
\paragraph{Art:} lustiges, nicht zu actionreiches Wortspiel
\paragraph{Ziel:} Wortassoziationen finden
\paragraph{Dauer:} 5-15 Minuten (je nach Lust und Laune)
\paragraph{Wir brauchen dazu:} Stühle für alle plus ein Extrastuhl
\paragraph{So geht es:} im Stuhlkreis stehen zwei Stühle etwas abseits als "`heiße Stühle"'. Ein Stuhl davon ist frei. Jetzt sagt die Spielerin auf dem besetzen Stuhl: "`Ich bin \ldots"' und dann einen Begriff. Dann setzt sich jemand auf den freien Stuhl und sagt einen dazu passenden Begriff. Als Nächstes setzt sich jemand auf den ersten Stuhl (die Spielerin steht vom Stuhl auf und setzt sich in die Runde) und sagt wiederum einen dazu passenden Begriff.

Beispiel:

"`Ich bin der Kopfschmerz."'\\
"`Ich bin das Aspirin."'\\
"`Ich bin die Hausapotheke."'\\
"`Ich bin das Hühneraugenpflaster."'\\
"`Ich bin der Fuß."'\\
"`Ich bin die Hand."'\\
\ldots
\paragraph{Wann einsetzen:} abends zum Warmwerden oder zwischendrin zum Auflockern



\section{Kontakt}
\index{Kontakt}
\paragraph{Art:} Wort-Ratespiel, das sehr von Allgemeinbildung profitiert.
\paragraph{Ziel:} Die Gruppe soll Buchstabe für Buchstabe ein Wort raten, dass sich eine Teilnehmerin ausgesucht hat.
\paragraph{Dauer:} 1--15~Minuten pro Runde (sehr unterschiedlich)
\paragraph{Wir brauchen dazu:} 4--15 Spielerinnen
\paragraph{So geht es:} Eine Teilnehmerin ist Wortgeberin und sucht sich ein Wort aus (am besten ein Substantiv oder ein Verb), zum Beispiel ``Korrelation'', und sagt der Gruppe den Anfangsbuchstaben ``K''.

Wenn eine Teilnehmerin eine Idee hat, welches Wort es sein könnte, fragt sie eine Frage, die mit ``Ist es …'' beginnt:
\begin{quote}
``Ist es ein Tier?'' (weil die Teilnehmerin an ``Känguru'' denkt)
\end{quote}

Wenn die Wortgeberin ein anderes Wort als das zu ratende Wort kennt, das auf die Frage passt und zu den bisher bekanntgegebenen Buchstaben passt, antwortet sie mit: ``Nein, es ist kein/keine …'':

\begin{quote}
``Nein, es ist keine Kuh.''
\end{quote}

Dabei ist es unerheblich, ob dies das Wort ist, dass die Ratende im Sinn hatte.

Danach kann die Gruppe die nächste Frage stellen.

Wenn nach einer Frage aus der Gruppe eine andere Spielerin glaubt, dasselbe Wort wie die ratende Spielerin zu kennen, kann sie ``Kontakt'' rufen. Daraufhin zählen beide Spielerinnen herunter: ``Drei … zwei … eins …'' und rufen dann beide das Wort, dass sie vermuten. Dabei gibt es drei Möglichkeiten:

\begin{itemize}
  \item Wenn während des Countdowns die Wortgeberin eine Wortidee hat, sagt sie ``Nein, es ist kein(e) …'', und sonst passiert nichts.
  \item Wenn beide Teilnehmerinnen unterschiedliche Wörter sagen, passiert auch nichts.
  \item Wenn beide Teilnehmerinnen dasselbe Wort sagen, gibt die Wortgeberin der Gruppe den nächsten Buchstaben: ``Ich gebe euch ein \emph{O}!''
  \item Wenn beide Teilnehmerinnen dasselbe Wort sagen und dies das gesuchte Wort ist, hat die Gruppe das Wort erraten, und jemand anderes ist die nächste Wortgeberin. Normalerweise ist dies die Spielerin, die ``Ist es …'' gefragt hat.
\end{itemize}

\paragraph{Wann einsetzen:} abends in geselliger Runde


\chapter{Spiele zur Wissensvermittlung}
\index{Wissensvermittlung}

\section{Gruppenquiz}
\index{Gruppenquiz}
\index{Quiz|see{Gruppenquiz}}
\paragraph{Art:} Quizspiel mit konkurrierenden Gruppen
\paragraph{Ziel:} Die Gruppen überlegen sich Fragen, die die anderen Gruppen beantworten müssen.
\paragraph{Dauer:} 30--45 Minuten
\paragraph{Wir brauchen dazu:} Flipchart oder Packpapier, Moderationskarten, Moderationsstifte, Uhr mit Sekundenanzeige oder Stoppuhr
\paragraph{So geht es:} Die Teilis teilen sich in 3--5 gleich große Gruppen auf, die sich Namen geben (oder welche bekommen), z.\,B.~\emph{Hennen, Eier, Hämmer} und \emph{Krokodile.} Jede Gruppe überlegt sich Fragen zu einem vorgegebenen Thema (eine pro Gruppenmitglied) und schreibt diese auf Moderationskarten. Jede aus der Gruppe muss die Frage beantworten können. Die Fragen sollen interessant, aber nicht zu einfach sein.

In der Zwischenzeit erstellt die Tutorin eine Punktetabelle auf Flipchart oder auf Packpapier: Eine Zeile pro Runde sowie eine Spalte pro Gruppe.

Wenn alle Gruppen ihre Fragen gesammelt und diskutiert haben, geht das eigentliche Spiel los: Aus der ersten Gruppe kommt eine Teili nach vorne und liest ihre Frage vor. Jetzt haben die anderen Gruppen eine Minute Zeit, gruppenintern über eine Antwort zu diskutieren. Die Gruppe, die nach Ablauf der Zeit (oder schon vorher) glaubt die Frage beantworten zu können, ruft lauft "`Hier!"' (oder so).

Die Fragerin zeigt dann willkürlich auf jemanden aus der Hier-Gruppe, die dann die Frage beantworten muss. Befindet die Fragerin die Antwort für richtig, bekommt die antwortende Gruppe einen Punkt (Strich in der Punkteliste). War die Antwort falsch, darf eine andere Gruppe antworten. Pro Runde darf jede Gruppe nur einmal antworten.

Weiß keine Gruppe die richtige Antwort, zeigt die Tutorin auf jemanden aus der fragenden Gruppe, die dann die gewünschte Antwort sagt. Dann stimmen die anderen Gruppen darüber ab, ob sie die Frage (und die Antwort natürlich) interessant fanden. Wenn mindestens ein Viertel der Teilis aus den anderen Gruppen sich für "`interessant"' melden, bekommt die \emph{fragende} Gruppe einen Punkt pro Gruppe, die eine \emph{falsche} Antwort gesagt hat. Sonst bekommt die fragende Gruppe nichts (deswegen \emph{interessante, aber nicht zu einfache} Fragen).
\paragraph{Besondere Hinweise:} Diese Methode habe ich von Daniel Butscher vom v.\,f.\,h.~gelernt.

\section{Scotland Yard}
\index{Scotland Yard}
\index{Stadtralley|see{Scotland Yard}}
\paragraph{Art:} Stadtralley nach Art des Brettspiels \emph{Scotland Yard}
\paragraph{Ziel:} die Stadt und das öffentliche Nahverkehrssystem kennen lernen
\paragraph{Dauer:} bis zu 3~Stunden
\paragraph{Wir brauchen dazu:}
\begin{itemize}
	\item Eine Möglichkeit, eine große Menge SMS zu verschicken (etwa über einen Online-Dienst wie \emph{web.de}). Bei einer Spieldauer von 3~Stunden mit 10~Gruppen macht das 120~SMS, die bezahlt werden müssen.
    \item Kopien von den Regeln auf Seite \pageref{scotland-yard-regeln} für alle Teilnehmerinnen.
	\item Schreibzeug für alle Spielerinnen
	\item mindestens ein Handy pro Gruppe und pro \emph{Mr.}
\end{itemize}

\paragraph{So geht es:} Dies sind die Aufgaben der Spielleitung:
\begin{enumerate}

\item  Vor dem Spielbeginn tragen sich alle Gruppen in eine Liste ein und geben pro Gruppe eine Handynummer an, an die die SMS
mit den Positionen der Spielerinnen versandt werden sollen.

\item Dann werden vor Spielbeginn die Regeln erklärt. Zu diesem Zeitpunkt sind die \emph{Mr.~X/Y/Z} bereits auf
ihren Startpositionen, die sie der Spielleitung bereits mitgeteilt haben. 

\item  Wichtig ist auch der Hinweis, dass die Gruppen ihre Handynummern austauschen können, um gemeinsam auf "`Jagd"' zu gehen.

\item  Das Spiel beginnt damit, dass die Spielleitung den Gruppen die Startpositionen der \emph{Mr.}s mitteilt.

\item  Eine Viertelstunde nach dem Start ruft die Spielleitung bei den \emph{Mr.}s an und lässt sich die aktuellen Positionen
geben. Diese werden dann sofort als SMS an die Suchgruppen gesendet. 
Dies wiederholt die Spielleitung im Viertelstundentakt. Die \emph{Mr.}s geben auch an, wie oft sie bereits gefangen wurden. Diese Info
kann man den Gruppen zusätzlich mitgeben (Leistungsdruck). Auch geben die \emph{Mr.}s an, wenn eine Gruppe alle drei
gefangen und damit gewonnen hat. Je nach Zeitpunkt kann man das Spiel dann mit einer SMS beenden oder die nächsten
Plätze ausspielen lassen.

\item  Nach etwa eineinhalb Stunden kann man die Suchgruppen anrufen und hören, ob sie immer noch dabei sind. Bei
diesem Gespräch kann man sie noch ein wenig aufmuntern oder vielleicht mal eine Hilfe geben, wenn sie noch gar keinen
Erfolg hatten. Dies kann man nun auch jede Stunde machen, damit die \emph{Mr.}s nicht umsonst durch die Stadt laufen.

\item  Wenn das das Zeitlimit erreicht ist, wird das Spiel per SMS beendet. Je nach Erfolgsquote der Gruppen kann man schreiben,
dass alle zurück zum Start kommen sollen und die \emph{Mr.}s noch bis zur Tür der PH gejagt werden dürfen, da diese auch
dorthin zurück kommen sollen. Auch den \emph{Mr.}s muss das Ende des Spiels mitgeteilt werden.
\end{enumerate}

\paragraph{Besondere Hinweise:} Kümmert euch \emph{rechtzeitig} vorher darum, dass genügend Geld für die SMS auf dem Konto ist!
\paragraph{Wann einsetzen:} Um den Erstis das Nahverkehrssystem nahe zu bringen. Außerdem lernen sich die Leute bei diesem Spiel gegenseitig kennen.


\chapter{Spiele zur Gruppenarbeit, Kommunikation und Gruppendynamik}
\index{Gruppenarbeit}
\index{Kommunikation}
\index{Gruppendynamik}

\section{NASA-Spiel}
\index{NASA-Spiel}
\label{nasa}
\paragraph{Art:} Teamspiel
\paragraph{Ziel:} Erkennen der unterschiedlichen Arbeitsweisen und Entscheidungsfindungsprozesse bei Einzelarbeit, Gruppenarbeit und Delegation
\paragraph{Dauer:} 1,5--2 Stunden
\paragraph{Wir brauchen dazu:} pro Person eine Spielanleitung (Seite \pageref{nasa-kopien}) sowie Schreibkram
\paragraph{So geht es:} Jede Spielerin erhält das Blatt aus dem Anhang. Für die zweite Phase bilden sich mindestens zwei Gruppen von etwa acht Spielerinnen.
\paragraph{Verlauf:} Zum Verlauf des Spiels wird den Teilnehmerinnen Folgendes erklärt:

\emph{In diese Übung spielen wir unsere Möglichkeiten, Entscheidungen zu treffen, an einem Modell durch. Wir erfahren dabei, wie sich Entscheidungen sinnvoll durchführen lassen und was für Hindernisse im Wege stehen können.}

Bitte achtet bei den ersten drei Spielphasen darauf, dass die Spielerinnen die vorgegebene Zeit nicht überschreiten. Zeitdruck ist ein sehr wichtiges Element bei diesem Spiel!
\subparagraph{1.~Einzelentscheidung (5 Minuten):} Ihr versucht~-- jede für sich allein~-- die gestellte Aufgabe zu lösen.
\subparagraph{2.~Gruppenentscheidung (15 Minuten):} Das Ziel ist ein Beschluss der Gruppe, mit dem jede von euch einverstanden ist. Das bedeutet, dass der Rang jedes der 15 Gegenstände, die für das Überleben notwendig sind, die Zustimmung einer jeden von euch haben muss, um ein Teil des Gruppenbeschlusses zu werden.

Es wird sich nicht in allen Punkten erreichen lassen, dass alle Gruppenmitglieder zu der gleichen Meinung kommen. Ihr versucht aber als Gruppe, jeden Punkt so zu diskutieren und zu beschließen, dass alle Mitglieder zumindest teilweise zustimmen können.
\subparagraph{3.~Delegiertenentscheidung (10 Minuten):} Jede Gruppe wählt aus ihrer Mitte zwei Vertreterinnen, die nach Meinung der Gruppe am besten mit der Materie umgehen können. Die Vertreterinnen aller Gruppen setzen sich zusammen und entscheiden im Plenum noch einmal. Die anderen Teilnehmerinnen dürfen dabei zuhören  und -schauen, aber während der Verhandlung nichts sagen. \emph{Diesen Durchgang könnt ihr bei Zeitknappheit wegfallen lassen.}
\subparagraph{4.~Auswertung (15--30 Minuten):} Die verschiedenen Ergebnisse werden untereinander und mit dem Sachverständigenergebnis der NASA-Fachleute verglichen:
	\begin{enumerate}
	\item Sauerstofftanks
	\item Wasser
	\item Sternenkarte
	\item Nahrungskonzentrat
	\item Fernmelde-Empfänger
	\item Sender
	\item Nylonseil
	\item Erste-Hilfe-Koffer
	\item Fallschirmseide
	\item Schlauchboot
	\item Signalpatronen
	\item Pistole
	\item Trockenmilch
	\item Heizgerät
	\item Magnetkompass
	\item Streichhölzer
	\end{enumerate}
Dies ist nicht die \emph{einzig wahre} Lösung. Ziel des Spieles ist nicht ein möglichst gutes Ergebnis, sondern die verschiedenen Wege dort hin und die Erfahrungen und Erkenntnisse dabei.

\emph{Bitte macht unbedingt die Auswertung!} Ohne Auswertung ist das NASA-Spiel wertlos!

Bei der Auswertung soll in etwa das herauskommen, was bei der OE-Vor\-be\-rei\-tungs\-fahrt 2000 herausgekommen ist:
\begin{itemize}
  \item neue/andere/bessere Ideen durch Gruppenarbeit statt Einzelarbeit
  \item vorher sollte man die Fakten klären und einen gleichen Wissensstand herstellen
  \item vorher sollte man auch das Ziel und die Prioritäten bei der Aufgabe klären
  \item es ist wichtiger, dass die Gruppe ein Ergebnis bekommt, als dass Einzelne ihr Ergebnis versuchen durchzubringen
  \item vorher sollte man sich über die Vorgehensweise einigen
  \item auch knappe Zeit sollte in Ruhe genutzt werden; Hektik ist kontraproduktiv
  \item man sollte sich bewusst machen, was man \emph{nicht} weiß und keine Zeit mit sinnlosem Herumraten verschwenden
  \item mögliche Entscheidungsfindungsmodelle:
    \begin{description}
      \item [Konsens:] alle können \emph{gut} mit dem Ergebnis leben; möglichst \emph{Win-Win-Lösung,} die nicht unbedingt in der Mitte zwischen den Ausgangspositionen liegen muss; erfordert das Klären von Interessen
      \item [Kompromiss:] man trifft sich in der Mitte; es reicht, die Standpunkte zu klären\footnote{ein Kompromiss kann über Konsens erreicht werden, muss aber nicht}
      \item [Mehrheitsentscheid:] schnelle Entscheidungen, auch wenn nicht jeder damit einverstanden ist; das Gegenteil von Konsens; Nachteile:
      \begin{itemize}
        \item nicht alle stehen unmittelbar dahinter und sabotieren eventuell das Ergebnis
        \item nicht alle verstehen unbedingt die Lösung
        \item kompetente Lösungen werden eventuell überstimmt
      \end{itemize}
    \end{description}
\end{itemize}
\paragraph{Besondere Hinweise:} Das NASA-Spiel wurde erstmals veröffentlicht bei \cite{nasa}.

\section{Schiffbruch}
\label{schiffbruch}
\index{Schiffbruch}
\paragraph{Art:} Teamspiel
\paragraph{Ziel:} Erkennen der unterschiedlichen Arbeitsweisen und Entscheidungsfindungsprozesse bei Einzelarbeit, Gruppenarbeit und Delegation
\paragraph{Dauer:} 1,5--2 Stunden
\paragraph{Wir brauchen dazu:} pro Person eine Spielanleitung (Seite \pageref{schiffbruch-kopien}) sowie Schreibkram
\paragraph{So geht es:} Jede Spielerin erhält das Blatt aus dem Anhang. Für die zweite Phase bilden sich mindestens zwei Gruppen von etwa acht Spielerinnen.
\paragraph{Verlauf:} Zum Verlauf des Spiels wird den Teilnehmerinnen Folgendes erklärt:

\emph{In diese Übung spielen wir unsere Möglichkeiten, Entscheidungen zu treffen, an einem Modell durch. Wir erfahren dabei, wie sich Entscheidungen sinnvoll durchführen lassen und was für Hindernisse im Wege stehen können.}

Bitte achtet bei den ersten drei Spielphasen darauf, dass die Spielerinnen die vorgegebene Zeit nicht überschreiten. Zeitdruck ist ein sehr wichtiges Element bei diesem Spiel!
\subparagraph{1.~Einzelentscheidung (5 Minuten):} Ihr versucht~-- jede für sich allein~-- die gestellte Aufgabe zu lösen.
\subparagraph{2.~Gruppenentscheidung (15 Minuten):} Das Ziel ist ein Beschluss der Gruppe, mit dem jede von euch einverstanden ist. Das bedeutet, dass der Rang jedes der 15 Gegenstände, die für das Überleben notwendig sind, die Zustimmung einer jeden von euch haben muss, um ein Teil des Gruppenbeschlusses zu werden.

Es wird sich nicht in allen Punkten erreichen lassen, dass alle Gruppenmitglieder zu der gleichen Meinung kommen. Ihr versucht aber als Gruppe, jeden Punkt so zu diskutieren und zu beschließen, dass alle Mitglieder zumindest teilweise zustimmen können.
\subparagraph{3.~Delegiertenentscheidung (10 Minuten):} Jede Gruppe wählt aus ihrer Mitte zwei Vertreterinnen, die nach Meinung der Gruppe am besten mit der Materie umgehen können. Die Vertreterinnen aller Gruppen setzen sich zusammen und entscheiden im Plenum noch einmal. Die anderen Teilnehmerinnen dürfen dabei zuhören und -schauen, aber während der Verhandlung nichts sagen. \emph{Diesen Durchgang könnt ihr bei Zeitknappheit wegfallen lassen.}
\subparagraph{4.~Auswertung (15--30 Minuten):} Die verschiedenen Ergebnisse werden untereinander und mit dem Sachverständigenergebnis (von US-Marineoffizierinnen) verglichen.

Laut den Expertinnen sind bei einem Schiffbruch diejenigen Artikel am wichtigsten, die einem helfen, sich bei potenziellen Retterinnen bemerkbar zu machen sowie um kurzfristig zu überleben.

Navigationsartikel sind nicht wichtig, weil man zu weit vom Land entfernt ist, um aus eigener Kraft dorthin zu kommen. Weder Nahrung noch Wasser würden außerdem lange genug dafür reichen. Der Mensch kann~-- ohne bleibenden Schaden zu nehmen~-- 36~Stunden ohne Wasser auskommen und 30~Tage ohne Nahrung.

Auf der südlichen Halbkugeln sind die Jahreszeiten den unsrigen entgegengesetzt: Dort ist also Sommer, wenn bei uns Winter ist (und umgekehrt). Die Meeresströmungen bewegen sich dort gegen den Uhrzeigersinn (auf der nördlichen Halbkugel im Uhrzeigersinn). Das Rettungsboot treibt also in Richtung Antarktis.

	\begin{enumerate}
		\item Rasierspiegel. Damit kann man die Sonne reflektieren und Signale senden.
		\item Dieselöl. Kann aufs Meer ausgegossen und entzündet werden (mit einem Geldschein oder einen Stück Kleidung und Streichhölzern).
		\item Wasser. Um nicht zu verdursten.
		\item Nahrungsration. Diese besteht nur aus Grundnahrungsmitteln und kann notfalls über mehrere Tage gestreckt werden.
		\item Plastikfolie. Damit kann man Regenwasser und Tau sammeln sowie sich gegen Unwetter schützen.
		\item Schokolade. Als Reservenahrung.
		\item Angel und Zubehör. Da es nicht sicher ist, ob man hier damit Fische fangen kann, ist die Schokolade wichtiger.
		\item Nylonschnur. Um bei einem Sturm wichtige Dinge festzubinden.
		\item Aufblasbares Kopfkissen. Als Schwimmhilfe/Rettungsring, wenn jemand ins Wasser gefallen ist.
		\item Haifisch-Abwehr-Flüssigkeit. Bringt nur etwas, wenn man ins Wasser geht.
		\item Cognac. Zum Desinfizieren von Wunden. Als Getränk ist Cognac in dieser Situation nicht geeignet, weil er die Poren öffnet (Wasserverlust) und durstig macht.
		\item FM-Transistorradio. Bringt nichts, da wegen der Erdkrümmung die Empfangsreichweite maximal 30~Kilometer beträgt und das Festland zu weit entfernt ist.
		\item Karte von indischen Ozean. Bringt nichts, da die Schiffbrüchigen weder ihre eigene Position genau bestimmen noch sich aus eigener Kraft landwärts fortbewegen können.
		\item Moskitonetz. So weit von Land entfernt gibt es keine Mücken. Zum Fischen ist das Netz auch nicht geeignet.
		\item Sextant. Ist ohne Chronometer und Tabellen relativ wertlos, weil man ihn dann für die Positionsbestimmung nicht einsetzen kann.
	\end{enumerate}
Dies ist nicht die \emph{einzig wahre} Lösung. Ziel des Spieles ist nicht ein möglichst gutes Ergebnis, sondern die verschiedenen Wege dort hin und die Erfahrungen und Erkenntnisse dabei.

\paragraph{Besondere Hinweise:} \emph{Bitte macht unbedingt die Auswertung!} Ohne Auswertung ist dieses Spiel wertlos! Die Ergebnisse sollten ähnlich wie beim NASA-Spiel (Seite~\pageref{nasa}) sein.

An dieses Spiel lässt sich direkt \emph{Insel ohne Wiederkehr} anschließen.

\section{Insel ohne Wiederkehr}
\index{Insel ohne Wiederkehr}
\index{Entscheidung unter Unsicherheit|see{Insel ohne Wiederkehr}}
\index{Inselspiel|see{Insel ohne Wiederkehr}}
\paragraph{Alias:} Inselspiel
\paragraph{Art:} Teamspiel
\paragraph{Ziel:} die gruppendynamischen Prozesse bei Entscheidung unter Unsicherheit verdeutlichen
\paragraph{Dauer:} 45--60 Minuten inklusive Auswertung
\paragraph{Wir brauchen dazu:} pro Person eine Kopie der Spielanleitung (Seite~\pageref{wiederkehr-kopien}) sowie der Karte (Seite~\pageref{wiederkehr-karte})
\paragraph{So geht es:} Jede Spielerin bekommt beide Kopien ausgehändigt. Die Spielleiterin erklärt kurz die Situation (siehe Seite~\pageref{wiederkehr-kopien}). Jetzt soll die gesamte Gruppe innerhalb von 30~Minuten eine Entscheidung fällen (keine einzelnen Liste, keine Delegiertenentscheidung).

Die Auswertung erfolgt, sobald die Gruppe eine Lösung gefunden hat oder die Zeit um ist.

Bei der Auswertung können unter anderem möglicherweise herauskommen:
\begin{itemize}
  \item Es gibt Spielerinnen, die lieber sicher überleben und dafür auf die Chance verzichten, in die Zivilisation zurückzukehren. Andere riskieren lieber ihr Leben, anstatt für immer auf ein Leben in der Zivilisation zu verzichten.
  \item Entscheidungsprozesse unter Unsicherheit (Werden wir auf der Luft- und Schifffahrtslinie gesehen werden?) sind extrem schwierig, weil man sich nicht auf sichere Fakten einigen kann.
  \item Es gibt Ausgrenzungs- und Abspaltungsprozesse innerhalb der Gruppe.
  \item Vielen Menschen ist das eigene Überleben sehr wichtig~-- wichtiger als das Überleben der anderen. Dafür setzen sie auch unfaire Diskussionsmittel ein.
\end{itemize}

Über Auswertungsergebnisse von stattgefundenen Spielen würde ich mich freuen, da ich die Ergebnisse meiner eigenen Runde leider nicht mehr habe. \url{oliver@spielereader.org}

\paragraph{Besondere Hinweise:} Auch bei diesem Spiel ist die Auswertung extrem wichtig!

Dieses Spiel habe ich von Daniel Butscher von vfh.
\paragraph{Wann einsetzen:} Im Anschluss an \emph{Schiffbruch} (Seite~\pageref{schiffbruch})~-- wenn die Themen \emph{Gruppendynamik} und \emph{Entscheidung unter Unsicherheit} behandelt werden sollen.


\section{Zwei Euro}
\index{Zwei Euro}
\index{Fünf Mark|see{Zwei Euro}}
\paragraph{Alias:} Fünf Mark
\paragraph{Art:} Teamspiel
\paragraph{Ziel:} eine auf den ersten Blick unmögliche Aufgabe im Team lösen (wie Mathe-Aufgaben halt)
\paragraph{Dauer:} 10--20 Minuten
\paragraph{Wir brauchen dazu:} pro Team ein 2-Euro-Stück
\paragraph{So geht es:} Die Gruppe wird in Teams von drei bis fünf Mitgliedern aufgeteilt. Jedes Team soll möglichst schnell und möglichst genau das Gewicht eines 2-Euro-Stücks bestimmen. An "`Werkzeug"' steht dafür alles zur Verfügung, was sich im Seminarraum befindet (Also besser diesen Spiele-Reader verstecken!). Der Raum darf während der Übung nicht verlassen werden.

Das Team, das das Gewicht zuerst (und möglichst genau) angeben kann, hat das Spiel gewonnen.

\subparagraph{Lösung:}
\rotatebox{180}{
	\begin{minipage}[c]{35em}
		Ein 2-Euro-Stück wiegt 8,5~Gramm.
	\end{minipage}
}

\section{Brücke bauen}
\index{Brücke bauen}
\paragraph{Art:} Teamspiel
\paragraph{Ziel:} im Team eine schwierige Aufgabe lösen
\paragraph{Dauer:} 5--15 Minuten
\paragraph{Wir brauchen dazu:} insgesamt 1 Schere und 1 Rolle Krepp-Klebeband, pro Team ein paar Blatt Papier und 2 Tische
\paragraph{So geht es:} Die Gruppe wird in Teams von drei bis fünf Mitgliedern aufgeteilt. Jedes Team soll möglichst schnell aus Papier und Klebeband eine tragfähige Brücke zwischen zwei Tischen bauen (je weiter auseinander, desto schwieriger~-- 50~cm sind ok). Die Brücke soll die Schere als Belastung aushalten können. Die Teams dürfen dabei die Schere zum Testen benutzen.

\section{Schere-Spiel}
\index{Schere-Spiel}
\index{Flaschenspiel|see{Schere-Spiel}}
\label{flaschenspiel}
\paragraph{Alias:} Flaschenspiel
\paragraph{Art:} Kommunikationsspiel mit "`Eingeweihten"'
\paragraph{Ziel:} erkennen, das es nicht auf das Vordergründige ankommt
\paragraph{Dauer:} 10--30 Minuten
\paragraph{Wir brauchen dazu:} Sitzkreis, 1 Schere (für das Schere-Spiel) bzw.~1 Flasche mit Deckel, ein paar Teilis, die das Spiel noch nicht kennen
\paragraph{So geht es:} Beim Schere-Spiel wird eine Schere im Kreis herumgegeben. Dabei ist es egal, \emph{wie} die Schere weitergegeben wird (also offen, geschlossen, mit dem Griff nach vorne\footnote{Das ist höchstens in Bezug auf die Verletzungsgefahr relevant.} oder nach hinten), aber die Spielerin sagt beim Weitergeben "`offen"' oder "`geschlossen"'.

Wenn eine Spielerin das Muster erkannt hat, sollte sie es nicht laut sagen, sondern zuerst durch weitere Versuche zu bestätigen versuchen. Danach sollte sie den Mund halten, damit sie den anderen den Spaß nicht verdirbt. Das Spiel geht so lange, bis alle das Schema herausbekommen haben.

\subparagraph{Lösung:}
\rotatebox{180}{
	\begin{minipage}[c]{35em}
		"`Offen"' sagt die Tutorin, wenn die entgegennehmende Spielerin die Beine offen (also nicht gekreuzt) hält, und "`geschlossen"' entsprechend umgekehrt.
	\end{minipage}
}
\vspace{.5em}

\paragraph{Wann einsetzen:} Einfach so zum Spaß in geselliger Runde. Oder um zum Thema \emph{Kommunikation} deutlich zu machen, dass es bei einer Mitteilung nicht nur auf das vordergründig Offensichtliche ankommt.
\paragraph{Varianten:}
\begin{itemize}
	\item Beim \emph{Flaschenspiel} wird eine Flasche mit Schraubverschluss im Kreis herumgegeben.
	\item
		\rotatebox{180}{
			\begin{minipage}[c]{37em}
				\emph{Offen/geschlossen} kann sich auch auf den Mund der weitergebenden Spielerin beziehen.
			\end{minipage}
		}
\end{itemize}

\section{Mörder, äh, Mörderin}
\index{Mörder}
\index{Mörderin|see{Mörder}}
\paragraph{Art:} sehr paranoides Spiel für nebenher
\paragraph{Ziel:} alle bringen sich nach und nach gegenseitig (symbolisch) um (und lernen dabei die Namen)
\paragraph{Dauer:} nebenher während der gesamten Veranstaltung, einen bis mehrere Tage lang
\paragraph{Wir brauchen dazu:} Loszettel mit den Namen aller Spielerinnen
\paragraph{So geht es:} Am Anfang der Veranstaltung werden die Zettel gemischt und an die Spielerinnen verteilt. Dabei erhält jede den Zettel mit dem Namen ihres zukünftigen Opfers:
	\begin{itemize}
		\item zufällig durch Ziehen, wenn niemand Externes da ist, die die Zettel verteilen kann, oder
		\item jemand, die nicht mitspielt, verteilt die Zettel so, dass sich aus allen Mörderinnen-Opfer-Beziehungen ein Kreis bildet (damit genau dann das eigene Opfer die eigene Mörderin ist, wenn nur noch zwei Spielerinnen überlebt haben)
	\end{itemize}
Dann wird ein Zettel mit den Namen aller Spielerinnen aufgehängt. Auf Zeichen beginnt das Spiel.

Ein Mord geht folgendermaßen vor sich:
	\begin{enumerate}
		\item Die Mörderin trifft ihr Opfer, das auf ihrem Zettel steht.
		\item Es dürfen keine (lebenden) Zeuginnen anwesend sein. Wenn beim Mord eine Zeugin anwesend ist, den Mord sieht und dies sofort sagt, dann ist der Mord ungültig.
		\item Die Mörderin gibt dem Opfer einen beliebigen Gegenstand, zum Beispiel eine Flasche oder einen Stift. Wenn das Opfer den Gegenstand freiwillig annimmt, ist es gemordet.
		\item Die Mörderin teilt dem Opfer mit, dass es nun tot ist.
		\item Das Opfer gibt den Zettel mit dem Namen seines Opfers der Mörderin. So erhält die Mörderin den Namen ihres nächsten Opfers.\footnote{Auf diese Weise entstehen Massenmörderinnen.}
		\item Das Opfer trägt auf der aushängenden Namensliste ein Kreuz neben dem Namen der Verblichenen ein sowie die Todeszeit und -art, wenn sie mag ("`Mit einem Grillspieß ermordet."', "`War beim Kiffen zu gierig."' oder so).
		\item Man darf nicht dazu lügen, ob man noch lebt oder schon tot ist (Auslassungslügen sind natürlich erlaubt). Tote dürfen noch Lebende außerdem nicht entlarven.
	\end{enumerate}
Wer überleben möchte, sollte deswegen keine Gegenstände direkt annehmen oder sicherstellen, dass eine lebende Zeugin zuschaut.

Das Spiel ist zu Ende, wenn beim Showdown der letzten beiden überlebenden Spielerinnen die eine die andere erfolgreich ermordet.
\paragraph{Besondere Hinweise:} Das Spiel läuft während des Seminars oder der Fahrt so nebenher. Es kann allerdings zu einer handfesten Paranoia führen und im Extremfall eine konstruktive Zusammenarbeit während des Seminars unmöglich machen. Es kann aber auch einen Heidenspaß machen und sehr spannend sein!
\paragraph{Varianten:}\begin{itemize}
		\item Der Mord geschieht durch ein einfaches "`Du bist tot!"'. Das hat aber den Nachteil, dass sich das Opfer nicht wehren kann.
		\item Der Mord geschieht durch einen symbolischen Mordakt, zum Beispiel:
			\begin{itemize}
				\item mit einer Wasserpistole erschießen
				\item mit Tabasco im Bier vergiften
				\item durch eine Zeitbombe zerreißen (mit einer Fotoapparat-Blitz\-licht\-birne als Sprengladung)
			\end{itemize}
		Damit niemand verletzt wird, darf niemand beim Autofahren ermordet werden (für den Mord kurz rechts heranfahren ist selbstverständlich in Ordnung). Eine echte Körperverletzung muss auch ausgeschlossen sein.
		
		Dabei kann es durchaus zu sehr schönen Mafiamorden kommen, wie etwa jemandem nachts an der Haustür mit der Wassermaschinenpistole aufzulauern.
	\end{itemize}
\paragraph{Wann einsetzen:} Bei Seminaren und längeren Veranstaltungen, wenn die Anwesenden Spaß an so umfangreichen Spielen haben.


\section{Nacht in Palermo}
\index{Nacht in Palermo}
\index{Palermo, Nacht in}
\index{Mafia|see{Nacht in Palermo}}
\index{Mörder|see{Nacht in Palermo}}
\index{Werwolf|see{Nacht in Palermo}}
\index{Werwölfe von Thiercelieux|see{Nacht in Palermo}}
\paragraph{Alias:} Mafia, Mörder, Werwolf, Werwölfe von Thiercelieux
\paragraph{Art:} Mörderinnenspiel mit extremem Psychofaktor
\paragraph{Ziel:} die Gruppe muss zwei Mörderinnen finden und hinrichten lassen
\paragraph{Dauer:} 15--45 Minuten
\paragraph{Wir brauchen dazu:} Sitzkreis mit mindestens 10 Teilis, Zettel als Lose, Stift, möglichst wenig Licht (optional)
\paragraph{So geht es:} Eine Spielerin ist die Spielleiterin. Die anderen ziehen Lose (oder Spielkarten), um ihre Rollen zugeteilt zu bekommen:
\begin{description}
	\item [2~Mörderinnen:] Zettel mit \emph{M}
	\item [1~Detektivin:] Zettel mit \emph{D}
	\item [Bürgerinnen:] leere Zettel für den Rest
\end{description}

Alle sitzen so im Kreis, dass jede jede sehen kann. Die Spielleiterin fängt an zu erzählen:
\begin{quotation}
	Es wird Nacht in Palermo. Die Bürgerinnen gehen schlafen. \emph{Alle Spielerinnen schließen die Augen.}
	
	Doch zwei Mörderinnen erwachen und suchen ein Opfer. \emph{Die beiden Mörderinnen erwachen und verständigen sich durch Blicke oder andere Zeichen auf ein Opfer und teilen der Spielleiterin das Opfer mit.} Die Mörderinnen schlafen wieder ein. \emph{Die Mörderinnen schließen wieder die Augen.}
	
	Die Detektivin erwacht. \emph{Die Detektivin öffnet die Augen. Sie deutet für die Spielleiterin auf eine Bürgerin. Die Spielleiterin signalisiert, ob dies eine Mörderin oder eine Bürgerin ist.} Die Detektivin schläft wieder ein. \emph{Die Detektivin schließt wieder die Augen.}
	
	Es ist Morgen. Die Bürgerinnen von Palermo erwachen~-- bis auf eine. \emph{Die Spielleiterin teilt dem Opfer mit, dass es jetzt tot ist.}
\end{quotation}
	
Die Bürgerinnen diskutieren jetzt und entscheiden, welche beiden aus ihrer Mitte sie als verdächtigte Mörderinnen auf die Anklagebank setzen. Diese beiden Bürgerinnen halten dann je eine \emph{kurze} Verteidigungsrede.

Die Bürgerinnen entscheiden danach, wen sie hinrichten. Die Hingerichtete muss danach mitteilen, ob sie eine Mörderin war oder nicht.

In der nächsten Nacht nehmen das Mordopfer und die Hingerichtete nur noch beobachtend am Spiel teil. Sie brauchen die Augen nicht zu schließen, dürfen aber auch nicht reden oder den Lebenden irgendwelche Zeichen geben.

Das Spiel geht so lange, bis beide Mörderinnen hingerichtet sind oder alle Nicht-Mörderinnen ermordet sind. Danach kann neu gelost werden.

Hier ein paar unverfälschte Zitate vom DPO-Freak auf einer OE-Vorbereitungsfahrt:
\begin{itemize}
  \item "`Hängt ihn höher!"'
  \item "`Bombenleger!"'
  \item "`Randgruppen zuerst!"'
  \item "`Das war unser letztes Weibchen!"'
\end{itemize} 

\paragraph{Besondere Hinweise:} Das Spiel kann sehr heftig werden: Gegenseitige Beschuldigungen, Sündenböcke, Verschwörungen, Intrigen. Mit Demokratie oder Konsens hat das nichts mehr zu tun.
\paragraph{Wann einsetzen:} Abends beim Bier oder so.
\paragraph{Variante:} Die Werwölfe von Thiercelieux

Unter \url{http://www.mafiaspiel.de/} ist eine umfangreiche Website zu diesem Spiel und seinen Varianten zu finden. Außerdem veranstaltet die EMSA Bonn (\url{http://www.emsa-bonn.de}) regelmäßig Mafia-Runden.

\section{Kommunikationsmetaphern}
\index{Kommunikationsmetaphern}
\index{Metaphern, Kommunikations-}
\paragraph{Art:} Selbsterfahrung durch Herumlaufen und Übergabe von Gegenständen
\paragraph{Ziel:} die Grundprinzipien der Kommunikation verdeutlichen
\paragraph{Dauer:} 20--30 Minuten
\paragraph{Wir brauchen dazu:} $\geq 8$ Teilis, 1--2 Jonglierbälle und viele andere Gegenstände, die sich werfen oder übergeben lassen (ich hatte neulich einen Igelball, ein großes Messer, Spielknete, einen Plüsch-Elch, ein Gedicht hinter Glas mit Bilderrahmen, ein Trinkglas und eine Frisbee)
\paragraph{So geht es:} Das Spiel unterteilt sich in fünf Runden. Am Anfang/Ende jeder Runde trifft sich die Gruppe zum Stehkreis, um die letzte Runde auszuwerten und die Anleitung für die nächste Runde zu bekommen. Die Tutorin kann auch mitspielen.

Die fünf Runden:
  \begin{enumerate}
    \item Jede suchen sich eine gute Stelle im Raum und stellt sich dort hin. Dort verhält sie sich ruhig und nimmt ihre Umgebung wahr. Sie kann die Augen dabei offen oder geschlossen halten.
    \item Alle laufen kreuz und quer herum und schauen sich den Raum bzw.~die Umwelt an.
    \item Die gleichen Instruktionen mit dem Unterschied, dass die Tutorin ankündigt, in dieser Runde etwas ändern zu wollen. 1--2 Minuten nach dem Start gibt sie einer Teilnehmerin einen Jonglierball in die Hand und schaut, was passiert.
    \item Genauso, nur dass die Tutorin jetzt auch die anderen Gegenstände verteilt.
    \item Alle bleiben im Kreis stehen und geben dort die Gegenstände weiter.
  \end{enumerate}
  Bei der Auswertung nach jeder Runde kann die Tutorin etwa folgende Fragen stellen:
  \begin{itemize}
    \item Was habe ich mich, die anderen und die Umwelt wahrgenommen? Wie ging es mir dabei?
    \item Auf welche Kriterien habe ich geachtet?
    \item Was passiert, wenn sich zwei Spielerinnen begegnen?
    \item Was wird gebraucht, damit die Übergabe eines Gegenstandes funktioniert?
    \item Warum und wie ist die Übergabe bei den verschiedenen Gegenständen verschieden?
  \end{itemize}
  Bei der Auswertung kann (oder sollte?) herauskommen, dass die Gegenstände Metaphern für Nachrichten in der Kommunikation darstellen. Blickkontakt ist bei der Übergabe wichtig. Verschiedene Gegenstände sind verschieden beliebt, werden verschieden weitergegeben und regen unterschiedlich zur Kreativität an.
\paragraph{Besondere Hinweise:} Dieses Spiel ist eventuell problematisch bei Gruppen, die auf alles allergisch reagieren, was nach Selbsterfahrung aussieht.

Soweit ich weiß, stammt dieses Spiel von Ansgar Kemmann vom v.\,f.\,h.
\paragraph{Wann einsetzen:} am Anfang von Rhetorik- oder Kommunikationsseminaren oder bei der OE-Vorbereitungsfahrt

\section{Trotzburg}
\index{Trotzburg}
\index{Belagerte Stadt|see{Trotzburg}}
\paragraph{Alias:} Die belagerte Stadt
\paragraph{Art:} heftiges Rollenspiel um eine belagerte mittelalterliche Stadt
\paragraph{Ziel:} die Teilnehmerinnen für gruppendynamische Prozesse, Entscheidungsfindung in Gruppen, Sündenbockmechanismen und den Umgang mit Konflikten sensibilisieren
\paragraph{Dauer:} 1,5--2 Stunden (10 Minuten Vorbereitung, 1 Stunde Spiel, 20--50 Minuten Auswertung)
\paragraph{Wir brauchen dazu:} Stuhlhalbkreis (Podium) mit 5 Plätzen, Kopien der Rollenbeschreibungen (ab Seite \pageref{trotzburg-rollen}) für die 5 Spielerinnen, Kopien der Auswertungsbögen für die Beobachterinnen  (Seite \pageref{trotzburg-auswertung})
\paragraph{So geht es:}
Fünf Spielerinnen aus der Gruppe erhalten~-- am besten einige Zeit vor dem Spiel~-- ihre Rollenbeschreibungen. Sie spielen fünf Bürger einer mittelalterlichen Stadt: Bürgermeister, Arzt, Krankenpfleger, Wächter und Schmied.

Die anderen Spielerinnen übernehmen im folgenden Spiel eher passive Rollen: Sie beobachten und dokumentieren den Spielverlauf.

Die fünf Spielerinnen werden angewiesen, ihre Rollen genau zu studieren und die Informationen auf keinen Fall einer anderen Person mitzuteilen.

Die Spielleitung sollte in Rollenspielen unerfahrene Spielerinnen darauf hinweisen, dass sie versuchen sollen, sich gut in die beschriebene Rolle einzufühlen und im Spiel stimmig zu agieren. Sich in die Rolle einzufühlen bedeutet auch, die Rolle entsprechend auszubauen und sich zu überlegen, welche Informationen von seiner Rolle man im Spiel sofort, später oder überhaupt nicht preisgibt.

Die Spielleitung teilt der Gruppe kurz das Szenario mit:
\begin{quote}
Die mittelalterliche Stadt Trotzburg wird von den Hochbergern belagert. Sie beschuldigen die Trotzburger, einen Kaufmann umgebracht zu haben und fordern innerhalb einer Stunde die Auslieferung der schuldigen Person.
\end{quote}

Die Spielleitung kann der Gruppe auch gleich zu Beginn den Zweck des Spieles erläutern:
Es geht darum, Verhaltensweisen zu studieren, die in jeder Gruppe oder Gesellschaft ablaufen können, die aber in einer ausweglosen Situation wie in diesem Spiel besonders deutlich hervortreten. Daher sollten die Zuschauerinnen das Verhalten der Spielerinnen beobachten und auf den Auswertungsbögen dokumentieren (Seite~\pageref{trotzburg-auswertung}).

Sinnvoll ist es auch, einzelnen Zuschauerinnen bestimmte Spielerinnen zuzuordnen, so dass nicht alle alles beobachten.

Für die Spielerinnen wird eine Art kleiner Bühne vorbereitet, die zu den Zuschauerinnen hin offen ist, so dass der Spielverlauf gut beobachtet werden kann.

Nachdem die Vorbereitungen abgeschlossen sind, wird die Beratung der fünf Beteiligten gespielt, in der entschieden werden muss, wer den Hochbergern als Schuldiger ausgeliefert werden soll.

Es wird relativ schnell passieren, dass den fünf Bürger die Idee kommt, mit den Hochbergern zu reden. Wenn die Spielerinnen im Spiel auch weitgehend frei in ihren Entscheidungen ist: Die Forderung der Hochberger ist klar ausgesprochen, und sie werden auch nicht davon abrücken.
Ebenso werden sie nach genau einer Stunde die Stadt stürmen und niederbrennen, wenn nicht ein Schuldiger ausgeliefert wird.

Auch eine Verteidigung ist für das arme Städtchen Trotzburg nicht denkbar oder sinnvoll.

\paragraph{Besondere Hinweise:} Wie andere Rollenspiele mit einer starken Dynamik kann auch \emph{Trotzburg} weit über den Rahmen des Spiels hinausgehen und zu Wut, Ohnmachtsgefühlen o.\,Ä.~übergehen. Die Spielleitung sollte auf jeden Fall deutlich machen, dass die gespielte Entwicklung der Beratung nicht von speziellen Personen abhängig ist, sondern in allen Gruppen so ähnlich ablaufen wird.

Dieses Spiel hat die Form eines Entscheidungsspiels, ist aber der Sache nach eine Experiment, und man sollte es auch so vor der Gruppe bezeichnen. Denn die Situation ist so konstruiert, dass eine Entscheidung gegen den Sündenbock-Mechanismus kaum mehr möglich ist. Denkbar wäre höchstens, dass alle fünf sich stellen~-- was praktisch nie vorkommt, weil mindestens eine sich weigert und ihre Unschuld beteuert,~-- oder dass die Auszuliefernde ausgelost wird, nachdem alle eingesehen haben, dass sie mitschuldig sind.

In der Regel wird heftig gekämpft~-- meist mit unsachlichen Argumenten~--, bis schließlich einer, bei der es am leichtesten geht, alle Schuld zugeschoben wird.

Bisher habe ich folgende brauchbare Lösungen kennen gelernt:
\begin{itemize}
	\item Die Spielerinnen erzählen alle wahrheitsgemäß ihre Geschichte und stellen dabei fest, dass alle eine Teilschuld trifft. Sie losen aus, wer ausgeliefert wird. Die restliche Zeit nutzen sie, um eine alternative Lösung zu finden.
	\item Es wird ein Gefangener aus dem Kerker ausgeliefert, der ohnehin hingerichtet werden sollte (eine sehr kreative Lösung).
\end{itemize}

\paragraph{Wann einsetzen:} um die Teilnehmerinnen für Kommunikation, Entscheidungsfindung und Gruppenverhalten zu sensibilisieren

\section{Die heimliche Freundin}
\index{Heimliche Freundin}
\index{Freundin, Heimliche}
\paragraph{Art:} Nebenher-Spiel, bei dem jede Teilnehmerin eine heimliche Freundin hat, die ihr während der Veranstaltung immer wieder heimlich etwas Gutes tut
\paragraph{Ziel:} eine positive und kooperative Stimmung während der Veranstaltung schaffen, das Kennenlernen fördern, die Gruppe näher zusammenbringen
\paragraph{Dauer:} nebenher während der gesamten Veranstaltung, einen bis mehrere Tage lang
\paragraph{Wir brauchen dazu:}  Loszettel mit den Namen aller Spielerinnen, eine Tüte, Schüssel o.\,Ä.~für die Lose
\paragraph{So geht es:}
\begin{description}
\item[Die Vorbereitung:] Am ersten Tag der Veranstaltung werden Zettel mit den Namen aller Teilnehmerinnen in eine Tüte o.\,Ä.~geworfen. Danach zieht jede einen Zettel. Sollte jemand den eigenen Namen ziehen, wird die ganze Prozedur wiederholt.

\item[Das Spiel:] Danach kann das Spiel sofort losgehen: Jede Spielerin hat bis zum Ende der Veranstaltung die Aufgabe, der Spielerin, deren Namen sie gezogen hat, jeden Tag mindestens einmal etwas Gutes zu tun. Was das ist, bleibt ganz der Fantasie der Spielerinnen überlassen: Blumen auf dem Tisch vor dem Platz der Beglückten, eine Tafel Schokolade, ein Liebesbrief, ein frisch gemachtes Bett \ldots

Wichtig ist, dass die Beschenkte vor Spielende die Identität ihrer heimlichen Freundin nicht erfahren soll. Wer etwas Gutes tut, tut dies also heimlich. Natürlich ist es den Spielerinnen trotzdem erlaubt, zu knobeln und zu forschen, wer ihre heimliche Freundin sein könnte.

\item[Die Auflösung:] Zum Ende der Veranstaltung wird das Spiel aufgelöst. Dafür bieten sich beispielsweise an:
	\begin{description}
		\item[Die Party:] Bei tanzbarer Musik geben sich alle heimlichem Freundinnen der entsprechenden Beschenkten zu erkennen.
		\item[Der heiße Stuhl:] Nacheinander wird jede Spielerin mit verbundenen Augen auf einen Stuhl in der Mitte des Raumes gesetzt. Jetzt tut ihr ihre heimliche Freundin noch ein letztes Mal etwas Gutes. Sie tut es so, dass die Beschenkte dabei die Chance hat, die Identität ihrer heimlichen Freundin herauszufinden.
	\end{description}
\end{description}

\paragraph{Besondere Hinweise:} Mit Erwachsenen funktioniert dieses Spiel in der Regel sehr gut.

Das Spiel funktioniert nicht, wenn die Stimmung in der Gruppe sehr schlecht ist oder zwischen einzelnen Spielerinnen eine große Abneigung besteht.
\paragraph{Wann einsetzen:} Um bestehende Gruppen näher zusammenzubringen. Und um (auch bei neu zusammengekommenen Gruppen) für die Veranstaltung eine sehr angenehme Atmosphäre zu schaffen.

\section{Kreisflucht}
\index{Kreisflucht}
\paragraph{Art:} Errate-das-Spiel mit Moral zum Thema Kommunikation und Querdenken.
\paragraph{Ziel:} aus einem Kreis von Leuten entkommen
\paragraph{Dauer:} 5 Minuten.
\paragraph{Wir brauchen dazu:} mindestens 5 Teilis und Platz für einen Kreis.
\paragraph{So geht es:} Eine Teilnehmerin, die das Spiel noch nicht kennt, verlässt den Raum. Die Spielleiterin kann jetzt das Spiel erklären, bevor die Gruppe die Spielerin wieder hereinruft: Die Spielerinnen halten sich an den Händen und bilden einen Kreis um die eine Spielerin. Diese versucht nun, aus dem Kreis zu entkommen. Die Gruppe hat den Auftrag, die Spielerin auf keinen Fall durch zu lassen.

\subparagraph{Lösung:}
\rotatebox{180}{
	\begin{minipage}[c]{35em}
		Die Gruppe lässt die Spielerin nur dann durch, wenn sie (mit Worten) darum bittet.
	\end{minipage}
}

\paragraph{Besondere Hinweise:} Funktioniert pro Gruppe nur einmal.

Ich habe noch nicht ausprobiert, was passiert, wenn die Spielerin Judo oder Jiu Jitsu kann.
\paragraph{Wann einsetzen:} Einfach so (zum Lachen) oder als Einleitung für eine Arbeitseinheit zu expliziter Kommunikation.


\section{Gummibärchenanalyse}
\index{Gummibärchenanalyse}
\paragraph{Art:} eine Gruppe mit Gummibärchen, Playmobil oder Lego nachstellen
\paragraph{Ziel:} Rollen und Beziehungen in einer Gruppe reflektieren und explizit machen
\paragraph{Dauer:} 45--60 Minuten
\paragraph{Wir brauchen dazu:} pro Kleingruppe (2--4 Leute) ein leeres Moderationsplakat und einen Satz Moderationsstifte; Platz auf dem Boden oder auf Tischen, so dass jede Kleingruppe ihr Plakat zum Arbeiten ausbreiten kann; einen großen Haufen Playmobil (Figure, Pferde, Waffen, Möbel \ldots) oder Lego oder eine \emph{sehr} breite Auswahl an verschiedenen Haribo-Figuren; Kleber zum Festkleben der Gummibärchen (nicht bei Lego oder Playmobil!)
\paragraph{So geht es:} Die Teilis bilden Kleingruppen à 2--4 Teilnehmerinnen. Auf einem Moderationsplakat baut nun jede Kleingruppe die Gruppe nach, deren Beziehungen und Rollen sie darstellen möchte, zum Beispiel: Der König (Vorsitzende) sitzt auf seinem Thron, hinter seinem Rücken stehen zwei andere Figuren und tuscheln \ldots

Mit den Moderationsstiften lassen sich noch Pfeile, Trennlinien, Namen und Ähnliches aufmalen, um die Gruppenstruktur noch deutlicher zu machen.

Wenn alle Kleingruppe fertig sind (hier schadet ein bisschen Zeitdruck von der Moderatorin nicht), stellt jede Kleingruppe ihr Kunstwerk dem Plenum vor. Applaus!
\paragraph{Besondere Hinweise:} Wenn sich das Plenum aus verschiedenen Organisationen zusammensetzt (zum Beispiel wenn Leute aus einer Fachschaft und vom NaBu da sind), sollten sich die Kleingruppen möglichst nach den Gruppen sortieren, deren Struktur nachgebaut werden soll.

Wenn viele Leute aus einer Gruppe da sind, können die Kleingruppen auch die gleiche Gruppe darstellen, so dass dabei unterschiedliche Sichtweisen und Darstellungen herauskommen.

Diese Methode funktioniert möglicherweise nicht, wenn der fiese Chef oder ein extrem unbeliebtes Gruppenmitglied anwesend ist und die Teilis dann ihre ehrliche Sicht der Gruppe nicht darstellen möchten.
\paragraph{Wann einsetzen:} um eine Gruppe spielerisch über ihre eigenen Strukturen, Rollen und Beziehungen reflektieren zu lassen


\chapter{Feedback und Auswertung}
\index{Feedback}
\index{Auswertung}

\section{Standpunkte}
Auf Seite \pageref{standpunkte} zu finden.

\section{Erwartungsposter}
\index{Erwartungsposter}
\index{Poster, Erwartungs-}
\paragraph{Art:} Kartenabfrage zu Erwartungen und Befürchtungen an eine Veranstaltung
\paragraph{Ziel:} Positive und negative Erwartungen für alle sichtbar machen (vor allem für die Tutorin)
\paragraph{Dauer:} 10--15 Minuten
\paragraph{Wir brauchen dazu:} vorbereitetes Plakat, Moderationskarten in 2 Farben, Moderationsstifte, Klebestift (oder Krepp-Klebeband)
\paragraph{So geht es:} Die Tutorin hat ein Plakat mit der Überschrift \textit{Was erwarte ich von diesem Seminar (der OE etc.)?} und zwei Spalten vorbereitet:
\begin{itemize}
\item Das Seminar wird gut wenn, \ldots
\item Das Seminar wird nicht so gut wenn, \ldots
\end{itemize}
Die Hintergrundfarben der beiden Spaltenüberschriften sollten mit den beiden Farben der Moderationskarten übereinstimmen.

Wenn das Plakat hängt, bekommen alle Teilnehmerinnen Karten und Stifte. Die Tutorin sammelt die fertigen Karten ein, mischt sie und klebt sie an (geordnet nach den beiden Farben).
\paragraph{Besondere Hinweise:} Eventuell sollte die Tutorin nach dieser Aktion das Programm anpassen, wenn abzusehen ist, dass die Teilnehmerinnen etwas überhaupt nicht mögen werden.

Am Ende der Veranstaltung kann es interessant sein, zusammen mit den Teilnehmerinnen zu schauen, welche Erwartungen sich erfüllt, nicht erfüllt oder verändert haben.
\paragraph{Wann einsetzen:} Am Anfang einer Veranstaltung

\section{Blitzlicht}
\index{Blitzlicht}
\paragraph{Art:} Feedback oder Momentaufnahme.
\paragraph{Ziel:} Die Stimmung in der Gruppe wird sichtbar.
\paragraph{Dauer:} Pro Person maximal eine Minute.
\paragraph{Wir brauchen dazu:} ---
\paragraph{So geht es:} Jede Teilnehmerin bekommt eine Minute "`Sprechzeit"'. Darin kann sie ein kurzes Statement dazu abgeben, wie sie sich momentan fühlt; ob sie zufrieden ist mit dem, was sie erlebt hat; wie die Zusammenarbeit in der Gruppe klappte usw.
\paragraph{Besondere Hinweise:} Jede kommt zu Wort, die Aussagen werden nicht diskutiert oder gewertet. Auch die Tutorinnen haben die Möglichkeit etwas zu sagen.
\paragraph{Wann einsetzen:} Wenn sich Schwierigkeiten bemerkbar machen. Oder als Feedback am Ende des Tages oder zum Abschluss eines Themenbereichs.

\section{Auswertungsgalerie}
\index{Auswertungsgalerie}
\index{Galerie, Auswertungs-}
\paragraph{Art:} Feedback und Nachbereitung
\paragraph{Ziel:} Anonymes, für alle sichtbares Feedback zu einer Veranstaltung
\paragraph{Dauer:} 15--30 Minuten
\paragraph{Wir brauchen dazu:} Mehrere Pinnwände oder Wände, vorbereitete Plakate, viele Moderationsstifte
\paragraph{So geht es:} Die Tutorin hat Plakate mit Fragen vorbereitet, zu denen sie etwas von den Teilnehmerinnen erfahren möchte. Mögliche Fragen:
\begin{itemize}
\item Die Seminarmoderation: Was hat mir gut gefallen, was hat mir nicht so gefallen?
\item Wie hat mir die Unterkunft gefallen?
\item Wie ging es mir mit der Gruppe auf diesem Seminar?
\item ein großer gemalter Koffer: Was ich von diesem Seminar an Erfahrungen und Wissen mit nach Hause nehme:
\item ein großer gemalter Mülleimer: Was ich lieber hier lassen möchte:
\item Was ich sonst noch sagen möchte:
\end{itemize}
Wenn alle Plakate aufgehangen sind und genügend Stifte in der Nähe jedes Plakats liegen, können sich die Teilnehmerinnen ans Werk machen und ihre Gedanken zu Plakat bringen.

Wenn niemand mehr etwas schreiben möchte, ist die Galerie noch einmal für alle zum Anschauen eröffnet.
\paragraph{Besondere Hinweise:} Schau als Tutorin den Teilnehmerinnen nicht beim Schreiben über die Schulter! (Sonst fühlen sie sich beobachtet, und es ist nicht mehr anonym.)
\paragraph{Varianten:} Beliebt ist auch die Variante "`Koffer und Mülleimer"'.
\paragraph{Wann einsetzen:} Am Ende der OE, eines Seminars oder einer längeren Arbeitseinheit

\section{Vier Felder}
\index{Vier Felder}
\index{Felder, vier}
\index{4 Felder}
\paragraph{Art:} Feedback und Nachbereitung
\paragraph{Ziel:} Anonymes, für alle sichtbares Feedback, das zu Widersprüchen und Kritik ermutigt
\paragraph{Dauer:} 5--10 Minuten
\paragraph{Wir brauchen dazu:} Pinnwand (oder Wand), das vorbereitete Plakat, Moderationskarten in den Farben Orange, Rot, Blau und Weiß, einen Moderationsstift pro Teilnehmerin
\paragraph{So geht es:} Die Tutorin hat ein Plakat vorbereitet, das in vier Bereiche mit je einer Frage geteilt ist. Die kursiven Wörter stehen auf einer Moderationskarte in der entsprechenden Farbe:
\begin{itemize}
\item Ein Gedanke, der mich \emph{fasziniert}: (orange Moderationskarten)
\item Ein Gedanke, dem ich \emph{nicht} zustimme: (rote Moderationskarten)
\item Was mir \emph{klar(er)} geworden ist: (blaue Moderationskarten)
\item Was mir \emph{unklar} (geblieben) ist: (weiße Moderationskarten)
\end{itemize}

Jede Teilnehmerin bekommt einen Moderationsstift sowie pro Farbe eine Moderationskarte (also insgesamt vier Karten pro Nase). Dann füllen alle ihre Karten aus und pinnen sie an. Dabei muss nicht jede Teilnehmerin für jedes Feld etwas schreiben.

Alternativ kann die Seminarleiterin auch die Karten einsammeln und aufhängen.

\paragraph{Wann einsetzen:} Am Ende eines Tages oder des kompletten Seminars.

\section{OE-Feedback-Bögen}
\index{OE-Feedback-Bögen}
\index{Feedback-Bögen}
\paragraph{Art:} schriftliche "`Umfrage"'
\paragraph{Ziel:} Feedback zur OE
\paragraph{Dauer:} 5--10 Minuten
\paragraph{Wir brauchen dazu:} pro Person einen Feedback-Bogen (Seite \pageref{fragebogen})
\paragraph{So geht es:} Alle Teilis bekommen einen Feedback-Bogen, füllen ihm anonym aus und geben ihn der Tutorin zurück.
\paragraph{Besondere Hinweise:} Die Zettel werden bei den Gruppen-Packs dabei sein. Bitte an beiden Tagen je einen Bogen ausfüllen lassen.
\paragraph{Wann einsetzen:} Donnerstag und Freitag, je am Ende der Arbeit in Gruppen. Wenn jemand früher geht, dann lasst sie bitte auch einen Bogen ausfüllen!

\section{Energiepegel-Anzeige}
\index{Energiepegel-Anzeige}
\index{Pegel-Anzeige, Energie-}
\paragraph{Art:} alle zeigen mit der Hand ihren persönlichen Energiepegel
\paragraph{Ziel:} schnelles Feedback darüber, wie viel Energie jede aus der Gruppe im Moment hat
\paragraph{Dauer:} 2 Minuten
\paragraph{Wir brauchen dazu:} Steh- oder Sitzkreis, in dem sich alle sehen können
\paragraph{So geht es:} Die Tutorin gibt vor, welche räumliche Höhe den Maximalpegel darstelle (z.\,B.~ Gürtelhöhe, Brusthöhe, Scheitel oder so). Der Boden bedeutet \textit{keine Energie}. Dann zeigen alle gleichzeitig mit der Hand, wie viel Energie sie im Moment noch haben.
\paragraph{Wann einsetzen:} Vor längeren Arbeitseinheiten, oder wenn die Gruppe insgesamt irgendwie schlapp aussieht. Oder später am Tag vor einer nicht mehr elementar wichtigen "`Zusatz-Arbeitseinheit"'.

\section{Theater}
\index{Theater}
\index{Raumschiff Enterprise|see{Theater}}
\index{Enterprise|see{Theater}}
\index{Piratenschiff|see{Theater}}
\paragraph{Alias:} Raumschiff Enterprise, Piratenschiff
\paragraph{Art:} alle positionieren sich in einem gemalten Theatergrundriss
\paragraph{Ziel:} Tages- oder Seminar-Feedback
\paragraph{Dauer:} 15--20 Minuten
\paragraph{Wir brauchen dazu:} vorbereitetes Plakat, Eddings oder Klebepunkte (evtl. in mehreren Farben)
\paragraph{So geht es:} Auf das Plakat hat die Tutorin den Grundriss eines Theaters gemalt: Bühne, Garderobe für Gäste, Foyer, Sitzplätze, Stehplätze, Loge, Regie, Maske, Künstlerinnengarderobe, Telefonzellen, Klos, Park, Technik, Bar \ldots

Alle Teilnehmerinnen tragen sich nun ein an der Stelle, an der sie sich (im übertragenen Sinne) heute (oder während des Seminars) gesehen haben. Wer sich gar nicht entscheiden kann, darf sich auch doppelt eintragen.\footnote{Oder sie trägt sich als Schmierfleck ein~-- als eine Raum-Zeit-Anomalie gewissermaßen.}
\paragraph{Varianten:} Wenn das Feedback anonym sein soll, können die Teilis auch Klebepunkte kleben, anstatt ihre Namen einzutragen.

Für ein zeitlich genaueres Feedback kann die Tutorin auch mehrere Farben benutzen (für Vormittag, Nachmittag, Abend \ldots). Dann sollte sie aber eine Legende in das Plakat integrieren.

Eine Alternative zum Theater kann auch ein anderes Gebäude, die \emph{Enterprise} oder ein Piratenschiff sein.
\paragraph{Wann einsetzen:} Am Ende eines Tages oder einer Veranstaltung

\section{Hand-Feedback}
\index{Hand-Feedback}
\index{Feedback, Hand-}
\paragraph{Art:} die fünf Finger einer Hand entsprechen fünf Fragen
\paragraph{Ziel:} Tages- oder Seminarkritik
\paragraph{Dauer:} 10--15 Minuten
\paragraph{Wir brauchen dazu:} ein Plakat, auf das eine große Hand gemalt ist, die Finger beschriftet mit den unten stehenden Fragen
\paragraph{So geht es:} Beginnend beim Daumen, geben die Teilis nacheinander ihr Feedback zu allen Fragen auf dem Plakat:
  \begin{description}
    \item[Daumen:] Daumen hoch für \ldots
    \item[Zeigefinger:] Darauf möchte ich hinweisen \ldots
    \item[Mittelfinger:] Im Mittelpunkt stand für mich \ldots
    \item[Ringfinger (mit Ring):] Mein Schmuckstück heute/auf dem Seminar war \ldots
    \item[Kleiner Finger:] Zu kurz kam für mich \ldots
  \end{description}
\paragraph{Besondere Hinweise:} Diese Methode habe ich von Vera Derschum vom v.\,f.\,h.~gelernt.
\paragraph{Wann einsetzen:} als Tages- oder Seminarkritik

\section{Ampel-Feedback}
\index{Ampel-Feedback}
\index{Feedback, Ampel-}
\index{Ampelreflexion}
\paragraph{Alias:} Ampelreflexion
\paragraph{Art:} Zustimmung zu Aussagen per Kartenheben darstellen
\paragraph{Ziel:} schnelle, überblickartige Tageskritik
\paragraph{Dauer:} 5--10 Minuten
\paragraph{Wir brauchen dazu:} Steh- oder Stuhlkreis, pro Teili je eine rote, gelbe und grüne Moderationskarte
\paragraph{So geht es:} Jede Teilnehmerin bekommt von jeder der drei Farben je eine Moderationskarte. Dann sagt nacheinander jede eine Behauptung (zum Beispiel: "`Ich habe viel Neues gelernt."'). Alle heben daraufhin eine der drei Moderationskarten, um ihre Zustimmung oder Ablehnung zu dieser Aussage zu zeigen:
\begin{itemize}
  \item grün: ich stimme zu
  \item gelb: ich weiß nicht (oder möchte mich dazu nicht äußern)
  \item rot: ich stimme nicht zu
\end{itemize}

\paragraph{Besondere Hinweise:} Danke an Björn Krüger für diese Methode.
\paragraph{Wann einsetzen:} Zur Tageskritik. Als Seminarkritik ist die Methode zu wenig qualitativ.

\section{Sektreflexion}
\index{Sektreflexion}
\index{Ich stoße an auf \ldots}
\paragraph{Alias:} Ich stoße an auf \ldots
\paragraph{Art:} Feedback durch Trinken verdeutlichen
\paragraph{Ziel:} kurzes, lustig aufgemachtes Feedback zum Tag oder Seminar
\paragraph{Dauer:} 5--10 Minuten
\paragraph{Wir brauchen dazu:} Sitz- oder Stehkreis, ein Getränk pro Person (Sekt, O-Saft, Bier \ldots)
\paragraph{So geht es:} Wer etwas sagen möchte (entweder der Reihe nach oder wer gerade Lust hat), fängt einen Satz mit einer der beiden folgenden Floskeln an:
\begin{itemize}
  \item "`Ich stoße an auch \ldots"'
  \item "'Ich spüle meinen Ärger hinunter über \ldots"'
\end{itemize}
Nach der Aussage trinken alle einen Schluck.

\paragraph{Besondere Hinweise:} Lässt sich nur anwenden, wenn genug Vertrauen zwischen den Teilis besteht, so dass diese Kritik auch nicht-anonym äußern können.

Danke an Björn Krüger für diese Methode.
\paragraph{Wann einsetzen:} Zur Tages- oder Seminarkritik.

\section{Writer's Workshop}
\index{Writer's Workshop}
\paragraph{Art:} Die Teilnehmerinnen geben Feedback zu einem Text, während die Autorin zwar zuhört, sich aber nicht äußern darf.
\paragraph{Ziel:} Feedback an die Autorin eines Papers (oder eines anderen Textes)
\paragraph{Dauer:} bis zu einer Stunde pro Text
\paragraph{Wir brauchen dazu:} einen Stuhlkreis~-- außerdem müssen alle Teilis vorher die Texte gelesen haben, die diskutiert werden
\paragraph{So geht es:}
Der Fokus eines \emph{Writer's Workshop} liegt auf dem Paper, nicht so sehr
auf der Präsentation. Tatsächlich findet es bei einem \emph{Writer's Workshop}
gar keine Präsentation eines Papers statt, sondern alle Teilnehmerinnen müssen das Paper vorher gelesen haben.

Das Format stammt aus Schriftstellerkreisen, die es verwenden, um sich
gegenseitig Gedichte, Kurzgeschichten und Ähnliches vorzustellen. (Dabei wird
kein Vortrag gehalten, sondern eine Passage aus dem Werk vorgelesen.)

Es ist dann Anfang bis Mitte der 90er auf den Pattern-Konferenzen
(PLoP, EuroPLoP \ldots) von Richard Gabriel eingeführt worden, um
Patterns zu besprechen, und hat sich dort bis heute gehalten. Richard Gabriel hat in \cite{writers} das Format detailliert beschrieben.

\paragraph{Ablauf:}

\begin{enumerate}
\item Die Autorin liest der Runde einen ihrer Meinung nach besonders
gelungenen Absatz aus ihrem Paper vor.

\item Anschließend tritt die Autorin aus der Runde und wird zur "`Fliege an
der Wand"' ("`fly on the wall"'). Eine "`Fliege an der Wand"' ist zwar
anwesend, aber so unscheinbar, dass alle anderen Anwesenden
die "`Fliege"' ignorieren. Die "`Fliege"' hört zu und macht sich Notizen, nimmt aber
nicht an der Diskussion teil und wird auch von den anderen nicht direkt
angesprochen. (Es heißt immer "`Die Autorin hat geschrieben \ldots"', nicht "`Du
hast geschrieben \ldots"'.)

\item Die Diskussion (wie gesagt, ohne die Autorin) läuft in vier Phasen ab:

\begin{enumerate}
\item \emph{Zusammenfassung}: Zuerst fasst jemand den Inhalt des Papers zusammen.
Die anderen können dies noch kommentieren und ergänzen werden, etwa falls 
sie in einem Punkt eine unterschiedliche Auffassung vertreten.

\item \emph{Positive Bestätigung:} Die Teilnehmerinnen versuchen
darzustellen, was ihnen am Paper gefallen hat. Dies kann sowohl Inhalt
als auch Darstellung umfassen: "'neue Erkenntnis"', "`überzeugende
Argumentation"', "`gute praktische Einsetzbarkeit"', aber auch: "`gute
Illustrationen"', "`schöne Formulierungen"', "`lesbare Zeichensätze"'.

\item \emph{Verbesserungsvorschläge:} Die Teilnehmerinnen machen
Vorschläge, wie das Paper verbessert werden könnte. Kritik darf
dabei nur mit dem Hintergedanken der Verbesserung formuliert werden.
Auch hier können alle Bereiche angesprochen werden~-- Inhaltliches und
Formales: "`Der Beweis ist unklar und sollte an folgenden Stellen
näher erläutert werden."', "`Ein UML-Diagramm würde zum Verständnis
beitragen."'

\item \emph{"`Sandwich"':} In der letzten Diskussionsrunde betonen die Teilis nochmals die
positiven Merkmale des Papers: Zum einen können einigen
Teilnehmerinnen aufgrund der Verbesserungsvorschläge noch weitere positive
Merkmale klar geworden sein. Zum anderen wird so verdeutlicht, dass es
sich trotz der Kritik um einen wertvollen Beitrag handelt.
\end{enumerate}

\item Nach der Diskussion kehrt die Autorin in die Runde zurück. Sie hat
Gelegenheit, Verständnisfragen zu stellen ("`Wie war die Bemerkung zu x
gemeint?"', "`Der Hinweis y war interessant~-- welche Literatur gibt es
dazu?"'), aber sie darf an dieser Stelle keine Erklärungen oder Verteidigungen des
Papers nachreichen. (Dies kann sie in den Pausen oder bei den \emph{social
events} im kleineren Kreis nachholen.)

\item Die Teilnehmerinnen stehen auf und applaudieren der Autorin.
Anschließend erzählt jemand einen Witz oder irgendeine Anekdote, die
in keinerlei Zusammenhang zu dem Paper steht.
\end{enumerate}

Manche Punkte in dieser Vorgehensweise erscheinen vielleicht etwas
merkwürdig, insbesondere Punkt~5. Aber es zeigt sich bei \emph{Writer's Workshops},
dass alle Punkte zusammen zu einer sehr konstruktiven
Atmosphäre führen. Insgesamt erhält die Autorin wertvolles Feedback,
insbesondere, was die Verständlichkeit und Darstellung ihres Papers anbelangt,
aber auch inhaltliche Hinweise.

In jedem Schritt (insbesondere in 3.c und 5., aber auch dadurch, dass die
Autorin nicht an der Diskussion teilnimmt) hilft, zu vermeiden, dass sich
jemand angegriffen fühlt oder aber Energie in unnötige Verteidigungen steckt. Beides ist destruktiv und hilft nicht bei einem der wichtigsten Ziele wissenschaftlicher Arbeit: nämlich verständliche und
erfolgreiche Papers zu schreiben. (Es ist eine Sache, in einem Vortrag
zwanzig Teilnehmer zu überzeugen, aber es ist eine andere, ein Paper so
zu schreiben, dass es auch ohne weitere Erklärung verständlich und
überzeugend ist.)

\paragraph{Besondere Hinweise:} Vielen Dank an Pascal Costanza für diese Methode.
\paragraph{Wann einsetzen:} Um vor einer Konferenz oder auf einen Schreibworkshop den Autorinnen Feedback zu ihren Texten zu geben.

\section{Zettel auf dem Rücken}
\index{Zettel auf dem Rücken}
\index{Worte verschenken|\see{Zettel auf dem Rücken}}
\paragraph{Alias:} Worte verschenken
\paragraph{Art:} sehr nettes persönliches Feedback zwischen den Teilis
\paragraph{Ziel:} jede Teilnehmerin darf den anderen Teilnehmerinnen noch nette Nachrichten mit auf den Weg geben
\paragraph{Dauer:} 5--10 Minuten
\paragraph{Wir brauchen dazu:} pro Teili je 1~etwa A3 großes Stück Packpapier, 1 schwarzen oder blauen Moderationsstift und ein paar Streifen Moderations-Klebeband
\paragraph{So geht es:} Jede Teilnehmerin klebt einer anderen Teilnehmerin mit Klebeband ein Stück Packpapier auf den Rücken. Dann schreibt jede Teili jeder anderen Teilnehmerin, der sie noch etwas auf den Weg geben möchte, eine Nachricht auf den Zettel, den diese auf dem Rücken trägt. Es muss allerdings nicht jede Teilnehmerin allen anderen etwas aufschreiben~-- sondern nur denen, denen sie noch etwas mitteilen möchte.

Die Nachrichten sollten nach Möglichkeit positiv sein, damit niemand auf dem Nachhauseweg traurig ist.

Die Mitteilungen sind dabei pseudo-anynom: Man muss sich nicht outen, aber oft ist es trotzdem klar, von wem eine Nachricht stammt.

\paragraph{Besondere Hinweise:} Achtet darauf, dass ihr auf jeden Fall nicht-durchschreibende Moderationsstifte benutzt (also die Neuland-Stifte statt der Eddings benutzen)!
\paragraph{Wann einsetzen:} ganz am Ende des Seminars nach der "`offiziellen"' Auswertung



\backmatter
\bibliography{spielereader}

\appendix
\chapter{Intelligenztest}
\label{iq}
\medskip
Datum: \rule{3cm}{0,4pt}
\bigskip

Lesen Sie bitte zuerst alle Fragen gründlich durch, bevor Sie sie beantworten. Sie haben insgesamt drei Minuten Zeit.

\begin{enumerate}
\item Wer komponierte die Oper \emph{Aida?} \hrulefill
\item Wer schrieb das Buch \emph{Krieg und Frieden?} \hrulefill
\item Wo fand 1954 die Fußballweltmeisterschaft statt? \hrulefill
\item Setzen Sie die Zahlenreihe fort: 2~--~4~--~6~-- \hrulefill
\item Von wem stammen die Figuren \emph{Max und Moritz?} \hrulefill
\item Wann lebte Karl der Große? \hrulefill
\item Wie viele Kontinente gibt es auf der Erde? \hrulefill
\item Wer erfand die Glühbirne? \hrulefill
\item Welches Land produziert das meiste Öl? \hrulefill
\item Wie heißt das westliche Verteidigungsbündnis? \hrulefill
\item Wie viele Tierkreiszeichen gibt es (Stier, Wassermann etc.)? Nur die Anzahl: \hrulefill
\item Durch wen wurde Napoleon besiegt? \hrulefill
\item Füllen Sie nur das heutige Datum oben links aus. Alles andere können Sie sich sparen. Genießen Sie noch zwei Minuten Ruhe.
\end{enumerate}

\chapter{Kennenlern-Bingo}
\label{bingo}
Gehe im Raum herum und finde Personen, die den Anforderungen in den Kästchen entsprechen. Für jedes Kästchen soll eine Person gefunden werden, die dann in dem entsprechenden Kästchen unterschreibt.

Wer vier Kästchen in einer Reihe ausgefüllt bekommt~-- waagerecht, senkrecht oder schräg~--, ruft laut \fett{"`Bingo!"'}. Je mehr \emph{Bingo}s, desto besser.

Denk daran, dass es Ziel des Spieles ist, die anderen TeilnehmerInnen kennen zu lernen: Deshalb unterhalte dich ruhig ein wenig mit ihnen, auch wenn du schon die Unterschrift hast!

\subsection*{Finde jemanden, der/die \ldots}

\renewcommand{\arraystretch}{1.27}
\noindent\begin{tabular}{|p{9.0em}|p{9.0em}|p{9.0em}|p{9.0em}|}
\hline
\ldots\ eine Sprache spricht, die du überhaupt nicht sprichst:&
\ldots\ ein Haustier hat:&
\ldots\ die gleiche Augenfarbe wie du hat:\vspace{5.5em}&
\ldots\ in den letzten drei Jahren einmal (als PatientIn) im Krankenhaus war:\\
\hline
\ldots\ mindestens ein Jahr außerhalb Deutschlands gelebt hat:&
\ldots\ das gleiche Musikinstrument spielt wie du:\vspace{4.5em}&
\ldots\ eine andere Staatsangehörigkeit als du hat:&
\ldots\ im gleichen Monat Geburtstag hat wie du:\\
\hline
\ldots\ den gleichen Lieblingsfilm hat wie du:&
\ldots\ etwas Handgemachtes trägt:\vspace{5.5em}&
\ldots\ im gleichen Jahr geboren wurde wie du:&
\ldots\ keinen Fernseher hat:\\
\hline
\ldots\ gleich viele Geschwister wie du hat:\vspace{5.5em}&
\ldots\ das gleiche Hauptverkehrsmittel wie du hat:&
\ldots\ in einem Land war, in dem du noch nie warst:&
\ldots\ denselben Sport wie du betreibt:\\
\hline
\end{tabular}
\renewcommand{\arraystretch}{1.0}

\chapter{NASA-Spiel}
\label{nasa-kopien}

Sie sind Mitglied einer Raumfahrtgruppe, die ursprünglich geplant hatte, auf der erhellten Oberfläche des Mondes mit einem Mutterschiff zusammenzutreffen. Infolge technischer Schwierigkeiten ist Ihr Raumschiff jedoch gezwungen worden, an einer Stelle in der Tagzone zu landen, die etwa 300~km von dem Treffpunkt entfernt liegt. Während der Landung ist ein großer Teil der Ausrüstung an Bord beschädigt worden.

Da die Aussicht zu überleben davon abhängt, ob Sie das Mutterschiff erreichen, müssen die wichtigsten der vorhandenen Dinge für den 300~km langen Weg gewählt werden. Unten finden Sie eine Liste von 15 Gegenständen, die nach der Landung unbeschädigt geblieben sind. Ihre Aufgabe ist es, diese Gegenstände in eine Rangordnung zu bringen, je nachdem, wie notwendig Sie Ihnen zum Erreichen des Treffpunktes erscheinen. Setzen Sie Nummer~1 neben den wichtigsten Gegenstand, Nummer~2 neben den zweitwichtigsten usw.

\subsection*{Liste der unbeschädigten Dinge}
\renewcommand{\arraystretch}{1.0}
\begin{tabular}{|lp{20em}|l|l|l|}
  \hline
  & & \multicolumn{3}{|c|}{\fett{Rangordnung}} \\
  \multicolumn{2}{|c|}{\fett{Artikel}} & \fett{Individuell} & \fett{Gruppe} & \fett{Plenum} \\
  \hline \hline
	1 & Schachtel Streichhölzer & & & \\
  \hline
	1 & Dose Nahrungskonzentrat pro Person (lässt sich mit einem Spezialventil an den Raumanzug anschließen) & & & \\
  \hline
	15~m & Nylonseil & & & \\
  \hline
	30~m$^2$ & Fallschirmseide (15\,$\times$\,2~m) & & & \\
  \hline
	1 & tragbares Heizgerät (mit Infrarot-Strahler als Wärmequelle) & & & \\
  \hline
	2 & Pistolen 7,654~mm & & & \\
  \hline
	1 & kleine Kiste Trockenmilch pro Person & & & \\
  \hline
	2 & Sauerstofftanks zu je 50~l pro Person & & & \\
  \hline
	1 & Sternenkarte (aus Mondperspektive) & & & \\
  \hline
	1 & Schlauchboot (automatisch aufblasbar durch integrierte CO$_2$-Kartuschen) & & & \\
  \hline
	1 & Magnetkompass & & & \\
  \hline
	22~l & Wasser pro Person (mit Spezialventil an den Raumanzug anschließbar) & & & \\
  \hline
	20 & Signalpatronen (auch im luftleeren Raum zündend, ohne Pistole abschießbar) & & & \\
  \hline
	1 & Erste-Hilfe-Koffer (u.\,a.~mit Injektionsnadeln) & & & \\
  \hline
	1 & Fernmelde-Empfänger und -Sender mit Solarzellen & & & \\
  \hline
\end{tabular}

\chapter{Schiffbruch}
\label{schiffbruch-kopien}
Sie segeln zu Weihnachten mit einer Privatjacht auf dem offenen Meer, etwa 800~Seemeilen südöstlich von Südafrika, als an Bord ein Brand ausbricht. Sie können nur noch die unten augelisteten 15~Gegenstände in das einzige vorhandene Rettungsboot mitnehmen. Da das Boot damit aber überlastet ist, müssen Sie sich einigen, welche Artikel Sie wegwerfen und welche Sie behalten wollen.

Ihre Aufgabe ist es, diese Gegenstände in eine Rangordnung zu bringen, je nachdem, wie notwendig Sie Ihnen in dieser Situation erscheinen. Setzen Sie Nummer 1 neben den wichtigsten Gegenstand, Nummer 2 neben den zweitwichtigsten usw.

Einige der Schiffbrüchigen haben außerdem Streichhölzer und Geld (Münzen und Scheine) dabei.

\subsection*{Die Liste der Gegenstände}

\renewcommand{\arraystretch}{1.27}
\begin{tabular}{|lp{20em}|l|l|l|}
  \hline
  & & \multicolumn{3}{|c|}{\fett{Rangordnung}} \\
  \multicolumn{2}{|c|}{\fett{Artikel}} & \fett{Individuell} & \fett{Gruppe} & \fett{Plenum} \\
  \hline \hline
  1 & Sextant (ohne weitere Dokumente) & & & \\
  \hline
  1 & Rasierspiegel & & & \\
  \hline
  25~l & Trinkwasser & & & \\
  \hline
  1 & großes Moskitonetz & & & \\
  \hline
  1 & Nahrungsration, reicht für 1~Tag (pro Person) & & & \\
  \hline
  1 & Karte des Indischen Ozeans & & & \\
  \hline
  1 & aufblasbares Kopfkissen & & & \\
  \hline
  24~l & Dieselöl & & & \\
  \hline
  1 & FM-Transistorradio & & & \\
  \hline
  1~l & Haifisch-Abwehr-Flüssigkeit & & & \\
  \hline
  10 m$^2$ & Plastikfolie & & & \\
  \hline
  1,5~l & Cognac & & & \\
  \hline
  5~m & Nylonschnur & & & \\
  \hline
  400~g & Schokolade (pro Person) & & & \\
  \hline
  1 & Angel mit Zubehör & & & \\
  \hline
\end{tabular}
\renewcommand{\arraystretch}{1.0}

\chapter{Insel ohne Wiederkehr}
\label{wiederkehr-kopien}
Nachdem Sie mit allen Personen und (erstaunlicherweise) allen Gegenständen von der Jacht fliehen, ohne dass das Rettungsboot sank, entdecken Sie noch einige weitere Gegenstände im Rettungsboot (fragen Sie besser nicht genauer nach, wie das Vieh dort hinein gekommen ist, warum es noch lebt und warum Sie es vorher nicht bemerkt haben \ldots).

\vspace*{.5em}
\begin{tabular}{rp{30em}}
  1 & Hammer \\
  1 & Säge \\
  1 & Packung Nägel \\
  1 & Gewehr \\
  500 & Schuss Munition für das Gewehr \\
  1 & Kuh \\
  1 & Stier \\
  1 & grobe Karte von Südafrika und dem südlich davon liegenden Meer (siehe Seite \pageref{wiederkehr-karte}) \\
\end{tabular}
\vspace*{.5em}

Auf der Karte ist zusätzlich noch eine Flug- und Schifffahrtslinie eingezeichnet, auf der in Ihnen nicht bekannten Abständen (Stunden? Wochen?) Flugzeuge und Schiffe verkehren. Mit dem verbleibenden Sprit wäre diese Linie gerade so zu erreichen. Wenn Sie dort von einem Flugzeug oder Schiff gesehen würden, könnten Sie gerettet werden und in die Zivilisation zurückkehren.

Weiterhin ist auf der Karte fernab aller Flug- und Schifffahrtslinien die \emph{Insel ohne Wiederkehr} eingezeichnet, auf die von allen Seiten Strömungen zulaufen. Mit dem verbleibenden Sprit wäre diese Insel ebenfalls gerade so zu erreichen. Wer einmal dort gestrandet ist, kann allerdings ohne Rettung von außen nicht mehr von dort entkommen. Die Überlebenschancen auf der Insel sind hingegen gut: Das Klima ist sehr angenehm, es gibt Wasser, viele essbare Pflanzen und nur wenig gefährliche Tiere.

Sie zeichnen auf der Karte sofort Ihre aktuelle Position ein. (Die Jacht ist inzwischen komplett gesunken.) Was tun Sie?

\pagebreak
\section*{Die Karte aus dem Rettungsboot}
\scalebox{0.4}[0.4]{\includegraphics*{insel-ohne-wiederkehr}}
\label{wiederkehr-karte}

\chapter{Trotzburg}
\section*{Auswertungsbogen}
\label{trotzburg-auswertung}
\enlargethispage{1,5cm}
\renewcommand{\arraystretch}{1.27}

\begin{tabular}{|l|l|l|}
  \hline
  \fett{Beobachtete Spielerin:} & \fett{Verteidigt sich:} & \fett{Gibt eigene Fehler zu:} \\
                                   \cline{2-3}
                                 & sachlich                & sachlich \\
                                   \cline{2-3}
                                 & unsachlich              & unsachlich \\

  \hline \hline

  \fett{Andere Spielerinnen:}    & \fett{Greift andere an:} & \fett{Hilft anderen:} \\
  \hline
  \fett{1}                       & sachlich                & sachlich \\
                                   \cline{2-3}
                                 & unsachlich              & unsachlich \\
  \hline   
  \fett{2}                       & sachlich                & sachlich \\
                                   \cline{2-3}
                                 & unsachlich              & unsachlich \\
  \hline   
  \fett{3}                       & sachlich                & sachlich \\
                                   \cline{2-3}
                                 & unsachlich              & unsachlich \\
  \hline   
  \fett{4}                       & sachlich                & sachlich \\
                                   \cline{2-3}
                                 & unsachlich              & unsachlich \\
  \hline \hline

  \fett{Schließt Koalitionen mit:}& \fett{Hetzt gegeneinander auf:} & \fett{Vermittelt zwischen:} \\
  \hline
  1 & 1 gegen 2 & 1 und 2 \\
  \hline
  2 & 1 gegen 3 & 1 und 3 \\
  \hline
  3 & 1 gegen 4 & 1 und 4 \\
  \hline
  4 & 2 gegen 3 & 2 und 3 \\
  \hline
    & 2 gegen 4 & 2 und 4 \\
  \cline{2-3}
    & 3 gegen 4 & 3 und 4 \\
  \hline
\end{tabular}
\renewcommand{\arraystretch}{1.0}

\subsection*{Anleitung:}
In der linken Spalte werden, von oben nach unten, Name und Rolle der beobachteten Spielerinnen eingetragen, darunter Namen und Rollen ihrer Mitspielerinnen (1, 2, 3, 4).

Jedes Mal, wenn die beobachtete Spielerin etwas sagt, wird ein Strich in das passende Kästchen gesetzt. Die Zeilen 1--4 betreffen die Spielerinnen, die die beobachtete Spielerin angreift oder denen sie hilft. Also kein Strich, wenn "`Spielerin Nr.~3"' etwas sagt, sondern wenn die beobachtete Spielerin zu Spielerin 3 etwas sagt. Entsprechend werden Striche im unteren Bereich gemacht, wenn die beobachtete Spielerin zwischen zwei anderen vermittelt, oder sie gegeneinander aufhetzt.

Trotzdem wird bei der Protokollierung eines Spiels bei bestimmten Äußerungen immer noch eine Grauzone übrig bleiben, etwa wenn es darum geht zu entscheiden, ob eine Aussage nun sachlich oder unsachlich war. In wichtigen Fällen sollten die Beobachterinnen zusätzliche Anmerkungen an den Rand schreiben.
\newpage

\label{trotzburg-rollen}
\section*{Rollenbeschreibung: Krankenpfleger}

\emph{Die kleine arme mittelalterliche Stadt Trotzburg ist zerstritten mit der großen reichen Nachbarstadt Hochberg.}

Zum Krankenpfleger von Trotzburg kommt eines Tages der Schmied und sagt: "`Draußen vor der Stadt liegt ein Kaufmann aus Hochberg. Er ist verwundet. Er hat mich angefallen, da habe ich mich gewehrt und ihn zusammengeschlagen. Wir können ihn nicht im Schnee liegen lassen. Komm und hilf mir, ihn hereinzutragen!"'

Der Krankenpfleger hat wenig Lust, einem Hochberger zu helfen. Deswegen sagt er: "`Du hast mir nicht zu sagen, was ich zu tun habe. Wenn's der Bürgermeister sagt, dann gehe ich hinaus, sonst nicht!"' Eigentlich ärgert er sich ja oft über den Bürgermeister, dass dieser ihn herumkommandiert. Aber jetzt ist es ihm ganz recht, dass er auf die Autorität des Bürgermeisters verweisen kann.

Der Schmied sagt, er sei sowohl schon beim Bürgermeister gewesen und auch beim Arzt. Beide wollen nicht wirklich etwas tun. Der Krankenpfleger bleibt trotzdem hart in seiner Position.

Der Schmied läuft weg. Nach einer Weile kommt er wieder und berichtet, der Bürgermeister habe es jetzt befohlen. Da geht der Krankenpfleger mit ihm hinaus, und sie holen den Verwundeten herein.

Der Arzt verbindet seine Wunden. Trotzdem stirbt der Kaufmann noch in der Nacht. Der Arzt sagt: "`Der war nicht mehr zu retten. Die Kälte hat ihn fertig gemacht. Wenn der Wächter gleich gesehen hätte, was los war, und uns Bescheid gegeben hätte, hätte ich ihn vielleicht durchgebracht."'

\emph{Kurze Zeit darauf kommen die Soldaten von Hochberg vor die Stadt. Sie sind in der Übermacht. Sie lassen den Trotzburgern eine Botschaft überbringen: "`Liefert uns bis in einer Stunde den Schuldigen aus, der den Kaufmann getötet hat! Sonst brennen wir die ganze Stadt nieder!"'}
\newpage

\section*{Rollenbeschreibung: Schmied}

\emph{Die kleine arme mittelalterliche Stadt Trotzburg ist zerstritten mit der großen reichen Nachbarstadt Hochberg.}

Der Schmied von Trotzburg sieht eines Tages vor der Stadt einen Kaufmann aus Hochberg vorbeikommen. Er denkt sich: "`Dem nehme ich sein Geld ab!"' Er überfällt ihn, schlägt ihn zusammen und raubt ihm Geld. Beim Anblick des Verwundeten bekommt der Schmied dann aber doch Angst und rennt in die Stadt, um Hilfe zu holen.

Zuerst geht er allerdings zum Wächter auf den Turm. Der Wächter hat alles beobachtet. Der Schmied gibt ihm deswegen die Hälfte des geraubten Geldes, damit er nichts verrät. Der Wächter verspricht zu schweigen.

Danach läuft der Schmied zum Bürgermeister und sagt zu ihm: "`Eben hat mich ein Kaufmann aus Hochberg überfallen wollen. Ich habe mich gewehrt und ihn verwundet. Jetzt liegt er draußen im Schnee."' Der Bürgermeister zählt gerade die Stadtkasse nach und sagt nur: "`Das werden wir schon wieder hinbekommen!"', tut aber nichts.

Da läuft der Schmied zum Arzt und sagt: "`Komm mit vor die Stadt hinaus und hilf dem verwundeten Kaufmann!"' Der Arzt sagt: "`Was? Zu einem Hochberger soll ich hinausgehen? Fällt mir gar nicht ein. Wenn ihr ihn hereinschafft, werde ich ihn vielleicht behandeln, sonst nicht."'

Da rennt der Schmied zum Krankenpfleger und bittet ihn: "`Trage doch mit mir den Kaufmann herein! Allein schaffe ich es nicht."' Der Krankenpfleger sagt: "`Du hast mir nicht zu sagen, was ich zu tun habe. Wenn's der Bürgermeister sagt, komme ich mit, sonst nicht."'

Der Schmied läuft wieder zum Bürgermeister, der immer noch beim Geldzählen ist. Dieser sagt schließlich: "`Meinetwegen soll er ihn hereinschaffen."'

Der Schmied läuft zum Krankenpfleger, und beide tragen den Kaufmann in die Stadt. Der Arzt verbindet seine Wunden, aber noch in derselben Nacht stirbt der Kaufmann.

Der Arzt sagt: "`Der war nicht mehr zu retten. Die Kälte hat ihn fertig gemacht. Wenn der Wächter gleich gesehen hätte, was los war, und uns Bescheid gegeben hätte, hätte ich ihn vielleicht durchgebracht."'

\emph{Kurze Zeit darauf kommen die Soldaten von Hochberg vor die Stadt. Sie sind in der Übermacht. Sie lassen den Trotzburgern eine Botschaft überbringen: "`Liefert uns bis in einer Stunde den Schuldigen aus, der den Kaufmann getötet hat! Sonst brennen wir die ganze Stadt nieder!"'}
\newpage

\section*{Rollenbeschreibung: Bürgermeister}

\emph{Die kleine arme mittelalterliche Stadt Trotzburg ist zerstritten mit der großen reichen Nachbarstadt Hochberg.}

Der Bürgermeister kann die Hochberger nicht ausstehen. Eines Tages, als er gerade die Stadtkasse nachzählt, kommt der Schmied angerannt und erzählt: "`Eben hat mich ein Kaufmann aus Hochberg überfallen wollen. Ich hab mich gewehrt und ihn verwundet. Jetzt liegt er draußen im Schnee."' Der Bürgermeister denkt sich: "`Das geschieht dem Hochberger recht!"' Und weil er die Hochberger nicht mag, bleibt er hinter seinem Geld sitzen und sagt nur: "`Das werden wir schon wieder hinbekommen!"'

Der Schmied läuft daraufhin zum Arzt, aber der will auch nichts tun. Er sagt, er werde den Verwundeten vielleicht behandeln, wenn man ihn hereinbringe. Der Schmied bittet daraufhin den Krankenpfleger, zusammen mit ihm den Kaufmann hereinzutragen. Aber der sagt: "`Wenn's der Bürgermeister sagt, komme ich mit, sonst nicht."'

Da geht der Schmied zum Bürgermeister zurück und erzählt ihm alles. Der Bürgermeister sagt: "`Na, meinetwegen soll er ihn hereinschaffen."' Sie schaffen den Kaufmann herein, und der Arzt verbindet seine Wunden, aber noch in der Nacht stirbt der Kaufmann.

Der Arzt sagt zu den anderen: "`Der war nicht mehr zu retten, die Kälte hat ihn fertig gemacht. Wenn der Wächter gesehen hätte, was los war, und uns sofort Bescheid gegeben hätte, hätte ich ihn vielleicht durchgebracht."' Der Wächter sagt, er habe von dem ganzen Vorfall nichts gesehen.

\emph{Kurze Zeit darauf kommen die Soldaten von Hochberg vor die Stadt. Sie sind in der Übermacht. Sie lassen den Trotzburgern eine Botschaft überbringen: "`Liefert uns bis in einer Stunde den Schuldigen aus, der den Kaufmann getötet hat! Sonst brennen wir die ganze Stadt nieder!"'}
\newpage

\section*{Rollenbeschreibung: Arzt}

\emph{Die kleine arme mittelalterliche Stadt Trotzburg ist zerstritten mit der großen reichen Nachbarstadt Hochberg.}

Eines Tages kommt der Schmied zum Arzt und sagt: "`Draußen vor der Stadt liegt ein Kaufmann aus Hochberg verwundet im Schnee. Komm doch raus und hilf ihm! Er hat mich überfallen wollen, ich habe mich gewehrt und ihn verwundet. Eben war ich schon beim Bürgermeister, aber der will nichts unternehmen!"'

Der Arzt denkt sich: "`Geschieht ihm recht, dem Hochberger!"' Er sagt: "`Was?! Ich soll zu einem Hochberger herausgehen in dieser Kälte? Fällt mir gar nicht ein. Bringt ihn rein, dann kann ich ihn vielleicht behandeln."'

Der Schmied läuft zum Krankenpfleger und bittet ihn, den Kaufmann mit in die Stadt zu tragen. Aber der will erst etwas unternehmen, wenn es der Bürgermeister befiehlt. Daraufhin rennt der Schmied zum Bürgermeister. Dieser befiehlt endlich, den Verwundeten hereinzuschaffen. Gemeinsam tragen der Schmied und der Krankenpfleger den Kaufmann zum Arzt.

Der Arzt sieht, dass der Kaufmann sterben wird, weil er solange im Schnee gelegen hat. Er verbindet die Wunden, aber Arznei gibt er dem Kaufmann nicht, weil er sich denkt: "`Wozu soll ich diesem Hochberger auch noch kostenlos meine teure Arznei geben?"'

In der Nacht stirbt der Kaufmann. Der Arzt sagt zu den anderen: "`Der war nicht mehr zu retten, die Kälte hat ihn fertig gemacht. Wenn der Wächter gesehen hätte, was los war, und uns sofort Bescheid gegeben hätte, hätte ich ihn vielleicht durchgebracht."'

\emph{Kurze Zeit darauf kommen die Soldaten von Hochberg vor die Stadt. Sie sind in der Übermacht. Sie lassen den Trotzburgern eine Botschaft überbringen: "`Liefert uns bis in einer Stunde den Schuldigen aus, der den Kaufmann getötet hat! Sonst brennen wir die ganze Stadt nieder!"'}

Kurz vor der Beratung, in der entschieden werden soll, wer den Hochbergern als Schuldiger ausgeliefert werden soll, kommt der Wächter zum Arzt und bezahlt eine längst fällige Rechnung.
\newpage

\section*{Rollenbeschreibung: Wächter}

\emph{Die kleine arme mittelalterliche Stadt Trotzburg ist zerstritten mit der großen reichen Nachbarstadt Hochberg.}

Der Wächter steht auf seinem Turm und beobachtet die Straße, die an der Stadt vorbeiführt. Eines Tages sieht er, wie der Schmied von Trotzburg einen Hochberger Kaufmann überfällt, niederschlägt und ausraubt.

Der Wächter meldet es aber nicht in der Stadt, weil er sich denkt: "`Was geht mich ein Hochberger an?"' Kurz darauf kommt der Schmied zu ihm auf den Turm und gibt ihm Geld, damit er in der Stadt sagt, er habe nichts gesehen. Dem Wächter ist das durchaus recht, und er verspricht, nichts zu sagen.

Der Schmied läuft weiter zum Bürgermeister und stellt die Sache so dar, als wäre er vom Kaufmann angefallen worden und hätte diesen in Notwehr verwundet. Der Bürgermeister tut nichts.

Danach läuft der Schmied weiter zum Arzt, der aber nicht hinausgehen will. Er würde den Hochberger höchstens dann behandeln, wenn jemand den Verwundeten hereinbrächte. Also rennt der Schmied zum Krankenpfleger. Dieser allerdings will nur hinausgehen, wenn es der Bürgermeister befiehlt.

Nachdem der Schmied vom Bürgermeister endlich den Befehl eingeholt hat, den Kaufmann zu holen, geht der Krankenpfleger zusammen mit dem Schmied vor die Stadt.

Aber es ist schon zu spät für den Kaufmann, der in der Nacht stirbt. Der Arzt sagt: "`Der war nicht mehr zu retten, die Kälte hat ihn fertig gemacht. Wenn der Wächter gesehen hätte, dass da einer verwundet im Schnee liegt, und uns Bescheid gegeben hätte, hätte ich ihn vielleicht durchgebracht."'

\emph{Kurze Zeit darauf kommen die Soldaten von Hochberg vor die Stadt. Sie sind in der Übermacht. Sie lassen den Trotzburgern eine Botschaft überbringen: "`Liefert uns bis in einer Stunde den Schuldigen aus, der den Kaufmann getötet hat! Sonst brennen wir die ganze Stadt nieder!"'}

Kurz vor der Beratung, wer den Hochbergern als Schuldiger ausgeliefert werden soll, kommt der Wächter zum Arzt und bezahlt eine längst fällige Rechnung.

\chapter{Scotland Yard}
\label{scotland-yard-regeln}
\section*{Idee des Spiels}
Unser \emph{Scotland Yard}-Spiel leitet sich von dem bekannten Brettspiel ab. Es werden
drei Personen, \emph{Mr.~X}, \emph{Mr.~Y} und \emph{Mr.~Z} in der Bonner Innenstadt gesucht. Da wir kein Spielbrett haben, werden die Standpunkte der drei \emph{Mr.~X/Y/Z} regelmäßig per SMS an die suchenden Detektivgruppen weitergegeben.
\section*{Ziel des Spiels}
Das Ziel für jede Gruppe ist es, jeden \emph{Mr.~X/Y/Z} einmal zu fangen. Das Spiel wird
beendet, wenn eine Gruppe alle drei Personen einmal entdeckt hat.
\section*{Spielregeln}
\subsection*{Start des Spiels}
Zum Spielbeginn wird den Detektivgruppen, die alle in der PH starten, der Standpunkt der
drei zu suchenden Personen mitgeteilt.
\subsection*{Bewegung der Teilnehmer}
\subsubsection*{Mr.~X/Y/Z}
Die drei zu suchenden Personen dürfen sich nur mit den öffentlichen
Nahverkehrsmitteln oder zu Fuß bewegen. Sie dürfen nur direkte Verbindungen
zwischen Punkten auf der Karte nehmen und sich nicht zu Fuß in der Altstadt
verkrümeln.
\subsubsection*{Detektive}
Die Detektive dürfen sich bewegen, wie sie wollen.
\subsection*{Fangen eines Mr.~X/Y/Z}
Wenn die Detektive einen \emph{Mr.~X/Y/Z} finden, sprechen sie ihn an. Er wird ihnen
dann eine Unterschrift mit Uhrzeit und Ort der Gefangennahme auf den Spielplan
geben. Danach trennt sich die Detektivgruppe wieder von der gefundenen Person und sucht weiter nach den
anderen. \emph{Mr.~X/Y/Z} spielt weiter mit, damit die andern Gruppen auch eine Chance
haben ihn zu finden. Hat eine Gruppe von allen drei \emph{Mr.~X/Y/Z} eine Unterschrift,
gibt der zuletzt Gefangene eine Nachricht an die Zentrale, so dass alle vom Ende
des Spiels informiert werden.
Es soll \emph{keine} wilden Verfolgungen über verschiedene Plätze in Bonn geben. Die \emph{Mr.~X/Y/Z} werden, sobald sie die Detektive sehen, nicht panisch lossprinten, können aber noch versuchen, so gerade mit einem Bus oder einer Bahn zu entkommen.
\subsection*{Positionsangaben}
Positionsangaben von \emph{Mr.~X/Y/Z} werden alle 15~Minuten per SMS an die
Detektivgruppen gegeben. Diese Angabe enthält den Namen der Haltestelle, an der
sich \emph{Mr.~X/Y/Z} zurzeit befindet oder zuletzt befunden hat. 

\emph{Mr.~X/Y/Z} bekommen keine Angaben über die suchenden Gruppen.
\subsection*{Kommunikation zwischen den Gruppen}
Die Detektivgruppen dürfen während dem Spiel miteinander telefonieren, um sich
bei der Suche abzusprechen.
\subsection*{Spielfeld}
Alle \emph{Mr.~X/Y/Z} müssen innerhalb der vorgegebenen Karte bleiben. Sie dürfen sich
nicht in irgendwelchen Gebäuden (Bars, Kneipen, etc.) aufhalten, sondern müssen
sich in der Öffentlichkeit bewegen. 
\section*{Hinweise}
Der Spielplan ist nicht ganz vollständig. Einige Haltestellen,
an denen man nicht umsteigen kann, wurden zur besseren Übersichtlichkeit
weggelassen. Auch sieht es auf diesem Plan so aus, als könne man einige Strecken
einfach durchfahren. Dies muss nicht unbedingt so sein. Um genau zu sehen, wo
welche Linie fährt, hängen an den meisten Haltestellen Netzpläne von Bonn, auf
denen die Linien farbig eingezeichnet sind.

Bei einigen Haltestellen muss man darauf achten, dass die Busse nur in einer
Richtung dort vorbeifahren.

Falls ihr euch untereinander absprechen wollt, solltet ihr eure Handynummern austauschen, bevor ihr loszieht.

\section*{Scotland-Yard-Karte für das Bonner Verkehrssystem}
\scalebox{0.9}[0.9]{\includegraphics*{karte}}

\chapter{Feedback-Fragebogen zur Orientierungseinheit}
\label{fragebogen}
\medskip
LiebeR ErstsemesterIn,
\medskip
\\
deine Meinung zu unserer Orientierungseinheit ist uns wichtig, denn sie hilft uns, die nächste OE noch besser zu gestalten.

Bitte fülle diesen Fragebogen so vollständig wie möglich aus. Sei ruhig schonungslos ehrlich~-- der Fragebogen ist anonym. Wenn dir etwas besonders gut gefallen hat, darfst du uns natürlich ebenso schonungslos loben.

Vielen Dank fürs Ausfüllen!

\smallskip
\paragraph*{Dieser Tag ist heute:} $\bigcirc$ Mittwoch \hspace{1cm} $\bigcirc$ Donnerstag \hspace{1cm} $\bigcirc$ Freitag

\paragraph*{Deine TutorInnen:} \rule{10cm}{0,3pt}

\bigskip

\begin{enumerate}
\item Was hat dir heute besonders gut gefallen?
\vspace{1,0cm}
\item Was hat dir heute nicht so gut gefallen?
\vspace{1,0cm}
\item Wie fandest du die Atmosphäre in deiner Gruppe heute?
\vspace{1,0cm}
\item Was hast du heute verstanden und behalten?
\vspace{1,0cm}
\item Was hast du verstanden, aber nicht in allen Einzelheiten behalten?
\vspace{1,0cm}
\item Was hast du heute nicht verstanden?
\vspace{1,0cm}
\item Was möchtest du uns sonst noch mitteilen? (Auf der Rückseite ist auch noch Platz.)
\end{enumerate}

\chapter{Über den Autor}
Oliver Klee wohnt und arbeitet in Bonn. Er hat seit 1999 über 50 Seminare und Workshops mit einem sehr breiten Themenspektrum gegeben, zum Beispiel:
\begin{itemize}
  \item Rhetorik
  \item Kommunikation
  \item Moderation, Gruppenleitung, Train-the-Trainer-Seminare
  \item Teamtrainings
  \item Zeitmanagement
  \item Präsentationstechnik
  \item Webdesign
  \item Layout
\end{itemize}
Der Autor geht auch selbst gerne auf (gute) Seminare und ist dabei für das gelegentliche Auflockerungsspiel fast immer zu haben.
\paragraph{Website:} \url{http://www.oliverklee.de/}
\paragraph{E-Mail:} \texttt{seminare@oliverklee.de} (zum Thema Seminare) sowie \texttt{oliver@spielereader.org} (für Angelegenheiten in Sachen Spielereader)


\chapter{Änderungshistorie}
\section*{??.??.????}
\begin{itemize}
  \item Spiele hinzugefügt:
    \begin{itemize}
      \item Kontakt
      \item Zeitungsninja
      \item Tippspiel/Zweikampf
    \end{itemize}
  \item Spiele entfernt:
    \begin{itemize}
      \item Scotland Yard
    \end{itemize}
  \item Lösungen für Ratespiele und Aufgaben sind jetzt kopfüber gedruckt.
  \item Literaturliste ergänzt
  \item die Methode \emph{Vier Felder} aus der \emph{Auswertungsgalerie} herausgelöst
\end{itemize}


\section*{02.10.2006}
\begin{itemize}
  \item Absatz-Abstände verringert, damit der Reader ein paar Seiten kürzer wird.
  \item zwei weitere Buchkritiken hinzugefügt
  \item Spiele hinzugefügt:
  \begin{itemize}
    \item Gegenstand aussuchen
    \item die verrückte Professorin
    \item Assoziationsspiel
    \item Gummibärchenanalyse
  \end{itemize}
\end{itemize}



\section*{03.10.2005}
\begin{itemize}
  \item Viele kleinere Fehlerkorrekturen und kosmetische Änderungen.
  \item Spiele hinzugefügt:
  \begin{itemize}
    \item Ampel-Feed-back
    \item Sektreflexion
    \item Writer's Workshop
    \item Insel ohne Wiederkehr
    \item Zettel auf dem Rücken
  \end{itemize}
  \item Die (abweichenden) Bezeichnungen aus dem vfh-Spielereader in den Index aufgenommen.
\end{itemize}

\section*{13.09.2004}
\begin{itemize}
  \item Der Spielereader steht jetzt unter einer Creative-Commons-Lizenz.
  \item Lösungen für Ratespiele und Aufgaben sind jetzt in Versalien gedruckt, damit man sie beim Überfliegen nicht so leicht aus Versehen liest.
  \item Neue Version der Scotland-Yard-Karte.
  \item Neue URL und neue E-Mail-Adresse für den Reader.
  \item Viele kleinere Fehlerkorrekturen und kosmetische Änderungen.
  \item Spiele hinzugefügt:
  \begin{itemize}
    \item Schuhsalat
    \item Daumenwrestling
    \item Wäscheklammern
    \item Rapunzel
    \item Kreisflucht
  \end{itemize}
  \item Spiele entfernt:
  \begin{itemize}
    \item Mein Bereich
    \item Lachspiel
  \end{itemize}
\end{itemize}


\printindex

\end{document}

\section{}
\index{}
\paragraph{Alias:}
\paragraph{Art:}
\paragraph{Ziel:}
\paragraph{Dauer:}
\paragraph{Wir brauchen dazu:}
\paragraph{So geht es:}
\paragraph{Besondere Hinweise:}
\paragraph{Wann einsetzen:}
\paragraph{Varianten:}

