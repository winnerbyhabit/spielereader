\chapter{Auflockerungsspiele}
\index{Auflockerungsspiele}

\section{Au ja!}
\index{Au ja!}
\paragraph{Art:} lustiges, unterschiedlich aktives Rumalberspiel
\paragraph{Ziel:} jede darf etwas sagen, was dann alle machen
\paragraph{Dauer:} 5--10 Minuten
\paragraph{Wir brauchen dazu:} ---
\paragraph{So geht es:} Alle stehen im Kreis. Wer anfängt, sagt "`Wir machen jetzt alle \ldots"'. Alle rufen "`Au ja! Au ja!"' und machen das Gesagte. So geht es reihum.

Beispiele: "`Wir machen jetzt alle eine Grimasse."', "`Wir klopfen uns alle auf den Bauch (den eigenen)."' oder "`Wir küssen uns alle auf die linke Schulter."'

Das Spiel ist beendet, wenn eine Tutorin sagt: "`Wir arbeiten jetzt alle weiter."'
\paragraph{Varianten:} Margerite (s.~u.)
\paragraph{Wann einsetzen:} Wenn die Gruppe lange sehr ernsthaft gearbeitet hat. Oder wenn sie ganz viel Albernheit herauslassen muss.

\section{Margerite}
\index{Margerite}
\paragraph{Art:} lustiges, unterschiedlich aktives Rumalberspiel
\paragraph{Ziel:} jede darf etwas vormachen, was dann im Kreis herumgeht
\paragraph{Dauer:} 10--15 Minuten
\paragraph{Wir brauchen dazu:} ---
\paragraph{So geht es:} Im Prinzip ahmt die Gruppe eine Margeriten-Blüte nach, deren Blütenblätter sich eins nach dem anderen öffnen oder schließen.

Alle stehen im Kreis. Wer anfängt, macht eine Bewegung und ein Geräusch. Die rechte Nachbarin macht es nach. Danach macht es deren rechte Nachbarin nach und so weiter, bis alle die Bewegung und das Geräusch machen.

Wenn Bewegung und Geräusch wieder bei der angekommen sind, die es vorgemacht hat, macht deren rechte Nachbarin etwas Neues vor.
\paragraph{Wann einsetzen:} Wenn die Gruppe lange sehr ernsthaft gearbeitet hat. Oder wenn sie ganz viel Albernheit herauslassen muss.

\section{Die Milch kocht über!}
\index{Milch kocht über!}
\index{Kochende Milch|\see{Milch kocht über!}}
\index{Brüllspiel|\see{Milch kocht über!}}
\paragraph{Alias:} Kochende Milch
\paragraph{Art:} heftiges Brüllspiel
\paragraph{Ziel:} zwei Gruppen brüllen sich immer lauter an
\paragraph{Dauer:} 5 Minuten
\paragraph{Wir brauchen dazu:} ---
\paragraph{So geht es:} Die Leute bilden zwei Gruppen, die sich gegenüberstehen. Die eine Gruppe sagt: "`Die Milch kocht über!"', worauf die andere Gruppe erwidert: "`Dann hol du sie doch vom Feuer!"', worauf die erste Gruppe wieder sagt: "`Die Milch kocht über!"' und so weiter. Da die jeweils andere Gruppe das eigene Anliegen offensichtlich nicht versteht, werden beide Gruppen nach und nach immer lauter, bis sich beide Gruppen schließlich anbrüllen. Das Spiel ist zu Ende, wenn es die Tutorin abbricht (oder wenn alle Teilnehmerinnen heiser sind).
\paragraph{Varianten:}
  \begin{description}
    \item[Im Restaurant beim Kampf um einen Platz:] "`Sie stehen jetzt sofort auf!"'~-- "`Nein, ich werde jetzt nicht aufstehen!"'
    \item[Kinder am Gartenzaun:] "`Nein!"'~-- "`Doch!"'
  \end{description}
\paragraph{Besondere Hinweise:} Es ist nicht weiter schlimm, wenn euch die Leute aus den Nebenräumen später etwas komisch angucken.
\paragraph{Wann einsetzen:} Wenn die Gruppe gefrustet ist und Dampf ablassen muss. Oder als Erfahrung, wie laut die eigene Stimme sein kann (z.\,B.~bei einem Rhetorikseminar).

\section{Jammern}
\index{Jammern}
\index{Stöhnen|see{Jammern}}
\paragraph{Alias:} Stöhnen
\paragraph{Art:} alle stöhnen oder jammern gleichzeitig
\paragraph{Ziel:} kontrolliert die Genervtheit herauslassen
\paragraph{Dauer:} 2-5 Minuten
\paragraph{Wir brauchen dazu:} ---
\paragraph{So geht es:} Alle stöhnen oder jammern gleichzeitig.
\paragraph{Besondere Hinweise:} Tut enorm gut und ist gar nicht so albern, wie es sich zuerst liest.
\paragraph{Wann einsetzen:} Wenn die Gruppe genervt oder gefrustet ist.

\section{Intelligenztest}
\index{Intelligenztest}
\paragraph{Art:} Gemeines Spiel der hinterhältigen Art.
\paragraph{Ziel:} Ein neues Bewusstsein schaffen für Aufmerksamkeit, vor allem für Klausuren.
\paragraph{Dauer:} 3 Minuten.
\paragraph{Wir brauchen dazu:} pro Person eine Kopie des Intelligenztests (Seite \pageref{iq})
\paragraph{So geht es:} Alle bekommen je eine Kopie des Testes und sollen den Bogen ohne zu sprechen innerhalb von drei Minuten ausfüllen. Wer fertig ist, dreht das Blatt um. Wer den Test schon einmal gemacht hat und im Vorteil wäre, wird gebeten, das Blatt sofort umzudrehen.
\paragraph{Besondere Hinweise:} Dieser "`Test"' dient nur dem Aha-Erlebnis und kann bei einigen Leuten zu enormen Heiterkeitsausbrüchen führen.
\paragraph{Wann einsetzen:} Am Beginn einer Phase, in der es um besondere Aufmerksamkeit geht. Auch gut als Pausenfüller, wenn eine Tutorin noch etwas vorbereiten muss.

\section{Pärchenfangen}
\index{Pärchenfangen}
\index{Anarchofangen|see{Pärchenfangen}}
\index{Ransetzen|see{Pärchenfangen}}
\index{Taxifangen|see{Pärchenfangen}}
\paragraph{Alias:} Anarchofangen, Ransetzen, Taxifangen
\paragraph{Art:} sehr schnelles Fangspiel mit Gehirneinsatz
\paragraph{Ziel:} Gejagte und Jäger, die schnell zu Gejagten werden
\paragraph{Dauer:} 10--15 Minuten
\paragraph{Wir brauchen dazu:} genügend Platz zum Laufen, am besten draußen
\paragraph{So geht es:} Alle teilen sich in Zweiergruppen\footnote{und eine Dreiergruppe, wenn die Anzahl der Spielerinnen ungerade ist} auf, die sich einhaken (nur zu zweit jeweils, nicht alle Pärchen zusammen) und auf der Spielfläche verteilen. Ein Pärchen teilt sich in Jägerin und Gejagte. Wenn die Jägerin die Gejagte erwischt, vertauschen sich die Rollen (Gejagte zu Jägerin und umgekehrt).

Die Gejagte kann sich retten, indem sie sich bei einem Pärchen einhakt. Dabei muss sie sich mit gleicher Blickrichtung wie das Pärchen einhaken (also nicht von vorne einhaken). Sobald die Spielerin, die jetzt in der Mitte ist, merkt, dass die Gejagte fest angedockt hat, kann sie die Spielerin am anderen Ende des Pärchens loslassen. Diese wird dann zur neuen Jägerin, und die alte Jägerin wird zur neuen Gejagten. So schnell kann's kommen!
\paragraph{Besondere Hinweise:} Es ist lustig: Immer gibt es ein, zwei Nasen, die die Regeln partout nicht raffen und als Gejagte auf die Jägerin zulaufen anstatt wegzulaufen.
\paragraph{Varianten:} Die Pärchen stehen nicht, sondern sitzen. Anstatt sich bei einem Pärchen einzuhaken setzt sich die Teilnehmerin auf der Flucht dort heran ("`Ransetzen"').
\paragraph{Wann einsetzen:} Wenn das Hirn qualmt oder auf Stand-by steht und Bewegung an frischer Luft angebracht ist.

\section{Knäuel weitergeben}
\index{Knäuel weitergeben}
\paragraph{Art:} sehr schnelles Actionspiel im Sitzen
\paragraph{Ziel:} zwei Knäuel schnell weitergeben, um nicht beide auf einmal zu bekommen.
\paragraph{Dauer:} 10--20 Minuten
\paragraph{Wir brauchen dazu:} 2 (getrennt) zu Knäuels geknotete Schals, Tücher, Mützen o.~Ä.
\paragraph{So geht es:} Die Gruppe sitzt eng im Kreis (auf Stühlen oder im Schneidersitz), so dass sich die Knie berühren. Die Knäuel starten an zwei verschiedenen Stellen im Kreis. Die Spielerinnen geben die Knäuel an eine Nachbarin weiter oder legen sie ihr in den Schoss. Die Knäuel dürfen dabei durchaus öfters die Richtung wechseln. Wer beide Knäuel auf einmal hat, scheidet aus, und der Kreis wird kleiner.

Das Spiel ist zu Ende, wenn nur noch zwei Spielerinnen übrig sind.
\paragraph{Wann einsetzen:} Bei allgemeiner Lethargie und wenn sich die Gruppe schon ein bisschen kennt.

\section{Zauberwald}
\index{Zauberwald}
\index{Zwerg, Riese, Zauberer|see{Zauberwald}}
\paragraph{Alias:} Zwerg, Riese, Zauberer
\paragraph{Art:} Fangspiel nach ähnlichem Prinzip wie das Knobelspiel \emph{Schnick-Schnack-Schnuck} \emph{(Stein, Schere, Papier)}
\paragraph{Ziel:} schnell entscheiden, schnell reagieren~-- und schnell weglaufen
\paragraph{Dauer:} 10--20 Minuten
\paragraph{Wir brauchen dazu:} genügend Platz zum Laufen, am besten draußen
\paragraph{So geht es:} Die Gruppe teilt sich in zwei ungefähr gleichstarke Fraktionen, die sich im Abstand von gut zwei Metern gegenüber aufstellen. Jeder Gruppe überlegt sich in einer Blitzberatung, welche der drei folgenden Figuren sie einheitlich darstellen möchte:
\begin{description}
  \item[Zwerg:] Mit beiden Armen wird eine stilisierte Zwergenmütze dargestellt. Der Zwerg besiegt den Zauberer, wird aber selbst vom Riesen besiegt.
  \item[Riese:] Beide Arme über dem Kopf hoch strecken, um die enorme Größe des Riesen darzustellen. Der Riese besiegt den Zwergen, kann selbst aber vom Zauberer verzaubert werden.
  \item[Zauberer:] Den rechten Arm als Zauberstab nach vorne strecken ("`\emph{\ldots bsssst!}"'). Der Zauberer verzaubert den Riesen, kann selbst aber vom Zwerg überwältigt werden.
\end{description}
Nach der Beratung zählen alle bis drei. Bei drei stellt jede Gruppe die Figur dar, die sie sich ausgesucht hat, und läuft weg oder versucht entsprechend die anderen zu fangen. Wer gefangen wird, wechselt zu anderen Gruppe über. Wer die "`sichere Zone"' erreicht (3--5~Meter hinter jeder Gruppe), ist in Sicherheit. Wenn beide Gruppen das Gleiche darstellen, beraten sich beide Gruppen neu.

Und weiter geht es mit der nächsten Runde, bis eine Gruppe völlig verschwunden (gefangen) ist oder niemand mehr kann oder will.
\paragraph{Besondere Hinweise:} Mit großen Gruppen macht das Spiel besonders viel Spaß (\emph{the more, the merrier})~-- ich selbst habe es einmal mit insgesamt 50~Leuten gespielt. Vielleicht könnt ihr euch mit ein oder zwei anderen OE-Gruppen dafür zusammentun?
\paragraph{Warnung:} Da auch bei bei diesem Spiel immer Leute dabei sind, die ab und an die Regeln durcheinander bringen, seid bitte vorsichtig. Es hat schon böse Platzwunden gegeben, weil zwei Spielerinnen frontal mit den Köpfen zusammengestoßen sind.
\paragraph{Wann einsetzen:} Bei allgemeiner Lethargie und zum schnellen Wachwerden.

\section{Gordischer Knoten}
\index{Gordischer Knoten}
\index{Knoten, Gordischer}
\paragraph{Art:} Problemlösung. Bewegungsspiel, nicht ganz sanft.
\paragraph{Ziel:} Auflösen des Gordischen Knotens, aber nicht mit Gewalt.
\paragraph{Dauer:} 10 Minuten.
\paragraph{Wir brauchen dazu:} $\geq 8$ Teilis, genug Platz für einen Stehkreis plus Sicherheitszone
\paragraph{So geht es:} Alle stehen in einem engen Kreis und strecken die Arme nach vorne in die Mitte des Kreises. Auf Kommando schließen alle die Augen, gehen auf die Mitte zu und fassen mit jeder Hand (genau) eine andere Hand~-- möglichst nicht die des Nachbarn.

Wenn alle Hände angedockt sind, öffnen alle wieder die Augen und versuchen, den entstandenen Knoten ohne Loslassen zu entwirren.
\paragraph{Besondere Hinweise:} Die Gruppe sollte schon ein Weilchen zusammen gearbeitet haben, damit die gröbsten Berührungsängste abgebaut sind.
\paragraph{Wann einsetzen:} Zum Auflockern. Besonders nützlich, wenn sich die Gruppe während der Arbeit geistig "`verknotet"' hat oder bei einem Problem nicht mehr weiterkommt.

\section{Maschine-Spiel}
\index{Maschine-Spiel}
\paragraph{Art:} kreatives, sehr lustiges Darstellungsspiel
\paragraph{Ziel:} Die Gruppe bildet zusammen eine große Maschine.
\paragraph{Dauer:} 5--10 Minuten
\paragraph{Wir brauchen dazu:} ---
\paragraph{So geht es:} Die Tutorin fängt an und sagt "`Ich bin eine Maschine. Ich mache eine Bewegung und ein Geräusch."' Dazu macht sie eine wiederkehrende Bewegung und ein Geräusch. Nach und nach wird die Maschine durch weitere Mitspielerinnen erweitert, die sich durch Kontakt mit der schon bestehenden Maschine verbinden und selber eine Bewegung ausführen und ein Geräusch machen.

Am Schluss bilden alle gemeinsam eine große, abgefahrene sinnfreie Maschine.
\paragraph{Wann einsetzen:} Bei Lust auf etwas ganz Anderes, und wenn die Gruppe die gröbsten Berührungsängste abgebaut hat.

\section{Wanderndes Klatschen}
\index{Wanderndes Klatschen}
\index{Klatschen, wanderndes}
\paragraph{Art:} ruhiges Aufmerksamkeitsspiel ohne Verliererinnen
\paragraph{Ziel:} gemeinsamen Rhythmus finden, Gruppengefühl stärken
\paragraph{Dauer:} 5--10 Minuten
\paragraph{Wir brauchen dazu:} ---
\paragraph{So geht es:} Die Spielerinnen stellen sich im Kreis auf. Jemand wendet sich einer Nachbarin zu und klatscht in die Hände. Die Nachbarin hat sich inzwischen ebenfalls der ersten zugewandt und klatscht zeitgleich in die Hände. Danach gibt sie das Klatschen in die gleiche Richtung weiter~-- oder gibt es wieder zurück. Direkt nach dem Zurückgeben darf nicht noch einmal zurückgegeben werden. Das Spiel läuft gut und macht richtig Spaß, wenn ein gemeinsamer Rhythmus entstanden ist. Nach und nach kann dann das Tempo erhöht werden.
\paragraph{Besondere Hinweise:} Dieses Spiel ist \emph{miteinander}, nicht \emph{gegeneinander}~-- es geht dabei nicht darum, die anderen auszutricksen! Ziel ist der gemeinsame Rhythmus und das Gruppenerlebnis dabei.
\paragraph{Wann einsetzen:} Zur Entlastung der linken Gehirnhälfte und um die Gruppe zusammenzubringen.

\section{Smaug}
\index{Smaug}
\index{Hobbit, Der kleine|see{Smaug}}
\index{Tolkien|see{Smaug}}
\paragraph{Art:} Actionspiel mit Fantasy-Hintergrund
\paragraph{Ziel:}
\begin{quote}
  "`Meine Rüstung ist ein zehnfacher Schild, meine Zähne sind Schwerter, meine Klauen Speere, das Aufschlagen meines Schwanzes ist ein Donnerkeil, meine Schwingen sind Wirbelstürme und mein Atem bringt den Tod!"', prahlt der Drache Smaug vor Bilbo Beutlin.
\end{quote}

In J.~R.~R.~Tolkiens Fantasy-Erzählung \emph{Der kleine Hobbit} zieht Bilbo aus, um Smaugs unermesslichen Gold- und Juwelenschatz zu rauben. In unserer Version der Geschichte ist zwar weder der Einsatz so hoch noch der Drache so schrecklich, aber auch hier geht es darum, den Kontakt mit dem tödlichen Drachen zu meiden und den Schatz zu rauben.
\paragraph{Dauer:} 5--15 Minuten
\paragraph{Wir brauchen dazu:} Taschentuch, Geschirrtuch, zusammengeknoteten Schal o.\,Ä.
\paragraph{So geht es:}
Eine Spielerin verwandelt sich in Smaug, der über seine Juwelen wacht. Ein am Boden ausgebreitetes Taschentuch ist zwar kein so kostbarer Schatz, aber doch sehr viel praktischer. Die anderen Spieler bilden um Smaug einen Kreis und versuchen, den Schatz zu stehlen ohne erwischt zu werden.

Smaug der Mächtige kann sich von seinem Schatz so weit entfernen, wie er es wagt. Wenn er eine Spielerin berührt, erstarrt diese auf der Stelle und bleibt so bis zum Ende des Spieles. Aber keine Sorge: Drachen, die länger als dreißig Sekunden regieren, sind äußerst selten.

Ein beliebter Trick der SchatzräuberInnen ist es, sich von hinten anzuschleichen und die Juwelen zwischen Smaugs Beinen hindurch an sich zu reißen. Auch wenn man nur so tut, als ob man schon erstarrt wäre, kann man manchmal den Drachen überlisten. Und schließlich bleibt euch noch die Möglichkeit eines Massenangriffs, bei dem es zwar die meisten erwischt, aber doch jemand den Schatz zu fassen kriegt. Die erfolgreiche Schatzräuberin ist der nächste Drache. Sollte es Smaug gelingen, alle zu versteinern, bevor es jemand schafft, den Schatz zu ergattern, dann darf er alle für die nächsten fünfhundert Jahre als Salzsäulen stehen lassen.
\paragraph{Wann einsetzen:} Zur Auflockerung

\section{Roboter parken}
\index{Roboter parken}
\index{Roboterspiel}
\paragraph{Art:} Koordinationsspiel mit viel Bewegung
\paragraph{Ziel:} zwei Roboter in Parkposition steuern
\paragraph{Dauer:} 10--15 Minuten
\paragraph{Wir brauchen dazu:} etwas Platz ohne böse Hindernisse, geht aber auch drinnen
\paragraph{So geht es:} Die Teilnehmerinnen finden sich in Dreierteams zusammen. Zwei der Teilnehmerinnen sind die Roboter und stellen sich Rücken an Rücken. Die dritte ist die "`Roboterführerin"'.

Die Roboter bewegen sich im An-Zustand mit kleinen, stetigen Schritten \emph{(stampf stampf)} immer geradeaus. Läuft ein Roboter gegen ein Hindernis oder einen anderen Roboter, dann läuft er auf der Stelle und piept dabei \emph{(miep miep)}.

Mit drei Kommandos lassen sich die Roboter steuern:
\begin{description}
  \item[An-Knopf:] Durch (leichtes!) Tippen auf den Kopf lassen sich die Roboter anschalten, woraufhin sie loslaufen. Es können immer nur beide Roboter gleichzeitig angeschaltet werden.
  \item[Linksdrehung:] Durch Tippen auf die linke Schulter eines Roboters dreht sich dieser Roboter um 90 Grad nach links.
  \item[Rechtsdrehung:] Analog zur Linksdrehung.
\end{description}
Ziel ist, beide Roboter so zu steuern, dass sie schließlich in "`Parkposition"' (Gesicht an Gesicht gegenüber) stehen und sich dann automatisch abschalten. Stehen alle Roboter in Parkposition, ist das Spiel beendet.
\paragraph{Wann einsetzen:} Zur allgemeinen Auflockerung und Erheiterung, zum Aufwärmen nach einer Pause.
\paragraph{Varianten:}
\begin{itemize}
  \item Die Roboter können auch kontinuierlich beschleunigen~-- langsam starten und dann immer schneller werden.
  \item Die Roboterführerin darf die Roboter nur abwechselnd bedienen.
  \item Die Roboter dürfen nicht parallel oder orthogonal zu einer Wand laufen.\footnote{Das macht die Strategie unmöglich, einfach beide Roboter gegen dieselbe Wand zu steuern und dann beide so zu drehen, dass sie an der Wand entlang aufeinander zugehen.}
\end{itemize}

\section{Was machst du denn?}
\index{Was machst du denn?}
\paragraph{Art:} Koordinationsspiel mit nicht zu viel Bewegung
\paragraph{Ziel:} etwas anderes sagen, als man tut
\paragraph{Dauer:} 5--10 Minuten
\paragraph{Wir brauchen dazu:} ---
\paragraph{So geht es:} Alle stellen sich im Kreis auf. Ein Beispielablauf: A macht eine Bewegung, z.\,B.~sie kratzt sich am Kopf. Die Nachbarin B fragt: "`Was machst du denn?"' A sagt etwas anderes: "`Ich hüpfe auf einem Bein."'
Daraufhin hüpft B auf einem Bein und von der nächsten Nachbarin gefragt, was sie denn macht. Sie antwortet z.\,B.: "`Ich spiele Klavier."' Und so weiter.

Das Spiel ist nach ein, zwei Runden beenden, wenn eine Tutorin antwortet: "`Ich beende dieses Spiel."'
\paragraph{Besondere Hinweise:} Es kann sehr unterschiedlich lange dauern, bis alle die Regeln auf die Reihe bekommen haben. Erschwert wird das Spiel, wenn die Tutorinnen die Regeln nicht explizit erklären, sondern einfach anfangen, und die Spielerinnen die Regeln dann selbst herausfinden müssen.
\paragraph{Wann einsetzen:} Zur allgemeinen Auflockerung und Erheiterung, nach einer Pause zum Brain-Booting.

\section{Bewegungskanon}
\index{Bewegungskanon}
\index{Kanon, Bewegungs-}
\paragraph{Art:} Rhythmisches Bewegungsspiel im Sitzen
\paragraph{Ziel:} Action~-- Schwitzen im Sitzen
\paragraph{Dauer:} 10--15 Minuten
\paragraph{Wir brauchen dazu:} Sitzkreis (ohne Stühle), mindestens zwölf Leute
\paragraph{So geht es:} Die OE-Tutorin bittet im ersten Durchgang alle, ihre Bewegungen nachzumachen:
\begin{enumerate}
\item 3x in die Hände klatschen
\item 3x auf die Oberschenkel schlagen
\item 3x die Hände in die Luft strecken
\item 3x mit beiden Füßen auf den Boden stampfen
\end{enumerate}
Wenn der Grundablauf sitzt, wird es zunehmend schwieriger: Die Tutorin teilt die Gruppe in zwei Hälften. Dann beginnt der Kanon:

Die erste Hälfte beginnt mit dem dem Klatschen. Wenn sie sich zum ersten Mal auf die Schenkel schlägt, beginnt die zweite Gruppe mit dem In-die-Hände-Klatschen.

Wenn auch das sitzt, kommt die schwierigste Stufe: Die Tutorin teilt die Teilis in vier Gruppen und dirigiert.
\paragraph{Wann einsetzen:} Zur allgemeinen Auflockerung und Erheiterung

\section{Labyrinth}
\index{Labyrinth}
\index{Matrix|see{Labyrinth}}
\index{Katze und Maus|see{Labyrinth}}
\paragraph{Alias:} Matrix, Katze und Maus
\paragraph{Art:} Fangen spielen mit Köpfchen
\paragraph{Ziel:} Die Maus muss der Katze entkommen und darf dazu das gemeinsame Labyrinth verändern.
\paragraph{Dauer:} 10--15 Minuten
\paragraph{Wir brauchen dazu:} Platz und einen trockenen Untergrund zum Laufen, mindestens 11~Teilis.
\paragraph{So geht es:} Eine Teili ist die Katze, eine andere Teili ist die Maus. Die anderen stellen sich in einer (möglichst quadratischen) Matrix in Armspannen-Abstand auf, also zum Beispiel in 4 Reihen \`{a} 4 Teilis. Dabei schauen alle in eine Richtung und halten die Arme in Schulterhöhe ausgestreckt. Die Maus hat nun drei Optionen:
\begin{itemize}
  \item Vor der Katze weglaufen durch die Reihen und um die Matrix herum. Unter den ausgestreckten Armen dürfen weder Katze noch Maus hindurchgehen.
  \item "`Labyrinth"' oder "`Matrix"' rufen. Daraufhin drehen sich alle stehen Teilis um 90 Grad. Dadurch werden aus den Längsgängen im Labyrinth plötzlich Quergänge.
  \item An eine Reihe andocken. Die Maus wird dann Teil des Labyrinths. Die alte Katze wird dann Maus, und die Teili am anderen Ende der Reihe, an die sich die Maus angedockt hat, wird neue Katze (Klicker-Effekt). Die Reihe rückt dann auf, damit sie wieder im Raster drin ist.
\end{itemize}
Wenn die Katze die Maus fängt, vertauschen sich die Rollen: Aus Katze wird Maus und umgekehrt.
\paragraph{Wann einsetzen:} Zum Wachwerden und weil's Spaß macht.

\section{Schlange und Hase}
\index{Schlange und Hase}
\index{Adler und Hase|see{Schlange und Hase}}
\paragraph{Alias:} Adler und Hase
\paragraph{Art:} Schlangenfangenspiel
\paragraph{Ziel:} Polonaisenkopf fängt Polonaisenschwanz
\paragraph{Dauer:} 5--10 Minuten
\paragraph{Wir brauchen dazu:} Platz zum Laufen
\paragraph{So geht es:} Alle Teilis stellen sich hintereinander auf (wie bei der Schlange im Supermarkt) und halten sich wie bei einer Polonaise an den Schultern. Der Kopf dieser Schlange ist der Adler, der den Hasen (das Ende der Schlange) fangen muss. Dabei darf die Schlange nicht auseinander reißen.

Wenn ihr Spiel mehrfach hintereinander spielen wollt, kann jemand anders der Hase sein. Der bisherige Schlangenkopf kann dann ans Ende der Schlange gehen.
\paragraph{Varianten:} Der Hase kann auch von der Schlange losgelöst sein. Damit es fair ist, darf der Hase aber nur hoppeln und nicht normal laufen. Das Spielfeld sollte dabei eingegrenzt sein (rundherum ist die "`gefährliche Autobahn"'), damit sich der Hase nicht einfach über eine Mauer retten kann.
\paragraph{Wann einsetzen:} Zum Wachwerden und weil's Spaß macht.

\section{Taaa-Tung!}
\index{Taaa-Tung!}
\index{Eddings weitergeben|see{Taaa-Tung!}}
\paragraph{Art:} Spaßspiel mit Konkurrenz
\paragraph{Ziel:} schnell Stifte im Rhythmus weitergeben
\paragraph{Dauer:} 10--15 Minuten
\paragraph{Wir brauchen dazu:} einen Kniekreis, 1 Stift o.~Ä.~pro Teilnehmerin
\paragraph{So geht es:} Alle knien dicht zusammen im Kreis. Jede hat einen Stift vor sich liegen. Die Tutorin gibt den Rhythmus vor: Taaa-tung (Taaa: Edding mit der rechten Hand greifen, tung: Edding der rechten Nachbarin hinlegen), Taaa-tung, Ta-tung-ta-tung (Ta: Edding greifen, tung: Edding nach rechts legen und dabei in der Hand behalten, ta: den selben Edding wieder nach links legen, tung: Edding der rechten Nachbarin hinlegen). Also: Taaa-tung, Taaa-tung, Ta-tung-ta-tung, Taaa-tung und so weiter. Dabei immer schneller werden. Wer gurkt\footnote{Edding nicht richtig weitergegeben, zwei Eddings am Platz, etc.}, fliegt raus. Danach geht es von vorne los, bis nur noch eine übrig ist.
\paragraph{Besondere Hinweise:} Solange es niemand merkt, kann man bei diesem Spiel auch gut schummeln.
\paragraph{Wann einsetzen:} Zur allgemeinen Erheiterung. Besonders gut nach der letzten Arbeitsphase am Tag.

%\section{Lachspiel}
%\index{Lachspiel}
%\paragraph{Art:} das Lachspiel halt
%\paragraph{Ziel:} alle liegen im Kreis und fangen irgendwann an zu lachen
%\paragraph{Dauer:} 5--10 Minuten
%\paragraph{Wir brauchen dazu:} einen Liegekreis auf einem halbwegs sauberen Fußboden, Kopf an Kopf, Köpfe nach innen
%\paragraph{So geht es:} Wer anfängt, sagt ein laut und deutlich vernehmliches "`Ha!"'. Danach sagt ihre Nachbarin "`Ha-Ha!"'. So geht es immer weiter im Kreis. Bei jeder Lacherin wird es ein "`Ha!"' mehr. Irgendwann wird sehr wahrscheinlich die ganze Runde in Lachen ausbrechen (das ist zumindest das Ziel des Spiels).
%\paragraph{Besondere Hinweise:} Nur einsetzen bei Gruppen, die ernsthaft genug sind, um hemmungslos albern sein zu können.
%\paragraph{Varianten:} Nur ein "`Ha!"' pro Teili. Dafür soll das "`Ha"' immer schneller (quasi: mit positiver Beschleunigung) im Kreis herumwandern. Diese Variante ist aber nicht ganz so lachhaft wie die Originalversion.
%\paragraph{Wann einsetzen:} Bei allgemeiner Anspannung oder nach einer langen anstrengenden Arbeitsphase.

\section{Tropengewitter}
\index{Tropengewitter}
\index{Gewitter|\see{Tropengewitter}}
\paragraph{Alias:} Gewitter
\paragraph{Art:} Geräuschspiel
\paragraph{Ziel:} ein selbst gemachtes Tropengewitter
\paragraph{Dauer:} 5--10 Minuten
\paragraph{Wir brauchen dazu:} einen Stehkreis und eine ernsthaft-spielerische Atmosphäre
\paragraph{So geht es:} Die Tutorin stellt sich in die Mitte des Kreises. Wenn sie eine Spielerin anschaut und etwas vormacht, macht diese es nach~-- solange, bis sie etwas anderes machen soll. Die Tutorin lässt alle zusammen nacheinander folgendes machen:
  \begin{description}
    \item[Die Ruhe vor dem Sturm:] konzentrierte Ruhe, Schweigen!
    \item[Das erste Rauschen der Blätter im Wind:] Fingerspitzen in einer Geldzählbewegung aneinander reiben
    \item[Das Rauschen des nahenden Regens:] Hände aneinander reiben
    \item[Die ersten schweren Tropfen:] mit den Fingern einer Hand langsam schnippen (können erfahrungsgemäß nicht alle)
    \item[Die Tropfen fallen dichter:] mit den Fingern beider Hände schnell schnippen
    \item[Der Regen prasselt vom Himmel:] schnell in die Hände klatschen (wie beim Applaus)
    \item[Es donnert:] auf den Boden springen (nur ein, zwei Leute)
  \end{description}
Nachdem der Höhepunkt des Unwetters erreicht ist, baut die Tutorin das Gewitter in umgekehrter Reihenfolge wieder langsam ab, bis am Ende die Ruhe nach dem Sturm folgt: Die Sonne ist wieder hervorgekommen!
\paragraph{Besondere Hinweise:} Das Spiel klappt meiner Erfahrung nach nicht mit pubertierenden Jugendlichen, da diese zwischendrin zu kichern anfangen oder zu unsicher zum ernsthaften Spielen sind.
\paragraph{Wann einsetzen:} Um den Kopf wieder klar zu bekommen (morgens oder nach einer Pause), oder um ein wenig zur Ruhe zu kommen.

\section{Im Kreis hinsetzen}
\index{Im Kreis hinsetzen}
\index{Hinsetzen, im Kreis}
\paragraph{Art:} nettes "`Einfach-so"'-Spiel
\paragraph{Ziel:} Alle sitzen im Kreis auf den Oberschenkeln der Hinterfrau.
\paragraph{Dauer:} 5 Minuten
\paragraph{Wir brauchen dazu:} einen Stehkreis mit mindestens 8 Mitspielerinnen
\paragraph{So geht es:} Alle drehen sich um 90~Grad nach links, so dass alle hintereinander im Kreis stehen. Dann rücken alle möglichst nahe zusammen. Dabei wird der Kreis kleiner. Sobald es nicht mehr näher geht (Haut an Haut), setzen sich alle gleichzeitig auf die Beine der Hinterfrau. Ist echt bequem!
\paragraph{Besondere Hinweise:} Problematisch bei Menschen, die körperliche Nähe überhaupt nicht mögen (Informatiker? Kleiner Scherz am Rande \ldots).
\paragraph{Wann einsetzen:} Zwischendrin für eine schnelle Spielpause (oder ein schnelles Pausenspiel).

\section{Paranoia}
\index{Paranoia}
\paragraph{Art:} Chaosdynamisches Laufspiel mit allen gleichzeitig
\paragraph{Ziel:} jede versucht, eine "`Beschützerin"' zwischen sich und eine "`Verfolgerin"' zu bringen
\paragraph{Dauer:} 5--10 Minuten
\paragraph{Wir brauchen dazu:} Platz zum Laufen
\paragraph{So geht es:} Jede Spielerin sucht sich eine Beschützerin und eine Verfolgerin, teilt dies aber niemandem mit. (Daher weiß niemand, ob und für wen sie Beschützerin oder Verfolgerin ist.) Dann versucht sie, sich so hinzustellen, dass sie die Beschützerin zwischen sich und der Verfolgerin bringt.

Nach der Halbzeit sind die Rollen vertauscht: Jede Spielerin hat jetzt vor der bisherigen Beschützerin Angst und fühlt sich durch die bisherige Verfolgerin beschützt.
\paragraph{Wann einsetzen:} Zum schnellen Auflockern draußen.

\section{Ich fühle mich jetzt so \ldots}
\index{Ich fühle mich jetzt so \ldots}
\paragraph{Art:} Kurzes Darstellungsspiel
\paragraph{Ziel:} durch Bewegung und Geräusch zeigen, wie man sich gerade fühlt
\paragraph{Dauer:} 5--10 Minuten
\paragraph{Wir brauchen dazu:} Stehkreis
\paragraph{So geht es:} Es geht reihum. Wer dran ist, sagt "`Ich fühle mich heute morgen (jetzt) so:"' und stellt dies durch eine Bewegung und ein Geräusch dar.
\paragraph{Wann einsetzen:} Zur Auflockerung und als Stimmungsbild. Sehr schön auch am Morgen vor der ersten Arbeitsphase.

\section{Zulutanz}
\index{Zulutanz}
\index{Tanz, Zulu-}
\paragraph{Art:} alberner (Pseudo-) Stammestanz der Zulu
\paragraph{Ziel:} singen und dabei im Kreis herumhüpfen
\paragraph{Dauer:} 10--15 Minuten
\paragraph{Wir brauchen dazu:} einen Stehkreis und keine neugierigen Nachbarn
\paragraph{So geht es:} Alle stehen im Kreis. Die Tutorin singt vor, alle singen mit:
\begin{quote}
  If you look at me,\\
  Zulu you will see.\\
  If you stand by me,\\
  Zulu you can be.\\
  Hey! Zulu! Attention!\\
  Look at me!
\end{quote}
Dann macht die Tutorin die Aktionen vor. Erst eins, dann nach jedem Durchgang eins mehr, während die anderen Bewegungen beibehalten werden:
\begin{description}
  \item[Right Hand:] die rechte Hand (im Takt) auf den rechten Oberschenkel schlagen
  \item[Left Hand:] auch die linke Hand auf den linke Oberschenkel schlagen
  \item[Right Foot:] mit den rechten Fuß leicht aufstampfen
  \item[Left Foot:] auch mit den linken Fuß leicht aufstampfen (also mit beiden Füßen hüpfen)
  \item[In a Circle:]  als ganze Gruppe im Kreis herumhüpfen (wie um ein Feuer)
  \item[Down:] in die Knie gehen
  \item[Nod:] mit dem Kopf nicken
  \item[Spin:] alle drehen sich zusätzlich um sich selbst
  \item[Backwards:] jetzt hüpft der Kreis rückwärts
  \item[Double time:] doppelt so schnell singen und tanzen
\end{description}
Das Spiel ist zu Ende, wenn niemand mehr kann oder alle durcheinander purzeln.
\paragraph{Besondere Hinweise:} Nicht nach dem Essen spielen. Die Melodie vorher üben.
\paragraph{Wann einsetzen:} Zur heftigen Auflockerung und zum Herumalbern.


\section{Schuhsalat}
\index{Schuhsalat}
\paragraph{Art:} lustiges Durcheinanderwuselspiel
\paragraph{Ziel:} einen Kreis mit zufällig ausgewählten Schuhen an den Füßen bilden
\paragraph{Dauer:} 10--15 Minuten
\paragraph{Wir brauchen dazu:} Platz (die ganze Gruppe muss in einen Kreis passen)
\paragraph{So geht es:} Alle ziehen ihre Schuhe aus und machen damit in der Mitte einen Haufen. Anschließend nimmt sich jede einen rechten und einen linken Schuh (aus unterschiedlichen Paaren). Jede zieht die Schuhe an (so gut es geht \ldots\ nur nicht kaputt machen). Nun müssen sich die Schuhpaare (an den Füßen) wiederfinden und nebeneinander stellen. Es bilden sich dadurch ein oder mehrere Kreise.
\paragraph{Besondere Hinweise:} Funktioniert nur, wenn die Gruppe halbwegs unverkrampft mit Körper- und Fußschweißkontakt umgehen kann.
\paragraph{Wann einsetzen:} Zum Auflockern zwischendurch.

Vielen Dank an Christoph aus Darmstadt für dieses Spiel.


\section{Daumenwrestling}
\index{Daumenwrestling}
\index{Fingerwrestling|see {Daumenwrestling}}
\index{Fingerringer|see {Daumenwrestling}}
\index{Multiplayer-Daumenwrestling|see {Daumenwrestling}}
\paragraph{Alias:} monochroms massives Multiplayer-Daumenwrestling
\paragraph{Art:} handfestes Daumenwrestling
\paragraph{Ziel:} mit dem eigenen Daumen einen anderen Daumen niederringen
\paragraph{Dauer:} 2--30 Minuten (je nach Anzahl der Runden)
\paragraph{Wir brauchen dazu:} 2--12 Spielerinnen (geht auch mit mehr)
\paragraph{So geht es:}
Die rechten Hände von zwei (oder mehr) Spielerinnen an den Fingerspitzen zusammen, dann die Finger zu einer Faust, Daumen hoch. Wer zuerst den Daumen der anderen runterdrückt und fixiert, hat gewonnen.

Bei sehr vielen Spielerinnen kann man die linke Hand dazu benutzen, verschiedene Netzwerkstrukturen zu bilden (sehr lustig für Informatikerinnen!).

\paragraph{Besondere Hinweise:} Aufwärmen per Fingergymnastik nicht vergessen. Achtet auf kurze Fingernägel und spielt fair!
\paragraph{Wann einsetzen:} zum Auflockern zwischendurch
\paragraph{Varianten:} siehe \url{http://www.monochrom.at/daumen/}


\section{Wäscheklammern}
\index{Wäscheklammern}
\index{Geben \& Nehmen|see{Wäscheklammern}}
\index{Klammern|see{Wäscheklammern}}
\paragraph{Alias:} Geben \& Nehmen
\paragraph{Art:} \emph{sehr} intensives Lauf- und Wuselspiel
\paragraph{Ziel:} den anderen Spielerinnen Wäscheklammern von der Kleidung abnehmen oder anstecken
\paragraph{Dauer:} 5--10 Minuten
\paragraph{Wir brauchen dazu:} drei Wäscheklammern pro Teili, viel Platz zum Laufen (am besten draußen)
\paragraph{So geht es:} Jede Spielerin bekommt drei Wäscheklammern und steckt sich diese gut sichtbar an die Kleidung. Die Tutorin erklärt vor jeder Runde die Regeln und eröffnet und beendet die Runden.

\emph{Erste Runde:} Jede versucht, den anderen Spielerinnen ihre Wäscheklammern abzunehmen und sich selbst an die Kleidung zu stecken.

\emph{Zweite Runde:} Jede Spielerin versucht, ihre eigenen Wäscheklammern anderen Spielerinnen an die Kleidung zu stecken. Wer eine Wäscheklammer fallen lässt, steckt diese wieder bei sich selbst an die Kleidung.

\paragraph{Besondere Hinweise:} Räumt mögliche Stolperfallen, Flaschen und Ähnliches vorher aus dem Weg. Wenn möglich, zieht Kleidungsstücke vorher aus, die leicht reißen (beispielsweise dünne Stoffjacken). Achtet darauf, dass niemand scharfkantige Uhren, Ringe oder andere gefährliche Schmuckstücke trägt.
\paragraph{Wann einsetzen:} Zur intensiven Auflockerung zwischendurch. Besser nicht direkt nach dem Mittagessen.

\section{Zeitungsninja}
\index{Zeitungsninja}
\paragraph{Art:} Alle-gegen-eine-Spiel, das für die eine Person \emph{sehr} intensiv ist
\paragraph{Ziel:} einer Spielerin  Wäscheklammern von der Kleidung abnehmen und dabei nicht von der Zeitung getroffen werden
\paragraph{Dauer:} 5--10 Minuten pro Runde
\paragraph{Wir brauchen dazu:} 10--15 Wäscheklammern, ein Tuch o.\,ä.~zum Verbindungen der Augen, eine zusammengerollte Zeitung, mindestens 2x2\,m Platz
\paragraph{So geht es:} Eine Spielerin ist der "`Ninja"'. Sie bekommt die Wäscheklammern an die Kleidung gesteckt, die Augen verbunden und die zusammengerollte Zeitung in eine Hand. Die Mitspielerinnen versuchen nun, ihr die Wäscheklammern abzunehmen. Währenddessen versucht der Ninja, möglichst viele Spielerinnen mit der Zeitung zu erwischen. Wer von der Zeitung getroffen wird, scheidet aus dem Runde aus und setzt sich an den Rand.

Die Runde ist zu Ende, wenn der Zeitungsninja keine Klammern mehr hat, alle Spielerinnen ausgeschieden sind oder der Ninja völlig K.\,O. ist.

\paragraph{Besondere Hinweise:} Die Mitspielerinnen sind dafür verantwortlich, dass der Zeitungsninja nirgendwo gegenstößt, und helfen ihr bei Bedarf, wieder in die Mitte des Raums zu gehen.

Besonders lustig ist es, wenn jemand mit Kampfkunsterfahrung den Zeitungsninja mimt.

\paragraph{Wann einsetzen:} Zur intensiven Auflockerung zwischendurch.


\section{Pause}
\index{Pause}
\index{Auszeit|see{Pause}}
\paragraph{Alias:} Auszeit
\paragraph{Art:} Pause zum Auslüften, Rauchen, Pinkeln oder einfach für andere Gedanken
\paragraph{Ziel:} besseres Weiterarbeiten danach
\paragraph{Dauer:} 5--15 Minuten
\paragraph{Wir brauchen dazu:} ---
\paragraph{So geht es:} Die Tutorin fragt die Gruppe, wie groß der Bedarf nach einer Pause ist. Wenn der Bedarf vorhanden ist, setzt die Tutorin eine Uhrzeit fest, zu der es nach der Pause weitergeht.
\paragraph{Besondere Hinweise:} Sehr wichtiges Spiel! Auch kleine Pausen können die Arbeitsfähigkeit der Gruppe enorm steigern und die gelebte Genervtheit enorm verringern.
\paragraph{Wann einsetzen:} Wenn die Gruppe genervt, erschöpft, hibbelig oder was auch immer ist. Nur am Anfang zum Wachwerden ist die Pause nicht so geeignet.

