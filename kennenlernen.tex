\chapter{Kennenlernspiele}
\index{Kennenlernspiele}

\section{Standpunkte}
\index{Standpunkte}
\index{Aufstellung|see{Standpunkte}}
\index{Menschenaufläufe|see{Standpunkte}}
\label{standpunkte}

\paragraph{Alias:} Aufstellung, Menschenaufläufe
\paragraph{Art:} sehr transparente Gruppenbildung nach inhaltlichen Kriterien
\paragraph{Ziel:} Aufzeigen von Gemeinsamkeiten, inhaltliches Gruppieren
\paragraph{Dauer:} 1--5 Minuten
\paragraph{Wir brauchen dazu:} ein bisschen Platz
\paragraph{So geht es:} Die Tutorin gibt vor, welcher Platz was bedeutet, zum Beispiel:
	\begin{itemize}
		\item am Fenster: will Thema 1 bearbeiten
		\item in der rechten hinteren Ecke: will Thema 2 bearbeiten
		\item an der Tür: will Thema 3 bearbeiten
	\end{itemize}
	Dann stellen sich die Teilis entsprechend ihrer Interessen auf. Es wird schnell sichtbar, wie viele sich für welches Thema interessieren, und ob eine Gruppe möglicherweise zu klein oder zu groß wird.
\paragraph{Varianten:}
	\begin{description}
		\item[Meinungsbild:] Stimme ich der These zu oder nicht?
		\item[Meinungsbild mit Zwischenstufen:] Die Teilis stellen sich auf einer gedachten Linie auf. Die beiden Endpunkte stellen jeweils die Extrempositionen dar, alles dazwischen sind entsprechend ein abgestuftes "`Sowohl-als-auch"'.
		\item [Orgakram:] Kaffee, Tee oder Kakao zum Frühstück? Vegetarisch oder totes Tier?
		\item[Sortieren:] siehe nächstes Spiel
	\end{description}
\paragraph{Wann einsetzen:} Um Gruppen nach Interesse zu bilden.

\section{Sortieren}
\index{Sortieren}
\label{sortieren}
\paragraph{Art:} Sortierspiel
\paragraph{Ziel:} die Teilis lernen sich in Bezug auf eine Eigenschaft kennen und mischen sich gleichzeitig
\paragraph{Dauer:} 1--5 Minuten
\paragraph{Wir brauchen dazu:} ein bisschen Platz
\paragraph{So geht es:} Die Tutorin gibt vor, wonach sich die Teilis sortieren, zum Beispiel:
	\begin{itemize}
		\item nach Matrikelnummern
		\item nach Anzahl der Semester
		\item nach Länge des Anreiseweges
		\item nach Anzahl der bisher besuchten Seminare
	\end{itemize}
\paragraph{Wann einsetzen:}
\begin{itemize}
	\item um die Gruppe neu zu mischen (etwa für eine zweite Runde von \emph{Mirko Mondsüchtig} (nächstes Spiel, Seite \pageref{mirko}))
	\item um das Kennenlernen in Bezug auf eine bestimmte Eigenschaft zu fördern und etwas Bewegung in die Gruppe zu bringen
\end{itemize}

\section{Mirko Mondsüchtig}
\label{mirko}
\index{Mirko Mondsüchtig}
\index{Alliterationen|see{Mirko Mondsüchtig}}
\index{Adjektiv-Name|see{Mirko Mondsüchtig}}
\paragraph{Alias:} Alliterationen, Adjektiv-Name
\paragraph{Art:} ruhiges, sehr lustiges Namenslernspiel
\paragraph{Ziel:} Namen hören und durch Assoziation mit Worten einprägen
\paragraph{Dauer:} bei $n$ Spielerinnen etwa $\frac{n^2}{10}$ Minuten
\paragraph{Wir brauchen dazu:} Platz für einen Stehkreis; 5-15 Spielerinnen (darüber dauert es zu lange)
\paragraph{So geht es:} Einen Kreis bilden. Reihum nennt jede einen Spruch wie "`Ich bin der jubelnde Julian"' oder "`Ich bin die singende Sabine"': Das Adjektiv soll also mit dem gleichen Buchstaben (oder Laut) beginnen wir der eigene Vorname. Dazu macht die Spielerin eine passende Geste oder Bewegung. Wer dran ist, wiederholt die vorherigen Sprüche und Bewegungen, bevor sie den eigenen bringt~-- ganz ähnlich dem Spiel \emph{Kofferpacken}.
\paragraph{Besondere Hinweise:} Durch die Verknüpfung mit anderen Eindrücken (Worte, Bewegungen) und Wahrnehmungen verbessert dieses Spiel die Gedächtnisleistung bei den Namen.
\paragraph{Varianten:} Wenn ihr einen besonders guten Lerneffekt habt, dann spielt zwei Runden, und verändert zwischen beiden Runden die Reihenfolge~-- zum Beispiel, indem ihr sie sich nach Matrikelnummern sortieren lasst (siehe die \emph{Sortieren} auf Seite \pageref{sortieren}).
\paragraph{Wann einsetzen:} Wenn die Gruppe die Namen noch nicht oder kaum kennt.

\section{Zipp-Zapp}
\index{Zipp-Zapp}
\paragraph{Art:} Schnelles, actionreiches Namenswiederholspiel
\paragraph{Ziel:} bereits gelernte Namen schnell wiederholen
\paragraph{Dauer:} 5--10 Minuten
\paragraph{Wir brauchen dazu:} ---
\paragraph{So geht es:}
Wieder sitzen alle im Kreis, nur diesmal eine in der Mitte. Diese dreht sich herum, zeigt auf Personen und sagt:
\begin{itemize}
\item Zipp: Betreffende Person nennt den Namen der linken Nachbarin, oder auch nicht
\item Zapp: Name der rechten Nachbarin
\item Zipp-Zapp: Alle tauschen die Plätze
\end{itemize}
Die angesprochene Person muss schnell reagieren und den richtigen Namen nennen! Gelingt ihr das nicht, muss sie in die Mitte. Dadurch ändert es sich schnell, wen man zum Nachbarn hat.
\paragraph{Wann einsetzen:} Wenn die Gruppen die Namen schon einmal gehört hat.

\section{Partnerinneninterview}
\index{Partnerinneninterview}
\paragraph{Art:} klassisches Kennenlernspiel über die Eigenschaften der Leute
\paragraph{Ziel:} Partnerin der Gruppe vorstellen
\paragraph{Dauer:} 20 Minuten plus (Teilnehmerinnen$\cdot$2) Minuten
\paragraph{Wir brauchen dazu:} Schreibzeug für alle, einen Sitzkreis
\paragraph{So geht es:}
Die Spielerinnen finden sich zu Pärchen zusammen (oder werden von der Seminarleiterin eingeteilt), die sich im Raum oder Gebäude verteilen. Beide Teile jedes Pärchens interviewen sich gegenseitig je 10 Minuten (also insgesamt 20 Minuten pro Pärchen).

Die Interviewerin kann alles fragen, was sie interessiert: Namen, Wohnort, Arbeit, Alter, Hobbys, Erwartungen, Haustiere, Anekdoten, Erlebnisse, \ldots\ Die Leute können sich auch ruhig ein paar privatere Fragen stellen (zum Beispiel nach dem Beziehungsstatus). Wenn es für das Seminar sinnvoll ist, kann die Seminarleiterin auch vorher ein paar Leitfragen anschreiben, an denen sich die Spielerinnen orientieren können. Beispiele:
  \begin{itemize}
    \item Was ging dir auf dem Weg hierher durch den Kopf?
    \item Was würdest du tun, wenn Geld keine Rolle spielte?
    \item Als was für ein Tier wärst du geboren worden?
    \item Was möchtest du in 5--10 Jahren sein?
    \item Was für Erwartungen hast du an das Seminar?
  \end{itemize}

Beim Interview kann es hilfreich sein, sich die Fakten aufzuschreiben.

Nach der Interviewphase kommen die Spielerinnen wieder zum Stuhlkreis zusammen. Die Spielerinnen stellen nun zwei Minuten lang ihre jeweilige Partnerin der Gruppe vor (und entsprechend umgekehrt natürlich auch).

Nach jeder Vorstellung fragt die Seminarleiterin, ob die Vorgestellte sich gut dargestellt findet. Wenn dem so ist, geht es mit einem Applaus\footnote{Sehr wichtig für das Selbstbewusstsein der Spielerinnen. Außerdem lässt sich für ein Rhetorikseminar der Applaus an dieser Stelle sehr gut einführen.} und der nächsten Vorstellung weiter.
\paragraph{Besondere Hinweise:} Weist die Spielerinnen darauf hin, dass sie selbst auf die Zeit achten sollen, damit beide Interviews etwa gleich lang werden.

Die Interviews können auch auf einem Spaziergang stattfinden. Dann wird es allerdings mit dem Aufschreiben schwierig.
\paragraph{Varianten:}
\begin{itemize}
  \item Lügen-Porträt (nächstes Spiel, Seite~\pageref{luegenportrait})
  \item Pöstchenvergabe (Seite~\pageref{poestchenvergabe})
  \item \emph{vor} dem Interview noch nicht erwähnen, dass die Spielerinnen danach ihre Partnerin vorstellen sollen~-- so wird das Interview viel persönlicher  
\end{itemize}

\paragraph{Wann einsetzen:} Wenn die Leute die Namen schon halbwegs kennen und sich jetzt etwas detaillierte kennen lernen sollen. Auch gut als erste Präsentationsübung auf einem Rhetorikseminar geeignet.

\section{Lügen-Porträt}
\index{Lügen-Porträt}
\label{luegenportrait}
\paragraph{Art:} ruhiges Kennenlern-Ratespiel über Eigenschaften der Leute
\paragraph{Ziel:} Partnerin mit wahren und erfundenen Informationen vorstellen
\paragraph{Dauer:} 30--45 Minuten
\paragraph{Wir brauchen dazu:} Schreibzeug für alle, leere Plakate und Moderationsstifte
\paragraph{So geht es:} Funktioniert wie das Partnerinneninterview. Bei der Vorstellungsphase gibt es allerdings Unterschiede:
Die Interviewerin berichtet der Gruppe die vier interessantesten Einzelheiten über die Interviewte. Die Tutorin kann diese Einzelheiten auf einem Plakat visualisieren, da sich die Erstis erfahrungsgemäß nicht alles merken können.

Eine Einzelheit soll dabei "`gelogen"' (von der Interviewerin erfunden) sein. Die ganze Gruppe soll dann raten, welches die erfundene Information war.
\paragraph{Wann einsetzen:} Wenn die Gruppe die Namen kennt und sich die Leute gegenseitig schon ein bisschen einschätzen können. Vielleicht nach dem Kennenlern-Bingo.

\section{Pöstchenvergabe}
\label{poestchenvergabe}
\index{Pöstchenvergabe}
\paragraph{Art:} ruhiges Kennenlernspiel über Eigenschaften der Leute mit "`Kandidatur"' als Aufhänger
\paragraph{Ziel:} Partnerin als "`Kandidatin"' für ein erfundenes Pöstchen vorstellen
\paragraph{Dauer:} 20 Minuten plus (Teilnehmerinnen$\cdot$2) Minuten
\paragraph{Wir brauchen dazu:} Schreibzeug für alle, einen Sitzkreis
\paragraph{So geht es:} Funktioniert wie das Partnerinneninterview. Die Leitfragen sind allerdings:
  \begin{itemize}
    \item In welcher Rolle hast du dich beim letzten Plenum oder deiner letzten Gruppendiskussion gesehen? (Für diesen "`Posten"' wird die Interviewte kandidieren.) Beispiele: Zwischenruferin, Flaschenumstoßerin, Protokollantin, Sprüchermacherin, Inzurückhaltungüberin, Zusammenfasserin \ldots
    \item Warum bist du für diesen "`Posten"' gut geeignet?
    \item Was sind deine weiteren Qualifikationen?
  \end{itemize}

Nach der Vorstellung jeder "`Kandidatin"' darf das Publikum noch Fragen an sie stellen. Gewählt wird allerdings nicht.
\paragraph{Besondere Hinweise:} Dieses Spiel haben Marlies und ich erfunden und auf der 30ten KIF in Dortmund erstmalig ausprobiert.
\paragraph{Wann einsetzen:} Nach der Namensrunde bei einem Seminar, das Kandidaturen, Diskussionen oder Ähnliches behandelt.

\section{Chaosrunde}
\index{Chaosrunde}
\paragraph{Art:} bewegtes Kennenlernspiel zu allen Aspekten
\paragraph{Ziel:} sich nacheinander kurz mit vielen anderen unterhalten
\paragraph{Dauer:} 15--20 Minuten
\paragraph{Wir brauchen dazu:} ---
\paragraph{So geht es:}
Alle gehen kreuz und quer durch den Raum. Wenn die Moderatorin in die Hände klatscht, finden sich die Teilis zu zweit zusammen und fragen sich gegenseitig aus. Wenn die Moderatorin wieder klatscht, gehen alle weiter, bis sie sich beim nächsten Klatschen mit jemand anderem unterhalten.

Nach mehreren Durchgängen setzen sich alle wieder in den Kreis. Dann werden reihum die Teilis vorgestellt, indem alle erzählen, was sie (eventuell) in den Gesprächen von der Teilnehmerin erfahren haben.
\paragraph{Wann einsetzen:} Wenn die Gruppe die Namen halbwegs kennt.

\section{Kennenlern-Obstsalat}
\index{Kennenlern-Obstsalat}
\index{Obstsalat, Kennenlern-}
\paragraph{Art:} sehr actionreiches Kennenlernspiel über die Eigenschaften der Leute
\paragraph{Ziel:} Gemeinsamkeiten finden, Auflockerung, Wachwerden.
\paragraph{Dauer:} 10--30 Minuten (je nach Lust und Laune kann es auch schon mal eine Stunde werden)
\paragraph{Wir brauchen dazu:} ---
\paragraph{So geht es:} Geschlossenen Sitzkreis mit einem Stuhl weniger, als Leute da sind. Rucksäcke, Blöcke, Stifte und andere Gegenstände sollten weit weg in Sicherheit sein. Alle Teilis sollten außerdem ihre Schuhe zugebunden haben.

 Eine steht in der Mitte und sagt "`Ich mag alle, die \ldots"' und danach etwas über sich. Beispiele: "`Ich mag alle, die keine Brille tragen."' (sie trägt also selbst keine Brille), "`\ldots\ die im ersten Quartal des Jahres Geburtstag haben."', "`\ldots\ die Informatik studieren."' (großes Gedränge) oder "`\ldots\ die morgens schlecht aus dem Bett kommen."'

Dann stehen alle auf, auf die dieses Merkmal zutrifft, und suchen sich einen neuen Platz (nicht den Platz der Nachbarin, sonst wird es zu einfach). Wer vorher in der Mitte war, sollte dabei versuchen, einen der freien Plätze zu bekommen. Wer keinen Platz kriegt, steht als Nächste in der Mitte.
\paragraph{Besondere Hinweise:} Wenn sich die Gruppe schon vertrauter ist, können auch Eigenschaften genannt werden wie "`\ldots\ die sich schon einmal in ihren Lehrer verliebt haben."',  "`\ldots\ die in der Schule schon einmal sitzen geblieben sind."' oder "`\ldots\ die erst nach 18 das erste Mal Sex hatten."' (Diese Frage kam mal beim Spielen spät abends auf einem Seminar.)

Achtet aber darauf, dass sich die Gruppe mit dem entsprechenden Level von "`Outing"' auch wohl fühlt. Es müssen ja auch nicht immer alle wahrheitsgemäß aufstehen bzw.~sitzen bleiben.

Schaut vorher auch, ob die Stühle so ein Spiel überhaupt aushalten~-- bei der Ersti-Fahrt 1999 haben wir mit diesem Spiel fünf Stühle geschrottet.

Außerdem sollten vorher alle Teilis ihre Schuhe zugebunden haben (wir hatten auch schon einen Schnürsenkel-Stolper-Unfall).

\paragraph{Wann einsetzen:} Eigentlich immer~-- egal, wie gut sich die Gruppe kennt.  Auch gut zur Auflockerung oder zum Wachwerden geeignet.

\section{Kennenlern-Bingo}
\index{Kennenlern-Bingo}
\index{Bingo|see{Kennenlern-Bingo}}
\paragraph{Art:} angeregtes Kennenlern-Ratespiel über Eigenschaften der Leute
\paragraph{Ziel:} Leute zu finden, die bestimmte Eigenschaften haben; Gemeinsamkeiten finden
\paragraph{Dauer:} 15--20 Minuten
\paragraph{Wir brauchen dazu:} pro Person eine Kopie des "`Kennenlern-Bingo"'-Zettels (Seite \pageref{bingo})
\paragraph{So geht es:} Alle bekommen eine Kopie des "`Kennenlern-Bingo"'-Zettels und gehen damit im Raum herum. Dabei versuchen sie jemanden zu finden, auf die die Beschreibung in einem der Kästchen zutrifft. Diejenige unterschreibt dann im entsprechenden Kästchen. Wer vier Kästchen in einer Reihe ausgefüllt bekommen hat~-- vertikal, horizontal oder diagonal~--, hat ein \emph{Bingo} und ruft lauft "`Bingo"'. Das Spiel geht dann aber trotzdem noch weiter, bis es von der Tutorin nach 10--15 Minuten beendet wird.

Danach in der Runde können alle noch kurz sagen, was sie besonders Interessantes beim Spielen herausgefunden haben (z.\,B.~"`Niemand trägt etwas Handgemachtes."' oder "`\ldots\ spricht russisch."').
\paragraph{Wann einsetzen:} Wenn die Gruppe schon die Namen kennt.

\section{Streichholzvorstellung}
\index{Streichholzvorstellung}
\paragraph{Art:} Selbstvorstellung mit begrenzter Redezeit
\paragraph{Ziel:} jede darf sich der Gruppe vorstellen, solange ein Streichholz brennt
\paragraph{Dauer:} 30 Sekunden pro Spielerin
\paragraph{Wir brauchen dazu:} eine Schachtel normal große Streichhölzer, halbwegs feuerfeste Tische, einen Sitzkreis
\paragraph{So geht es:} Eine Schachtel Streichhölzer geht im Sitzkreis herum. Wer die Schachtel hat, entzündet ein Streichholz und stellt sich der Gruppe vor. Wenn das Streichholz ausgeht (oder die Spielerin es ausschüttelt, um sich nicht die Finger zu verbrennen), gibt die Spielerin die Schachtel (und damit das Wort) weiter.
\paragraph{Besondere Hinweise:} Nicht auf Polstersesseln oder bei Tischdecken benutzen, da Spielerinnen heiße Streichhölzer schon mal fallen lassen. Vorsicht ist auch bei praktisch extrem unbegabten Spielerinnen geboten. Einige Spielerinnen fallen bei der Vorstellung erfahrungsgemäß heraus, da ihnen das Streichholz sehr kurz nach dem Anzünden wieder ausgeht.
\paragraph{Wann einsetzen:} Zur kurzen, überblicksartigen Vorstellung, wenn die Leute sich im Laufe des Seminars noch besser kennen lernen können. Auch bei relativ großen Gruppen und Vielrednerinnen sehr effektiv.

\section{Gegenstand aussuchen}
\index{Gegenstand aussuhen}
\paragraph{Art:} Kennenlernspiel mit Metaphern und persönlichen Assoziationen
\paragraph{Ziel:} einen Gegenstand aussuchen, mit dem man etwas assoziiert
\paragraph{Dauer:} bei $n$ Personen und $m$ Runden etwa $\frac{m\cdot n}{2}$ Minuten \
\paragraph{Wir brauchen dazu:} 10--20 unterschiedliche Gegenstände, zum Beispiel ein Buch, ein Stofftier, einen Jonglierball, einen Filzstift, ein Apfel, einen Schal, eine Kaffeetasse \ldots je verschiedener die Gegenstände, desto besser. 
\paragraph{So geht es:} Die Teilis sitzen im Stuhlkreis, in dessen Mitte die Gegenstände liegen. Jetzt sucht sich jede der Reihe nach einen Gegenstand aus, mit dem sie sich indentifiziert oder mit dem sie etwas Persönliches assoziiert. Sie zeigt den anderen den Gegenstand, erzählt die dazu passende Geschichte und legt den Gegenstand dann wieder zurück.
\paragraph{Wann einsetzen:} Für spielerisches Kennenlernen über die bloßen Fakten hinaus. Funktioniert am besten bei Veranstaltungen, die einen gewissen Selbsterfahrungsanteil besitzen, weniger bei einer Office-Schulung. 
\paragraph{Varianten:} Lässt sich auch mit dem Partnerinterview kombinieren oder als Feed-back-Technik zur Persönlichkeit benutzen, wenn die anderen aus der Gruppe Gegenstände für diejenige aussuchen, die auf dem "`heißen Stuhl"' sitzt.

