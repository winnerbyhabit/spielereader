\chapter{Spiele zur Gruppenarbeit, Kommunikation und Gruppendynamik}
\index{Gruppenarbeit}
\index{Kommunikation}
\index{Gruppendynamik}

\section{NASA-Spiel}
\index{NASA-Spiel}
\label{nasa}
\paragraph{Art:} Teamspiel
\paragraph{Ziel:} Erkennen der unterschiedlichen Arbeitsweisen und Entscheidungsfindungsprozesse bei Einzelarbeit, Gruppenarbeit und Delegation
\paragraph{Dauer:} 1,5--2 Stunden
\paragraph{Wir brauchen dazu:} pro Person eine Spielanleitung (Seite \pageref{nasa-kopien}) sowie Schreibkram
\paragraph{So geht es:} Jede Spielerin erhält das Blatt aus dem Anhang. Für die zweite Phase bilden sich mindestens zwei Gruppen von etwa acht Spielerinnen.
\paragraph{Verlauf:} Zum Verlauf des Spiels wird den Teilnehmerinnen Folgendes erklärt:

\emph{In diese Übung spielen wir unsere Möglichkeiten, Entscheidungen zu treffen, an einem Modell durch. Wir erfahren dabei, wie sich Entscheidungen sinnvoll durchführen lassen und was für Hindernisse im Wege stehen können.}

Bitte achtet bei den ersten drei Spielphasen darauf, dass die Spielerinnen die vorgegebene Zeit nicht überschreiten. Zeitdruck ist ein sehr wichtiges Element bei diesem Spiel!
\subparagraph{1.~Einzelentscheidung (5 Minuten):} Ihr versucht~-- jede für sich allein~-- die gestellte Aufgabe zu lösen.
\subparagraph{2.~Gruppenentscheidung (15 Minuten):} Das Ziel ist ein Beschluss der Gruppe, mit dem jede von euch einverstanden ist. Das bedeutet, dass der Rang jedes der 15 Gegenstände, die für das Überleben notwendig sind, die Zustimmung einer jeden von euch haben muss, um ein Teil des Gruppenbeschlusses zu werden.

Es wird sich nicht in allen Punkten erreichen lassen, dass alle Gruppenmitglieder zu der gleichen Meinung kommen. Ihr versucht aber als Gruppe, jeden Punkt so zu diskutieren und zu beschließen, dass alle Mitglieder zumindest teilweise zustimmen können.
\subparagraph{3.~Delegiertenentscheidung (10 Minuten):} Jede Gruppe wählt aus ihrer Mitte zwei Vertreterinnen, die nach Meinung der Gruppe am besten mit der Materie umgehen können. Die Vertreterinnen aller Gruppen setzen sich zusammen und entscheiden im Plenum noch einmal. Die anderen Teilnehmerinnen dürfen dabei zuhören  und -schauen, aber während der Verhandlung nichts sagen. \emph{Diesen Durchgang könnt ihr bei Zeitknappheit wegfallen lassen.}
\subparagraph{4.~Auswertung (15--30 Minuten):} Die verschiedenen Ergebnisse werden untereinander und mit dem Sachverständigenergebnis der NASA-Fachleute verglichen:
	\begin{enumerate}
	\item Sauerstofftanks
	\item Wasser
	\item Sternenkarte
	\item Nahrungskonzentrat
	\item Fernmelde-Empfänger
	\item Sender
	\item Nylonseil
	\item Erste-Hilfe-Koffer
	\item Fallschirmseide
	\item Schlauchboot
	\item Signalpatronen
	\item Pistole
	\item Trockenmilch
	\item Heizgerät
	\item Magnetkompass
	\item Streichhölzer
	\end{enumerate}
Dies ist nicht die \emph{einzig wahre} Lösung. Ziel des Spieles ist nicht ein möglichst gutes Ergebnis, sondern die verschiedenen Wege dort hin und die Erfahrungen und Erkenntnisse dabei.

\emph{Bitte macht unbedingt die Auswertung!} Ohne Auswertung ist das NASA-Spiel wertlos!

Bei der Auswertung soll in etwa das herauskommen, was bei der OE-Vor\-be\-rei\-tungs\-fahrt 2000 herausgekommen ist:
\begin{itemize}
  \item neue/andere/bessere Ideen durch Gruppenarbeit statt Einzelarbeit
  \item vorher sollte man die Fakten klären und einen gleichen Wissensstand herstellen
  \item vorher sollte man auch das Ziel und die Prioritäten bei der Aufgabe klären
  \item es ist wichtiger, dass die Gruppe ein Ergebnis bekommt, als dass Einzelne ihr Ergebnis versuchen durchzubringen
  \item vorher sollte man sich über die Vorgehensweise einigen
  \item auch knappe Zeit sollte in Ruhe genutzt werden; Hektik ist kontraproduktiv
  \item man sollte sich bewusst machen, was man \emph{nicht} weiß und keine Zeit mit sinnlosem Herumraten verschwenden
  \item mögliche Entscheidungsfindungsmodelle:
    \begin{description}
      \item [Konsens:] alle können \emph{gut} mit dem Ergebnis leben; möglichst \emph{Win-Win-Lösung,} die nicht unbedingt in der Mitte zwischen den Ausgangspositionen liegen muss; erfordert das Klären von Interessen
      \item [Kompromiss:] man trifft sich in der Mitte; es reicht, die Standpunkte zu klären\footnote{ein Kompromiss kann über Konsens erreicht werden, muss aber nicht}
      \item [Mehrheitsentscheid:] schnelle Entscheidungen, auch wenn nicht jeder damit einverstanden ist; das Gegenteil von Konsens; Nachteile:
      \begin{itemize}
        \item nicht alle stehen unmittelbar dahinter und sabotieren eventuell das Ergebnis
        \item nicht alle verstehen unbedingt die Lösung
        \item kompetente Lösungen werden eventuell überstimmt
      \end{itemize}
    \end{description}
\end{itemize}
\paragraph{Besondere Hinweise:} Das NASA-Spiel wurde erstmals veröffentlicht bei \cite{nasa}.

\section{Schiffbruch}
\label{schiffbruch}
\index{Schiffbruch}
\paragraph{Art:} Teamspiel
\paragraph{Ziel:} Erkennen der unterschiedlichen Arbeitsweisen und Entscheidungsfindungsprozesse bei Einzelarbeit, Gruppenarbeit und Delegation
\paragraph{Dauer:} 1,5--2 Stunden
\paragraph{Wir brauchen dazu:} pro Person eine Spielanleitung (Seite \pageref{schiffbruch-kopien}) sowie Schreibkram
\paragraph{So geht es:} Jede Spielerin erhält das Blatt aus dem Anhang. Für die zweite Phase bilden sich mindestens zwei Gruppen von etwa acht Spielerinnen.
\paragraph{Verlauf:} Zum Verlauf des Spiels wird den Teilnehmerinnen Folgendes erklärt:

\emph{In diese Übung spielen wir unsere Möglichkeiten, Entscheidungen zu treffen, an einem Modell durch. Wir erfahren dabei, wie sich Entscheidungen sinnvoll durchführen lassen und was für Hindernisse im Wege stehen können.}

Bitte achtet bei den ersten drei Spielphasen darauf, dass die Spielerinnen die vorgegebene Zeit nicht überschreiten. Zeitdruck ist ein sehr wichtiges Element bei diesem Spiel!
\subparagraph{1.~Einzelentscheidung (5 Minuten):} Ihr versucht~-- jede für sich allein~-- die gestellte Aufgabe zu lösen.
\subparagraph{2.~Gruppenentscheidung (15 Minuten):} Das Ziel ist ein Beschluss der Gruppe, mit dem jede von euch einverstanden ist. Das bedeutet, dass der Rang jedes der 15 Gegenstände, die für das Überleben notwendig sind, die Zustimmung einer jeden von euch haben muss, um ein Teil des Gruppenbeschlusses zu werden.

Es wird sich nicht in allen Punkten erreichen lassen, dass alle Gruppenmitglieder zu der gleichen Meinung kommen. Ihr versucht aber als Gruppe, jeden Punkt so zu diskutieren und zu beschließen, dass alle Mitglieder zumindest teilweise zustimmen können.
\subparagraph{3.~Delegiertenentscheidung (10 Minuten):} Jede Gruppe wählt aus ihrer Mitte zwei Vertreterinnen, die nach Meinung der Gruppe am besten mit der Materie umgehen können. Die Vertreterinnen aller Gruppen setzen sich zusammen und entscheiden im Plenum noch einmal. Die anderen Teilnehmerinnen dürfen dabei zuhören und -schauen, aber während der Verhandlung nichts sagen. \emph{Diesen Durchgang könnt ihr bei Zeitknappheit wegfallen lassen.}
\subparagraph{4.~Auswertung (15--30 Minuten):} Die verschiedenen Ergebnisse werden untereinander und mit dem Sachverständigenergebnis (von US-Marineoffizierinnen) verglichen.

Laut den Expertinnen sind bei einem Schiffbruch diejenigen Artikel am wichtigsten, die einem helfen, sich bei potenziellen Retterinnen bemerkbar zu machen sowie um kurzfristig zu überleben.

Navigationsartikel sind nicht wichtig, weil man zu weit vom Land entfernt ist, um aus eigener Kraft dorthin zu kommen. Weder Nahrung noch Wasser würden außerdem lange genug dafür reichen. Der Mensch kann~-- ohne bleibenden Schaden zu nehmen~-- 36~Stunden ohne Wasser auskommen und 30~Tage ohne Nahrung.

Auf der südlichen Halbkugeln sind die Jahreszeiten den unsrigen entgegengesetzt: Dort ist also Sommer, wenn bei uns Winter ist (und umgekehrt). Die Meeresströmungen bewegen sich dort gegen den Uhrzeigersinn (auf der nördlichen Halbkugel im Uhrzeigersinn). Das Rettungsboot treibt also in Richtung Antarktis.

	\begin{enumerate}
		\item Rasierspiegel. Damit kann man die Sonne reflektieren und Signale senden.
		\item Dieselöl. Kann aufs Meer ausgegossen und entzündet werden (mit einem Geldschein oder einen Stück Kleidung und Streichhölzern).
		\item Wasser. Um nicht zu verdursten.
		\item Nahrungsration. Diese besteht nur aus Grundnahrungsmitteln und kann notfalls über mehrere Tage gestreckt werden.
		\item Plastikfolie. Damit kann man Regenwasser und Tau sammeln sowie sich gegen Unwetter schützen.
		\item Schokolade. Als Reservenahrung.
		\item Angel und Zubehör. Da es nicht sicher ist, ob man hier damit Fische fangen kann, ist die Schokolade wichtiger.
		\item Nylonschnur. Um bei einem Sturm wichtige Dinge festzubinden.
		\item Aufblasbares Kopfkissen. Als Schwimmhilfe/Rettungsring, wenn jemand ins Wasser gefallen ist.
		\item Haifisch-Abwehr-Flüssigkeit. Bringt nur etwas, wenn man ins Wasser geht.
		\item Cognac. Zum Desinfizieren von Wunden. Als Getränk ist Cognac in dieser Situation nicht geeignet, weil er die Poren öffnet (Wasserverlust) und durstig macht.
		\item FM-Transistorradio. Bringt nichts, da wegen der Erdkrümmung die Empfangsreichweite maximal 30~Kilometer beträgt und das Festland zu weit entfernt ist.
		\item Karte von indischen Ozean. Bringt nichts, da die Schiffbrüchigen weder ihre eigene Position genau bestimmen noch sich aus eigener Kraft landwärts fortbewegen können.
		\item Moskitonetz. So weit von Land entfernt gibt es keine Mücken. Zum Fischen ist das Netz auch nicht geeignet.
		\item Sextant. Ist ohne Chronometer und Tabellen relativ wertlos, weil man ihn dann für die Positionsbestimmung nicht einsetzen kann.
	\end{enumerate}
Dies ist nicht die \emph{einzig wahre} Lösung. Ziel des Spieles ist nicht ein möglichst gutes Ergebnis, sondern die verschiedenen Wege dort hin und die Erfahrungen und Erkenntnisse dabei.

\paragraph{Besondere Hinweise:} \emph{Bitte macht unbedingt die Auswertung!} Ohne Auswertung ist dieses Spiel wertlos! Die Ergebnisse sollten ähnlich wie beim NASA-Spiel (Seite~\pageref{nasa}) sein.

An dieses Spiel lässt sich direkt \emph{Insel ohne Wiederkehr} anschließen.

\section{Insel ohne Wiederkehr}
\index{Insel ohne Wiederkehr}
\index{Entscheidung unter Unsicherheit|see{Insel ohne Wiederkehr}}
\index{Inselspiel|see{Insel ohne Wiederkehr}}
\paragraph{Alias:} Inselspiel
\paragraph{Art:} Teamspiel
\paragraph{Ziel:} die gruppendynamischen Prozesse bei Entscheidung unter Unsicherheit verdeutlichen
\paragraph{Dauer:} 45--60 Minuten inklusive Auswertung
\paragraph{Wir brauchen dazu:} pro Person eine Kopie der Spielanleitung (Seite~\pageref{wiederkehr-kopien}) sowie der Karte (Seite~\pageref{wiederkehr-karte})
\paragraph{So geht es:} Jede Spielerin bekommt beide Kopien ausgehändigt. Die Spielleiterin erklärt kurz die Situation (siehe Seite~\pageref{wiederkehr-kopien}). Jetzt soll die gesamte Gruppe innerhalb von 30~Minuten eine Entscheidung fällen (keine einzelnen Liste, keine Delegiertenentscheidung).

Die Auswertung erfolgt, sobald die Gruppe eine Lösung gefunden hat oder die Zeit um ist.

Bei der Auswertung können unter anderem möglicherweise herauskommen:
\begin{itemize}
  \item Es gibt Spielerinnen, die lieber sicher überleben und dafür auf die Chance verzichten, in die Zivilisation zurückzukehren. Andere riskieren lieber ihr Leben, anstatt für immer auf ein Leben in der Zivilisation zu verzichten.
  \item Entscheidungsprozesse unter Unsicherheit (Werden wir auf der Luft- und Schifffahrtslinie gesehen werden?) sind extrem schwierig, weil man sich nicht auf sichere Fakten einigen kann.
  \item Es gibt Ausgrenzungs- und Abspaltungsprozesse innerhalb der Gruppe.
  \item Vielen Menschen ist das eigene Überleben sehr wichtig~-- wichtiger als das Überleben der anderen. Dafür setzen sie auch unfaire Diskussionsmittel ein.
\end{itemize}

Über Auswertungsergebnisse von stattgefundenen Spielen würde ich mich freuen, da ich die Ergebnisse meiner eigenen Runde leider nicht mehr habe. \url{oliver@spielereader.org}

\paragraph{Besondere Hinweise:} Auch bei diesem Spiel ist die Auswertung extrem wichtig!

Dieses Spiel habe ich von Daniel Butscher von vfh.
\paragraph{Wann einsetzen:} Im Anschluss an \emph{Schiffbruch} (Seite~\pageref{schiffbruch})~-- wenn die Themen \emph{Gruppendynamik} und \emph{Entscheidung unter Unsicherheit} behandelt werden sollen.


\section{Zwei Euro}
\index{Zwei Euro}
\index{Fünf Mark|see{Zwei Euro}}
\paragraph{Alias:} Fünf Mark
\paragraph{Art:} Teamspiel
\paragraph{Ziel:} eine auf den ersten Blick unmögliche Aufgabe im Team lösen (wie Mathe-Aufgaben halt)
\paragraph{Dauer:} 10--20 Minuten
\paragraph{Wir brauchen dazu:} pro Team ein 2-Euro-Stück
\paragraph{So geht es:} Die Gruppe wird in Teams von drei bis fünf Mitgliedern aufgeteilt. Jedes Team soll möglichst schnell und möglichst genau das Gewicht eines 2-Euro-Stücks bestimmen. An "`Werkzeug"' steht dafür alles zur Verfügung, was sich im Seminarraum befindet (Also besser diesen Spiele-Reader verstecken!). Der Raum darf während der Übung nicht verlassen werden.

Das Team, das das Gewicht zuerst (und möglichst genau) angeben kann, hat das Spiel gewonnen.

\fett{Achtung Lösung:} \textsc{Ein 2-Euro-Stück wiegt 8,5~Gramm.}

\section{Brücke bauen}
\index{Brücke bauen}
\paragraph{Art:} Teamspiel
\paragraph{Ziel:} im Team eine schwierige Aufgabe lösen
\paragraph{Dauer:} 5--15 Minuten
\paragraph{Wir brauchen dazu:} insgesamt 1 Schere und 1 Rolle Krepp-Klebeband, pro Team ein paar Blatt Papier und 2 Tische
\paragraph{So geht es:} Die Gruppe wird in Teams von drei bis fünf Mitgliedern aufgeteilt. Jedes Team soll möglichst schnell aus Papier und Klebeband eine tragfähige Brücke zwischen zwei Tischen bauen (je weiter auseinander, desto schwieriger~-- 50~cm sind ok). Die Brücke soll die Schere als Belastung aushalten können. Die Teams dürfen dabei die Schere zum Testen benutzen.

\section{Schere-Spiel}
\index{Schere-Spiel}
\index{Flaschenspiel|see{Schere-Spiel}}
\label{flaschenspiel}
\paragraph{Alias:} Flaschenspiel
\paragraph{Art:} Kommunikationsspiel mit "`Eingeweihten"'
\paragraph{Ziel:} erkennen, das es nicht auf das Vordergründige ankommt
\paragraph{Dauer:} 10--30 Minuten
\paragraph{Wir brauchen dazu:} Sitzkreis, 1 Schere (für das Schere-Spiel) bzw.~1 Flasche mit Deckel, ein paar Teilis, die das Spiel noch nicht kennen
\paragraph{So geht es:} Beim Schere-Spiel wird eine Schere im Kreis herumgegeben. Dabei ist es egal, \emph{wie} die Schere weitergegeben wird (also offen, geschlossen, mit dem Griff nach vorne\footnote{Das ist höchstens in Bezug auf die Verletzungsgefahr relevant.} oder nach hinten), aber die Spielerin sagt beim Weitergeben "`offen"' oder "`geschlossen"'.

Wenn eine Spielerin das Muster erkannt hat, sollte sie es nicht laut sagen, sondern zuerst durch weitere Versuche zu bestätigen versuchen. Danach sollte sie den Mund halten, damit sie den anderen den Spaß nicht verdirbt. Das Spiel geht so lange, bis alle das Schema herausbekommen haben.

\fett{Achtung Lösung:} \textsc{"`Offen"' sagt die Tutorin, wenn die entgegennehmende Spielerin die Beine offen (also nicht gekreuzt) hält, und "`geschlossen"' entsprechend umgekehrt.}

\paragraph{Wann einsetzen:} Einfach so zum Spaß in geselliger Runde. Oder um zum Thema \emph{Kommunikation} deutlich zu machen, dass es bei einer Mitteilung nicht nur auf das vordergründig Offensichtliche ankommt.
\paragraph{Varianten:} \begin{itemize}
		\item Beim \emph{Flaschenspiel} wird eine Flasche mit Schraubverschluss im Kreis herumgegeben.
		\item \textsc{\emph{Offen/geschlossen} kann sich auch auf den Mund der weitergebenden Spielerin beziehen.}
	\end{itemize}

\section{Mörder, äh, Mörderin}
\index{Mörder}
\index{Mörderin|see{Mörder}}
\paragraph{Art:} sehr paranoides Spiel für nebenher
\paragraph{Ziel:} alle bringen sich nach und nach gegenseitig (symbolisch) um (und lernen dabei die Namen)
\paragraph{Dauer:} nebenher während der gesamten Veranstaltung, einen bis mehrere Tage lang
\paragraph{Wir brauchen dazu:} Loszettel mit den Namen aller Spielerinnen
\paragraph{So geht es:} Am Anfang der Veranstaltung werden die Zettel gemischt und an die Spielerinnen verteilt. Dabei erhält jede den Zettel mit dem Namen ihres zukünftigen Opfers:
	\begin{itemize}
		\item zufällig durch Ziehen, wenn niemand Externes da ist, die die Zettel verteilen kann, oder
		\item jemand, die nicht mitspielt, verteilt die Zettel so, dass sich aus allen Mörderinnen-Opfer-Beziehungen ein Kreis bildet (damit genau dann das eigene Opfer die eigene Mörderin ist, wenn nur noch zwei Spielerinnen überlebt haben)
	\end{itemize}
Dann wird ein Zettel mit den Namen aller Spielerinnen aufgehängt. Auf Zeichen beginnt das Spiel.

Ein Mord geht folgendermaßen vor sich:
	\begin{enumerate}
		\item Die Mörderin trifft ihr Opfer, das auf ihrem Zettel steht.
		\item Es dürfen keine (lebenden) Zeuginnen anwesend sein. Wenn beim Mord eine Zeugin anwesend ist, den Mord sieht und dies sofort sagt, dann ist der Mord ungültig.
		\item Die Mörderin gibt dem Opfer einen beliebigen Gegenstand, zum Beispiel eine Flasche oder einen Stift. Wenn das Opfer den Gegenstand freiwillig annimmt, ist es gemordet.
		\item Die Mörderin teilt dem Opfer mit, dass es nun tot ist.
		\item Das Opfer gibt den Zettel mit dem Namen seines Opfers der Mörderin. So erhält die Mörderin den Namen ihres nächsten Opfers.\footnote{Auf diese Weise entstehen Massenmörderinnen.}
		\item Das Opfer trägt auf der aushängenden Namensliste ein Kreuz neben dem Namen der Verblichenen ein sowie die Todeszeit und -art, wenn sie mag ("`Mit einem Grillspieß ermordet."', "`War beim Kiffen zu gierig."' oder so).
		\item Man darf nicht dazu lügen, ob man noch lebt oder schon tot ist (Auslassungslügen sind natürlich erlaubt). Tote dürfen noch Lebende außerdem nicht entlarven.
	\end{enumerate}
Wer überleben möchte, sollte deswegen keine Gegenstände direkt annehmen oder sicherstellen, dass eine lebende Zeugin zuschaut.

Das Spiel ist zu Ende, wenn beim Showdown der letzten beiden überlebenden Spielerinnen die eine die andere erfolgreich ermordet.
\paragraph{Besondere Hinweise:} Das Spiel läuft während des Seminars oder der Fahrt so nebenher. Es kann allerdings zu einer handfesten Paranoia führen und im Extremfall eine konstruktive Zusammenarbeit während des Seminars unmöglich machen. Es kann aber auch einen Heidenspaß machen und sehr spannend sein!
\paragraph{Varianten:}\begin{itemize}
		\item Der Mord geschieht durch ein einfaches "`Du bist tot!"'. Das hat aber den Nachteil, dass sich das Opfer nicht wehren kann.
		\item Der Mord geschieht durch einen symbolischen Mordakt, zum Beispiel:
			\begin{itemize}
				\item mit einer Wasserpistole erschießen
				\item mit Tabasco im Bier vergiften
				\item durch eine Zeitbombe zerreißen (mit einer Fotoapparat-Blitz\-licht\-birne als Sprengladung)
			\end{itemize}
		Damit niemand verletzt wird, darf niemand beim Autofahren ermordet werden (für den Mord kurz rechts heranfahren ist selbstverständlich in Ordnung). Eine echte Körperverletzung muss auch ausgeschlossen sein.
		
		Dabei kann es durchaus zu sehr schönen Mafiamorden kommen, wie etwa jemandem nachts an der Haustür mit der Wassermaschinenpistole aufzulauern.
	\end{itemize}
\paragraph{Wann einsetzen:} Bei Seminaren und längeren Veranstaltungen, wenn die Anwesenden Spaß an so umfangreichen Spielen haben.


\section{Nacht in Palermo}
\index{Nacht in Palermo}
\index{Palermo, Nacht in}
\index{Mafia|see{Nacht in Palermo}}
\index{Mörder|see{Nacht in Palermo}}
\index{Werwolf|see{Nacht in Palermo}}
\index{Werwölfe von Thiercelieux|see{Nacht in Palermo}}
\paragraph{Alias:} Mafia, Mörder, Werwolf, Werwölfe von Thiercelieux
\paragraph{Art:} Mörderinnenspiel mit extremem Psychofaktor
\paragraph{Ziel:} die Gruppe muss zwei Mörderinnen finden und hinrichten lassen
\paragraph{Dauer:} 15--45 Minuten
\paragraph{Wir brauchen dazu:} Sitzkreis mit mindestens 10 Teilis, Zettel als Lose, Stift, möglichst wenig Licht (optional)
\paragraph{So geht es:} Eine Spielerin ist die Spielleiterin. Die anderen ziehen Lose (oder Spielkarten), um ihre Rollen zugeteilt zu bekommen:
\begin{description}
	\item [2~Mörderinnen:] Zettel mit \emph{M}
	\item [1~Detektivin:] Zettel mit \emph{D}
	\item [Bürgerinnen:] leere Zettel für den Rest
\end{description}

Alle sitzen so im Kreis, dass jede jede sehen kann. Die Spielleiterin fängt an zu erzählen:
\begin{quotation}
	Es wird Nacht in Palermo. Die Bürgerinnen gehen schlafen. \emph{Alle Spielerinnen schließen die Augen.}
	
	Doch zwei Mörderinnen erwachen und suchen ein Opfer. \emph{Die beiden Mörderinnen erwachen und verständigen sich durch Blicke oder andere Zeichen auf ein Opfer und teilen der Spielleiterin das Opfer mit.} Die Mörderinnen schlafen wieder ein. \emph{Die Mörderinnen schließen wieder die Augen.}
	
	Die Detektivin erwacht. \emph{Die Detektivin öffnet die Augen. Sie deutet für die Spielleiterin auf eine Bürgerin. Die Spielleiterin signalisiert, ob dies eine Mörderin oder eine Bürgerin ist.} Die Detektivin schläft wieder ein. \emph{Die Detektivin schließt wieder die Augen.}
	
	Es ist Morgen. Die Bürgerinnen von Palermo erwachen~-- bis auf eine. \emph{Die Spielleiterin teilt dem Opfer mit, dass es jetzt tot ist.}
\end{quotation}
	
Die Bürgerinnen diskutieren jetzt und entscheiden, welche beiden aus ihrer Mitte sie als verdächtigte Mörderinnen auf die Anklagebank setzen. Diese beiden Bürgerinnen halten dann je eine \emph{kurze} Verteidigungsrede.

Die Bürgerinnen entscheiden danach, wen sie hinrichten. Die Hingerichtete muss danach mitteilen, ob sie eine Mörderin war oder nicht.

In der nächsten Nacht nehmen das Mordopfer und die Hingerichtete nur noch beobachtend am Spiel teil. Sie brauchen die Augen nicht zu schließen, dürfen aber auch nicht reden oder den Lebenden irgendwelche Zeichen geben.

Das Spiel geht so lange, bis beide Mörderinnen hingerichtet sind oder alle Nicht-Mörderinnen ermordet sind. Danach kann neu gelost werden.

Hier ein paar unverfälschte Zitate vom DPO-Freak auf einer OE-Vorbereitungsfahrt:
\begin{itemize}
  \item "`Hängt ihn höher!"'
  \item "`Bombenleger!"'
  \item "`Randgruppen zuerst!"'
  \item "`Das war unser letztes Weibchen!"'
\end{itemize} 

\paragraph{Besondere Hinweise:} Das Spiel kann sehr heftig werden: Gegenseitige Beschuldigungen, Sündenböcke, Verschwörungen, Intrigen. Mit Demokratie oder Konsens hat das nichts mehr zu tun.
\paragraph{Wann einsetzen:} Abends beim Bier oder so.
\paragraph{Variante:} Die Werwölfe von Thiercelieux

Unter \url{http://www.mafiaspiel.de/} ist eine umfangreiche Website zu diesem Spiel und seinen Varianten zu finden. Außerdem veranstaltet die EMSA Bonn (\url{http://www.emsa-bonn.de}) regelmäßig Mafia-Runden.

\section{Kommunikationsmetaphern}
\index{Kommunikationsmetaphern}
\index{Metaphern, Kommunikations-}
\paragraph{Art:} Selbsterfahrung durch Herumlaufen und Übergabe von Gegenständen
\paragraph{Ziel:} die Grundprinzipien der Kommunikation verdeutlichen
\paragraph{Dauer:} 20--30 Minuten
\paragraph{Wir brauchen dazu:} $\geq 8$ Teilis, 1--2 Jonglierbälle und viele andere Gegenstände, die sich werfen oder übergeben lassen (ich hatte neulich einen Igelball, ein großes Messer, Spielknete, einen Plüsch-Elch, ein Gedicht hinter Glas mit Bilderrahmen, ein Trinkglas und eine Frisbee)
\paragraph{So geht es:} Das Spiel unterteilt sich in fünf Runden. Am Anfang/Ende jeder Runde trifft sich die Gruppe zum Stehkreis, um die letzte Runde auszuwerten und die Anleitung für die nächste Runde zu bekommen. Die Tutorin kann auch mitspielen.

Die fünf Runden:
  \begin{enumerate}
    \item Jede suchen sich eine gute Stelle im Raum und stellt sich dort hin. Dort verhält sie sich ruhig und nimmt ihre Umgebung wahr. Sie kann die Augen dabei offen oder geschlossen halten.
    \item Alle laufen kreuz und quer herum und schauen sich den Raum bzw.~die Umwelt an.
    \item Die gleichen Instruktionen mit dem Unterschied, dass die Tutorin ankündigt, in dieser Runde etwas ändern zu wollen. 1--2 Minuten nach dem Start gibt sie einer Teilnehmerin einen Jonglierball in die Hand und schaut, was passiert.
    \item Genauso, nur dass die Tutorin jetzt auch die anderen Gegenstände verteilt.
    \item Alle bleiben im Kreis stehen und geben dort die Gegenstände weiter.
  \end{enumerate}
  Bei der Auswertung nach jeder Runde kann die Tutorin etwa folgende Fragen stellen:
  \begin{itemize}
    \item Was habe ich mich, die anderen und die Umwelt wahrgenommen? Wie ging es mir dabei?
    \item Auf welche Kriterien habe ich geachtet?
    \item Was passiert, wenn sich zwei Spielerinnen begegnen?
    \item Was wird gebraucht, damit die Übergabe eines Gegenstandes funktioniert?
    \item Warum und wie ist die Übergabe bei den verschiedenen Gegenständen verschieden?
  \end{itemize}
  Bei der Auswertung kann (oder sollte?) herauskommen, dass die Gegenstände Metaphern für Nachrichten in der Kommunikation darstellen. Blickkontakt ist bei der Übergabe wichtig. Verschiedene Gegenstände sind verschieden beliebt, werden verschieden weitergegeben und regen unterschiedlich zur Kreativität an.
\paragraph{Besondere Hinweise:} Dieses Spiel ist eventuell problematisch bei Gruppen, die auf alles allergisch reagieren, was nach Selbsterfahrung aussieht.

Soweit ich weiß, stammt dieses Spiel von Ansgar Kemmann vom v.\,f.\,h.
\paragraph{Wann einsetzen:} am Anfang von Rhetorik- oder Kommunikationsseminaren oder bei der OE-Vorbereitungsfahrt

\section{Trotzburg}
\index{Trotzburg}
\index{Belagerte Stadt|see{Trotzburg}}
\paragraph{Alias:} Die belagerte Stadt
\paragraph{Art:} heftiges Rollenspiel um eine belagerte mittelalterliche Stadt
\paragraph{Ziel:} die Teilnehmerinnen für gruppendynamische Prozesse, Entscheidungsfindung in Gruppen, Sündenbockmechanismen und den Umgang mit Konflikten sensibilisieren
\paragraph{Dauer:} 1,5--2 Stunden (10 Minuten Vorbereitung, 1 Stunde Spiel, 20--50 Minuten Auswertung)
\paragraph{Wir brauchen dazu:} Stuhlhalbkreis (Podium) mit 5 Plätzen, Kopien der Rollenbeschreibungen (ab Seite \pageref{trotzburg-rollen}) für die 5 Spielerinnen, Kopien der Auswertungsbögen für die Beobachterinnen  (Seite \pageref{trotzburg-auswertung})
\paragraph{So geht es:}
Fünf Spielerinnen aus der Gruppe erhalten~-- am besten einige Zeit vor dem Spiel~-- ihre Rollenbeschreibungen. Sie spielen fünf Bürger einer mittelalterlichen Stadt: Bürgermeister, Arzt, Krankenpfleger, Wächter und Schmied.

Die anderen Spielerinnen übernehmen im folgenden Spiel eher passive Rollen: Sie beobachten und dokumentieren den Spielverlauf.

Die fünf Spielerinnen werden angewiesen, ihre Rollen genau zu studieren und die Informationen auf keinen Fall einer anderen Person mitzuteilen.

Die Spielleitung sollte in Rollenspielen unerfahrene Spielerinnen darauf hinweisen, dass sie versuchen sollen, sich gut in die beschriebene Rolle einzufühlen und im Spiel stimmig zu agieren. Sich in die Rolle einzufühlen bedeutet auch, die Rolle entsprechend auszubauen und sich zu überlegen, welche Informationen von seiner Rolle man im Spiel sofort, später oder überhaupt nicht preisgibt.

Die Spielleitung teilt der Gruppe kurz das Szenario mit:
\begin{quote}
Die mittelalterliche Stadt Trotzburg wird von den Hochbergern belagert. Sie beschuldigen die Trotzburger, einen Kaufmann umgebracht zu haben und fordern innerhalb einer Stunde die Auslieferung der schuldigen Person.
\end{quote}

Die Spielleitung kann der Gruppe auch gleich zu Beginn den Zweck des Spieles erläutern:
Es geht darum, Verhaltensweisen zu studieren, die in jeder Gruppe oder Gesellschaft ablaufen können, die aber in einer ausweglosen Situation wie in diesem Spiel besonders deutlich hervortreten. Daher sollten die Zuschauerinnen das Verhalten der Spielerinnen beobachten und auf den Auswertungsbögen dokumentieren (Seite~\pageref{trotzburg-auswertung}).

Sinnvoll ist es auch, einzelnen Zuschauerinnen bestimmte Spielerinnen zuzuordnen, so dass nicht alle alles beobachten.

Für die Spielerinnen wird eine Art kleiner Bühne vorbereitet, die zu den Zuschauerinnen hin offen ist, so dass der Spielverlauf gut beobachtet werden kann.

Nachdem die Vorbereitungen abgeschlossen sind, wird die Beratung der fünf Beteiligten gespielt, in der entschieden werden muss, wer den Hochbergern als Schuldiger ausgeliefert werden soll.

Es wird relativ schnell passieren, dass den fünf Bürger die Idee kommt, mit den Hochbergern zu reden. Wenn die Spielerinnen im Spiel auch weitgehend frei in ihren Entscheidungen ist: Die Forderung der Hochberger ist klar ausgesprochen, und sie werden auch nicht davon abrücken.
Ebenso werden sie nach genau einer Stunde die Stadt stürmen und niederbrennen, wenn nicht ein Schuldiger ausgeliefert wird.

Auch eine Verteidigung ist für das arme Städtchen Trotzburg nicht denkbar oder sinnvoll.

\paragraph{Besondere Hinweise:} Wie andere Rollenspiele mit einer starken Dynamik kann auch \emph{Trotzburg} weit über den Rahmen des Spiels hinausgehen und zu Wut, Ohnmachtsgefühlen o.\,Ä.~übergehen. Die Spielleitung sollte auf jeden Fall deutlich machen, dass die gespielte Entwicklung der Beratung nicht von speziellen Personen abhängig ist, sondern in allen Gruppen so ähnlich ablaufen wird.

Dieses Spiel hat die Form eines Entscheidungsspiels, ist aber der Sache nach eine Experiment, und man sollte es auch so vor der Gruppe bezeichnen. Denn die Situation ist so konstruiert, dass eine Entscheidung gegen den Sündenbock-Mechanismus kaum mehr möglich ist. Denkbar wäre höchstens, dass alle fünf sich stellen~-- was praktisch nie vorkommt, weil mindestens eine sich weigert und ihre Unschuld beteuert,~-- oder dass die Auszuliefernde ausgelost wird, nachdem alle eingesehen haben, dass sie mitschuldig sind.

In der Regel wird heftig gekämpft~-- meist mit unsachlichen Argumenten~--, bis schließlich einer, bei der es am leichtesten geht, alle Schuld zugeschoben wird.

Bisher habe ich folgende brauchbare Lösungen kennen gelernt:
\begin{itemize}
	\item Die Spielerinnen erzählen alle wahrheitsgemäß ihre Geschichte und stellen dabei fest, dass alle eine Teilschuld trifft. Sie losen aus, wer ausgeliefert wird. Die restliche Zeit nutzen sie, um eine alternative Lösung zu finden.
	\item Es wird ein Gefangener aus dem Kerker ausgeliefert, der ohnehin hingerichtet werden sollte (eine sehr kreative Lösung).
\end{itemize}

\paragraph{Wann einsetzen:} um die Teilnehmerinnen für Kommunikation, Entscheidungsfindung und Gruppenverhalten zu sensibilisieren

\section{Die heimliche Freundin}
\index{Heimliche Freundin}
\index{Freundin, Heimliche}
\paragraph{Art:} Nebenher-Spiel, bei dem jede Teilnehmerin eine heimliche Freundin hat, die ihr während der Veranstaltung immer wieder heimlich etwas Gutes tut
\paragraph{Ziel:} eine positive und kooperative Stimmung während der Veranstaltung schaffen, das Kennenlernen fördern, die Gruppe näher zusammenbringen
\paragraph{Dauer:} nebenher während der gesamten Veranstaltung, einen bis mehrere Tage lang
\paragraph{Wir brauchen dazu:}  Loszettel mit den Namen aller Spielerinnen, eine Tüte, Schüssel o.\,Ä.~für die Lose
\paragraph{So geht es:}
\begin{description}
\item[Die Vorbereitung:] Am ersten Tag der Veranstaltung werden Zettel mit den Namen aller Teilnehmerinnen in eine Tüte o.\,Ä.~geworfen. Danach zieht jede einen Zettel. Sollte jemand den eigenen Namen ziehen, wird die ganze Prozedur wiederholt.

\item[Das Spiel:] Danach kann das Spiel sofort losgehen: Jede Spielerin hat bis zum Ende der Veranstaltung die Aufgabe, der Spielerin, deren Namen sie gezogen hat, jeden Tag mindestens einmal etwas Gutes zu tun. Was das ist, bleibt ganz der Fantasie der Spielerinnen überlassen: Blumen auf dem Tisch vor dem Platz der Beglückten, eine Tafel Schokolade, ein Liebesbrief, ein frisch gemachtes Bett \ldots

Wichtig ist, dass die Beschenkte vor Spielende die Identität ihrer heimlichen Freundin nicht erfahren soll. Wer etwas Gutes tut, tut dies also heimlich. Natürlich ist es den Spielerinnen trotzdem erlaubt, zu knobeln und zu forschen, wer ihre heimliche Freundin sein könnte.

\item[Die Auflösung:] Zum Ende der Veranstaltung wird das Spiel aufgelöst. Dafür bieten sich beispielsweise an:
	\begin{description}
		\item[Die Party:] Bei tanzbarer Musik geben sich alle heimlichem Freundinnen der entsprechenden Beschenkten zu erkennen.
		\item[Der heiße Stuhl:] Nacheinander wird jede Spielerin mit verbundenen Augen auf einen Stuhl in der Mitte des Raumes gesetzt. Jetzt tut ihr ihre heimliche Freundin noch ein letztes Mal etwas Gutes. Sie tut es so, dass die Beschenkte dabei die Chance hat, die Identität ihrer heimlichen Freundin herauszufinden.
	\end{description}
\end{description}

\paragraph{Besondere Hinweise:} Mit Erwachsenen funktioniert dieses Spiel in der Regel sehr gut.

Das Spiel funktioniert nicht, wenn die Stimmung in der Gruppe sehr schlecht ist oder zwischen einzelnen Spielerinnen eine große Abneigung besteht.
\paragraph{Wann einsetzen:} Um bestehende Gruppen näher zusammenzubringen. Und um (auch bei neu zusammengekommenen Gruppen) für die Veranstaltung eine sehr angenehme Atmosphäre zu schaffen.

\section{Kreisflucht}
\index{Kreisflucht}
\paragraph{Art:} Errate-das-Spiel mit Moral zum Thema Kommunikation und Querdenken.
\paragraph{Ziel:} aus einem Kreis von Leuten entkommen
\paragraph{Dauer:} 5 Minuten.
\paragraph{Wir brauchen dazu:} mindestens 5 Teilis und Platz für einen Kreis.
\paragraph{So geht es:} Eine Teilnehmerin, die das Spiel noch nicht kennt, verlässt den Raum. Die Spielleiterin kann jetzt das Spiel erklären, bevor die Gruppe die Spielerin wieder hereinrufe: Die Spielerinnen halten sich an den Händen und bilden einen Kreis um die eine Spielerin. Diese versucht nun, aus dem Kreis zu entkommen. Die Gruppe hat den Auftrag, die Spielerin auf keinen Fall durch zu lassen.

\fett{Achtung Lösung:} \textsc{Die Gruppe lässt die Spielerin nur dann durch, wenn sie (mit Worten) darum bittet.}
\paragraph{Besondere Hinweise:} Funktioniert pro Gruppe nur einmal.

Ich habe noch nicht ausprobiert, was passiert, wenn die Spielerin Judo oder Jiu Jitsu kann.
\paragraph{Wann einsetzen:} Einfach so (zum Lachen) oder als Einleitung für eine Arbeitseinheit zu expliziter Kommunikation.


\section{Gummibärchenanalyse}
\index{Gummibärchenanalyse}
\paragraph{Art:} eine Gruppe mit Gummibärchen, Playmobil oder Lego nachstellen
\paragraph{Ziel:} Rollen und Beziehungen in einer Gruppe reflektieren und explizit machen
\paragraph{Dauer:} 45--60 Minuten
\paragraph{Wir brauchen dazu:} pro Kleingruppe (2--4 Leute) ein leeres Moderationsplakat und einen Satz Moderationsstifte; Platz auf dem Boden oder auf Tischen, so dass jede Kleingruppe ihr Plakat zum Arbeiten ausbreiten kann; einen großen Haufen Playmobil (Figure, Pferde, Waffen, Möbel \ldots) oder Lego oder eine \emph{sehr} breite Auswahl an verschiedenen Haribo-Figuren; Kleber zum Festkleben der Gummibärchen (nicht bei Lego oder Playmobil!)
\paragraph{So geht es:} Die Teilis bilden Kleingruppen à 2--4 Teilnehmerinnen. Auf einem Moderationsplakat baut nun jede Kleingruppe die Gruppe nach, deren Beziehungen und Rollen sie darstellen möchte, zum Beispiel: Der König (Vorsitzende) sitzt auf seinem Thron, hinter seinem Rücken stehen zwei andere Figuren und tuscheln \ldots

Mit den Moderationsstiften lassen sich noch Pfeile, Trennlinien, Namen und Ähnliches aufmalen, um die Gruppenstruktur noch deutlicher zu machen.

Wenn alle Kleingruppe fertig sind (hier schadet ein bisschen Zeitdruck von der Moderatorin nicht), stellt jede Kleingruppe ihr Kunstwerk dem Plenum vor. Applaus!
\paragraph{Besondere Hinweise:} Wenn sich das Plenum aus verschiedenen Organisationen zusammensetzt (zum Beispiel wenn Leute aus einer Fachschaft und vom NaBu da sind), sollten sich die Kleingruppen möglichst nach den Gruppen sortieren, deren Struktur nachgebaut werden soll.

Wenn viele Leute aus einer Gruppe da sind, können die Kleingruppen auch die gleiche Gruppe darstellen, so dass dabei unterschiedliche Sichtweisen und Darstellungen herauskommen.

Diese Methode funktioniert möglicherweise nicht, wenn der fiese Chef oder ein extrem unbeliebtes Gruppenmitglied anwesend ist und die Teilis dann ihre ehrliche Sicht der Gruppe nicht darstellen möchten.
\paragraph{Wann einsetzen:} um eine Gruppe spielerisch über ihre eigenen Strukturen, Rollen und Beziehungen reflektieren zu lassen

