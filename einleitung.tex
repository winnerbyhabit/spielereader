\chapter{Einleitung}
\section{Willkommen zum Spiele- und Methodenreader!}

Wenn ihr auf einem Seminar spielen wollt, dann tut das von Anfang an. Wenn ihr erst später mit Spielen anfangt, sind die Teilnehmerinnen oft nur noch schwer dazu zu bewegen, zu laufen oder alberne Dinge zu tun. Diskutiert außerdem nicht, ob gespielt wird oder nicht, sondern verbreitet einfach Enthusiasmus und Motivation und reißt die TeilnehmerInnen mit. Wer nicht mitspielen möchte, kann trotzdem nicht gezwungen werden.

Spielt mit den TeilnehmerInnen möglichst nur die Spiele, die ihr schon einmal ausprobiert habt. Dann klappt's mit dem Erklären besser, und ihr kennt den Ablauf schon. Ausnahmen bestätigen wie üblich die Regel (wenn ihr schon sehr viel Erfahrung mit Spielen und Seminaren habt oder wenn ihr einfach neugierig seid).

Wenn ihr ein Spiel für ein Seminar plant, sucht euch auch noch ein, zwei Reservespiele aus, falls ihr spontan feststellt, das das Spiel doch nicht ganz in die entsprechende Situation passt. (Das passiert erstaunlich oft.)

Spielt viel und oft! Auf der OE-Vorbereitungsfahrt 1999 fanden die Leute die meisten Spiele sehr spaßig. Und falls noch ihr zweifeln solltet, ob eure TeilnehmerInnen wirklich mitspielen werden: Auf einem Java-Seminar habe ich sogar einen Haufen spießige BankerInnen zum Spielen bewegen können.

\emph{Verdammt, das ist ja fürchterlich. Ich schmeiße jetzt alle geschlechtsneutralen Doppelformen raus und nehme an den üblen Stellen die weibliche Form. Männchen und Weibchen mögen sich bitte gleichermaßen angesprochen fühlen.}

\section{Wo kommen die Spiele und Methoden her?}
Die meisten Spiele und Methoden habe ich auf folgenden Veranstaltungen kennen gelernt:
\begin{itemize}
  \item Seminare des \emph{Vereins zur Förderung politischen Handelns (v.\,f.\,h.)}, \url{http://www.vfh-online.de/}
  \item Seminare der \emph{Werkstatt für Gewaltfreie Aktion, Baden}, \url{http://www.wfga.de/}
  \item Konferenz der deutschsprachigen Informatikfachschaften (KIF), \url{http://kif.fsinf.de/}
  \item Seminare, die ich mit anderen Menschen zusammen geleitet habe
\end{itemize}

\section{Unter welchen Bedingungen könnt ihr den Spielereader benutzen?}
Dieser Reader ist unter einer \emph{Creative-Commons}-Lizenz lizensiert, und zwar unter der \emph{Namensnennung-Weitergabe unter gleichen Bedingungen~3.0 Deutschland}. Das bedeutet, dass ihr den Reader unter diesen Bedingungen für euch kostenlos verbreiten, bearbeiten und nutzen könnt (auch kommerziell):
\begin{description}
  \item[Namensnennung.] Ihr müsst den Namen des Autors (Oliver Klee) nennen. Wenn ihr außerdem auch noch die Quelle (also \url{http://www.spielereader.org/}) nennt, wäre das nett. Und wenn ihr mir zusätzlich eine Freude machen möchtet, sagt mir per E-Mail Bescheid.
  \item[Weitergabe unter gleichen Bedingungen.] Wenn ihr diesen Inhalt bearbeitet oder in anderer Weise umgestaltet, verändert oder als Grundlage für einen anderen Inhalt verwendet, dann dürft ihr den neu entstandenen Inhalt nur unter Verwendung identischer Lizenzbedingungen weitergeben.
  \item Wenn ihr den Reader weiter verbreitet, müsst ihr dabei auch die Lizenzbedingungen nennen oder beifügen.
\end{description}

Die ausführliche Version dieser Lizenz findet ihr unter \url{http://creativecommons.org/licenses/by-sa/3.0/de/}.

\section{Wie könnt ihr Lob und Kritik loswerden?}
Wenn ihr Lob, Kritik, Korrekturen\footnote{bitte auch zu Tippfehlern, Sprachgurken und inhaltlichen Fehlern}, Anregungen oder sonstige Kommentare habt, würde ich mich über eine E-Mail an \texttt{oliver@spielereader.org} freuen.

Ich wünsche euch viel Spaß beim Spielen!
