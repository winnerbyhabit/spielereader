\chapter{Spiele zur Wissensvermittlung}
\index{Wissensvermittlung}

\section{Gruppenquiz}
\index{Gruppenquiz}
\index{Quiz|see{Gruppenquiz}}
\paragraph{Art:} Quizspiel mit konkurrierenden Gruppen
\paragraph{Ziel:} Die Gruppen überlegen sich Fragen, die die anderen Gruppen beantworten müssen.
\paragraph{Dauer:} 30--45 Minuten
\paragraph{Wir brauchen dazu:} Flipchart oder Packpapier, Moderationskarten, Moderationsstifte, Uhr mit Sekundenanzeige oder Stoppuhr
\paragraph{So geht es:} Die Teilis teilen sich in 3--5 gleich große Gruppen auf, die sich Namen geben (oder welche bekommen), z.\,B.~\emph{Hennen, Eier, Hämmer} und \emph{Krokodile.} Jede Gruppe überlegt sich Fragen zu einem vorgegebenen Thema (eine pro Gruppenmitglied) und schreibt diese auf Moderationskarten. Jede aus der Gruppe muss die Frage beantworten können. Die Fragen sollen interessant, aber nicht zu einfach sein.

In der Zwischenzeit erstellt die Tutorin eine Punktetabelle auf Flipchart oder auf Packpapier: Eine Zeile pro Runde sowie eine Spalte pro Gruppe.

Wenn alle Gruppen ihre Fragen gesammelt und diskutiert haben, geht das eigentliche Spiel los: Aus der ersten Gruppe kommt eine Teili nach vorne und liest ihre Frage vor. Jetzt haben die anderen Gruppen eine Minute Zeit, gruppenintern über eine Antwort zu diskutieren. Die Gruppe, die nach Ablauf der Zeit (oder schon vorher) glaubt die Frage beantworten zu können, ruft lauft "`Hier!"' (oder so).

Die Fragerin zeigt dann willkürlich auf jemanden aus der Hier-Gruppe, die dann die Frage beantworten muss. Befindet die Fragerin die Antwort für richtig, bekommt die antwortende Gruppe einen Punkt (Strich in der Punkteliste). War die Antwort falsch, darf eine andere Gruppe antworten. Pro Runde darf jede Gruppe nur einmal antworten.

Weiß keine Gruppe die richtige Antwort, zeigt die Tutorin auf jemanden aus der fragenden Gruppe, die dann die gewünschte Antwort sagt. Dann stimmen die anderen Gruppen darüber ab, ob sie die Frage (und die Antwort natürlich) interessant fanden. Wenn mindestens ein Viertel der Teilis aus den anderen Gruppen sich für "`interessant"' melden, bekommt die \emph{fragende} Gruppe einen Punkt pro Gruppe, die eine \emph{falsche} Antwort gesagt hat. Sonst bekommt die fragende Gruppe nichts (deswegen \emph{interessante, aber nicht zu einfache} Fragen).
\paragraph{Besondere Hinweise:} Diese Methode habe ich von Daniel Butscher vom v.\,f.\,h.~gelernt.
