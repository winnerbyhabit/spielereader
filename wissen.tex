\chapter{Spiele zur Wissensvermittlung}
\index{Wissensvermittlung}

\section{Gruppenquiz}
\index{Gruppenquiz}
\index{Quiz|see{Gruppenquiz}}
\paragraph{Art:} Quizspiel mit konkurrierenden Gruppen
\paragraph{Ziel:} Die Gruppen überlegen sich Fragen, die die anderen Gruppen beantworten müssen.
\paragraph{Dauer:} 30--45 Minuten
\paragraph{Wir brauchen dazu:} Flipchart oder Packpapier, Moderationskarten, Moderationsstifte, Uhr mit Sekundenanzeige oder Stoppuhr
\paragraph{So geht es:} Die Teilis teilen sich in 3--5 gleich große Gruppen auf, die sich Namen geben (oder welche bekommen), z.\,B.~\emph{Hennen, Eier, Hämmer} und \emph{Krokodile.} Jede Gruppe überlegt sich Fragen zu einem vorgegebenen Thema (eine pro Gruppenmitglied) und schreibt diese auf Moderationskarten. Jede aus der Gruppe muss die Frage beantworten können. Die Fragen sollen interessant, aber nicht zu einfach sein.

In der Zwischenzeit erstellt die Tutorin eine Punktetabelle auf Flipchart oder auf Packpapier: Eine Zeile pro Runde sowie eine Spalte pro Gruppe.

Wenn alle Gruppen ihre Fragen gesammelt und diskutiert haben, geht das eigentliche Spiel los: Aus der ersten Gruppe kommt eine Teili nach vorne und liest ihre Frage vor. Jetzt haben die anderen Gruppen eine Minute Zeit, gruppenintern über eine Antwort zu diskutieren. Die Gruppe, die nach Ablauf der Zeit (oder schon vorher) glaubt die Frage beantworten zu können, ruft lauft "`Hier!"' (oder so).

Die Fragerin zeigt dann willkürlich auf jemanden aus der Hier-Gruppe, die dann die Frage beantworten muss. Befindet die Fragerin die Antwort für richtig, bekommt die antwortende Gruppe einen Punkt (Strich in der Punkteliste). War die Antwort falsch, darf eine andere Gruppe antworten. Pro Runde darf jede Gruppe nur einmal antworten.

Weiß keine Gruppe die richtige Antwort, zeigt die Tutorin auf jemanden aus der fragenden Gruppe, die dann die gewünschte Antwort sagt. Dann stimmen die anderen Gruppen darüber ab, ob sie die Frage (und die Antwort natürlich) interessant fanden. Wenn mindestens ein Viertel der Teilis aus den anderen Gruppen sich für "`interessant"' melden, bekommt die \emph{fragende} Gruppe einen Punkt pro Gruppe, die eine \emph{falsche} Antwort gesagt hat. Sonst bekommt die fragende Gruppe nichts (deswegen \emph{interessante, aber nicht zu einfache} Fragen).
\paragraph{Besondere Hinweise:} Diese Methode habe ich von Daniel Butscher vom v.\,f.\,h.~gelernt.

\section{Scotland Yard}
\index{Scotland Yard}
\index{Stadtralley|see{Scotland Yard}}
\paragraph{Art:} Stadtralley nach Art des Brettspiels \emph{Scotland Yard}
\paragraph{Ziel:} die Stadt und das öffentliche Nahverkehrssystem kennen lernen
\paragraph{Dauer:} bis zu 3~Stunden
\paragraph{Wir brauchen dazu:}
\begin{itemize}
	\item Eine Möglichkeit, eine große Menge SMS zu verschicken (etwa über einen Online-Dienst wie \emph{web.de}). Bei einer Spieldauer von 3~Stunden mit 10~Gruppen macht das 120~SMS, die bezahlt werden müssen.
    \item Kopien von den Regeln auf Seite \pageref{scotland-yard-regeln} für alle Teilnehmerinnen.
	\item Schreibzeug für alle Spielerinnen
	\item mindestens ein Handy pro Gruppe und pro \emph{Mr.}
\end{itemize}

\paragraph{So geht es:} Dies sind die Aufgaben der Spielleitung:
\begin{enumerate}

\item  Vor dem Spielbeginn tragen sich alle Gruppen in eine Liste ein und geben pro Gruppe eine Handynummer an, an die die SMS
mit den Positionen der Spielerinnen versandt werden sollen.

\item Dann werden vor Spielbeginn die Regeln erklärt. Zu diesem Zeitpunkt sind die \emph{Mr.~X/Y/Z} bereits auf
ihren Startpositionen, die sie der Spielleitung bereits mitgeteilt haben. 

\item  Wichtig ist auch der Hinweis, dass die Gruppen ihre Handynummern austauschen können, um gemeinsam auf "`Jagd"' zu gehen.

\item  Das Spiel beginnt damit, dass die Spielleitung den Gruppen die Startpositionen der \emph{Mr.}s mitteilt.

\item  Eine Viertelstunde nach dem Start ruft die Spielleitung bei den \emph{Mr.}s an und lässt sich die aktuellen Positionen
geben. Diese werden dann sofort als SMS an die Suchgruppen gesendet. 
Dies wiederholt die Spielleitung im Viertelstundentakt. Die \emph{Mr.}s geben auch an, wie oft sie bereits gefangen wurden. Diese Info
kann man den Gruppen zusätzlich mitgeben (Leistungsdruck). Auch geben die \emph{Mr.}s an, wenn eine Gruppe alle drei
gefangen und damit gewonnen hat. Je nach Zeitpunkt kann man das Spiel dann mit einer SMS beenden oder die nächsten
Plätze ausspielen lassen.

\item  Nach etwa eineinhalb Stunden kann man die Suchgruppen anrufen und hören, ob sie immer noch dabei sind. Bei
diesem Gespräch kann man sie noch ein wenig aufmuntern oder vielleicht mal eine Hilfe geben, wenn sie noch gar keinen
Erfolg hatten. Dies kann man nun auch jede Stunde machen, damit die \emph{Mr.}s nicht umsonst durch die Stadt laufen.

\item  Wenn das das Zeitlimit erreicht ist, wird das Spiel per SMS beendet. Je nach Erfolgsquote der Gruppen kann man schreiben,
dass alle zurück zum Start kommen sollen und die \emph{Mr.}s noch bis zur Tür der PH gejagt werden dürfen, da diese auch
dorthin zurück kommen sollen. Auch den \emph{Mr.}s muss das Ende des Spiels mitgeteilt werden.
\end{enumerate}

\paragraph{Besondere Hinweise:} Kümmert euch \emph{rechtzeitig} vorher darum, dass genügend Geld für die SMS auf dem Konto ist!
\paragraph{Wann einsetzen:} Um den Erstis das Nahverkehrssystem nahe zu bringen. Außerdem lernen sich die Leute bei diesem Spiel gegenseitig kennen.

